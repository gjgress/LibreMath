\documentclass{memoir}
\usepackage{notestemplate}

%\logo{~/School-Work/Auxiliary-Files/resources/png/logo.png}
%\institute{Rice University}
%\faculty{Faculty of Whatever Sciences}
%\department{Department of Mathematics}
%\title{Class Notes}
%\subtitle{Based on MATH xxx}
%\author{\textit{Author}\\Gabriel \textsc{Gress}}
%\supervisor{Linus \textsc{Torvalds}}
%\context{Well, I was bored...}
%\date{\today}

%\makeindex

\begin{document}

\subsection{Matrix Groups}
\label{sub:matrix_groups}

For each \(n \in \Z^{+}\), let \(GL_n(F)\) be the set of all \(n\times n\) matrices whose entries are from a field \(F\) and whose determinant is nonzero, i.e.
\begin{align*}
	GL_n(F) = \left\{A \mid A \text{ is an \(n\times n\) matrix with entries from \(F\) and } \textrm{det}(A) \neq 0 \right\} 
\end{align*}
The definitions of fields, matrices, and determinants can be read from a standard linear algebra resource. The product and sum of matrices forms a group operation, and because
\begin{align*}
	\textrm{det}(AB) = \textrm{det}(A) \textrm{det}(B)
\end{align*}
it follows that the set is closed under the grouo operation. Hence, \(GL_n(F)\) forms a group called the \textbf{general linear group of degree \(n\)}.

\subsection{Quaternion Group}
\label{sub:quaternion_group}

The \textbf{quaternion group \(Q_8\)} is defined by
\begin{align*}
	Q_8 = \left\{ 1,-1,i,-i,j,-j,k,-k \right\} 
\end{align*}
with the group operation computed as follows:
\begin{align*}
	1\cdot a = a\cdot 1 = a\\
	(-1) \cdot (-1) = 1, \quad (-1)\cdot a = a \cdot (-1) = -a
\end{align*}
for all \(a \in Q_8\), and
\begin{align*}
	i\cdot i = j\cdot j = k\cdot k = -1\\
	i\cdot j = k \quad j\cdot i = -k\\
	j\cdot k = i \quad k\cdot j = -i\\
	k\cdot i = j \quad i \cdot k = -j
\end{align*}
One can check as an exercise that this in fact satisfies the group axioms. Notice that \(Q_8\) is non-abelian of order \(8\).


% \printindex
\end{document}
