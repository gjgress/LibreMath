\documentclass{memoir}
\usepackage{notestemplate}

%\logo{~/School-Work/Auxiliary-Files/resources/png/logo.png}
%\institute{Rice University}
%\faculty{Faculty of Whatever Sciences}
%\department{Department of Mathematics}
%\title{Class Notes}
%\subtitle{Based on MATH xxx}
%\author{\textit{Author}\\Gabriel \textsc{Gress}}
%\supervisor{Linus \textsc{Torvalds}}
%\context{Well, I was bored...}
%\date{\today}

%\makeindex

\begin{document}

% \maketitle

% Notes taken on 05/31/21

\subsection{Generators and Relations}
\label{sec:generators_and_relations}

First, we will give an initial perspective of groups via generators and relations, and then revisit the symmetric group under this light.
\begin{defn}[Generator]
	A subset \(S\) of elements of a group \(G\) is called a \textbf{generator} if every element can be written as a finite product of elements (and their inverses) of \(S\). We denote this by \(\langle S \rangle =G\) and when this holds say that \(G\) is \textbf{generated by} \(S\).
\end{defn}
Of course, we need to know exactly what happens to these products of elements in order to get a structure for \(G\).

\begin{defn}[Relations and Presentations]
	Let \(G = \langle S \rangle \) be a group generated by \(S\). Any equations \(R_i\) that elements in the generator satisfy are called \textbf{relations}. If there is some collection of relations \(R_1,\ldots,R_m\) such that any relation among elements of \(S\) can be deduced from, we call these generators and relations a \textbf{presentation} of \(G\), denoted by
	\begin{align*}
		G = \langle S \mid R_1,R_2,\ldots,R_m \rangle .
	\end{align*}
\end{defn}

\begin{exmp}[Dihedral Group]
	The dihedral group \(D_{2n}\) can be viewed as a presentation given by
	\begin{align*}
		D_{2n} = \langle r,s \mid r^{n}=^2=1, \; rs = sr^{-1} \rangle .
	\end{align*}
\end{exmp}
Showing that a presentation is equivalent to a group structure is non-trivial. Relations can often be tricky, in that there are often "hidden" relations that arise from combining two relations in different ways. As a result, often the best way to show that a presentation indeed corresponds to our group is to show bounds on the order of the presentation, and show that it has the same order as the group desired. Then provided your group satisfies the relations, it is likely that it is the only group (of that order) that does so.



% \printindex
\end{document}
