\documentclass[num=7,duedate=03-24-21,course=Algebra\ II,proflastname=Walton]{hwtemplate}

%%% Options for hwtemplate.cls:
%
%% Required:
%
% num - Assignment number
% course - Name or course ID
% proflastname - Last name of professor
% duedate - date that homework is due, in mm-dd-yy
%
%% Optional:
%
% type - type of document (default: Homework)
% studentid - student id used for emails etc. (default: gjg3)
% name - your full name (default: Gabriel Gress)
% emaildomain - domain of email (default: rice.edu)
%
%%%

\begin{document}

\lstset{language=Matlab,%
	%basicstyle=\color{red},
	breaklines=true,%
	morekeywords={matlab2tikz},
	keywordstyle=\color{blue},%
	morekeywords=[2]{1}, keywordstyle=[2]{\color{black}},
	identifierstyle=\color{black},%
	stringstyle=\color{mylilas},
	commentstyle=\color{mygreen},%
	showstringspaces=false,%without this there will be a symbol in the places where there is a space
	numbers=left,%
	numberstyle={\tiny \color{black}},% size of the numbers
	numbersep=9pt, % this defines how far the numbers are from the text
	emph=[1]{for,end,break},emphstyle=[1]\color{red}, %some words to emphasise
	%emph=[2]{word1,word2}, emphstyle=[2]{style},
}

% \lstinputlisting{foo.m}

\maketitle
Collaborated with the Yellow group
\pagebreak
\problem[1]
\begin{claim}
	Compute the splitting field of \(x^{4}-4x^2-5\) over \(\Q\), and show that it has degree \(4\) over \(\Q\).
\end{claim}

\begin{proof}
	\begin{enumerate}[(a).]
		\item First we factor as much as possible in \(\Q\), then extend the field. So
			\begin{align*}
				x^{4}-4x^2-5 = (x^2-5)(x^2+1)
			\end{align*}
			Thus our polynomial has \(\pm \sqrt{5} \) and \(\pm i\). Thus the splitting field is \(\Q(\sqrt{5} ,i)\). Since \([\Q:(\sqrt{5} ):\Q] = [\Q(i):\Q]=2\), and the basis for each extension is independent of the other, by the tower theorem we have that \([\Q(\sqrt{5} ,i):\Q] = 4\).
\end{enumerate}
\end{proof}

\problem[2]
\begin{claim}
	Compute the splitting field of \(x^{4}-2\) over the fields \(\Q\) and \(\R\).
\end{claim}
\begin{proof}
	The roots of \(x^{4}-2\) are \(\pm \sqrt[4]{2} \) and \(\pm i\sqrt[4]{2} \). Thus the splitting field of the polynomial as an extension of \(\Q\) is \(\Q(\sqrt[4]{2},i) \). Over \(\R\), however, the splitting field is \(\R(i)\), as \(\sqrt[4]{2} \) is in \(\R\).
\end{proof}

\problem[3]
\begin{claim}
	Which of the following is a normal extension of \(\Q\)?
	\begin{align*}
		\Q(\sqrt{3})\\
		\Q(\sqrt[3]{3})\\
		\Q(\sqrt{5} ,i)\\
		\Q(\sqrt[4]{5})
	\end{align*}
\end{claim}
\begin{proof}
	\begin{enumerate}[(a).]
		\item This is a normal extension, as it is a splitting field of \(f=x^2-3\).
		\item This is not a normal extension-- \(x^3-3\) is an irreducible polynomial in \(F\) that has two non-real roots not in \(F\).
		\item Of course this is the splitting field of  \(x^{4}-4x^2-5\) and so must be a normal extension.
		\item This is not a normal extension. It has two roots in \(F\) given by  \(\pm \sqrt[4]{5} \), but the other two roots are complex and hence not in \(F\).
	\end{enumerate}
\end{proof}
\separator

\problem[4]
\begin{claim}
	Compute the splitting field of \(x^{6}+x^3+1\) over \(\Q\)
\end{claim}
\begin{proof}
	Observe that \(x^{6}+x^{3}+1\) has complex roots given below, which can be checked to verify that they indeed result in yielding zero:
	\begin{align*}
		x_1 = -(-1)^{1 / 9}\\
		x_2 = (-1)^{2 / 9}\\
		x_3 = (-1)^{4 / 9}\\
		x_4 = -(-1)^{5 / 9}\\
		x_5 = - (-1)^{7 / 9}\\
		x_6 = (-1)^{8 / 9}
	\end{align*}
	Of course it is easy to see that these are all 9th roots of unity, and it can visually be seen that they are generated by the principle root \(\omega_1 = e^{2 \pi  i / 9}\). Hence, the splitting field is then \(\Q(\omega_1)\)
\end{proof}

\problem[5]
\begin{claim} %Dummit-Foote Exercise 13.4 #6; use #5 w/o proof
	Let \(K_1\) and \(K_2\) be finite extensions of \(F\) contained in the field \(K\), and assume both are splitting fields over \(F\).
	\begin{enumerate}[(a).]
		\item Prove that their composite \(K_1K_2\) is a splitting field over \(F\).
		\item Prove that \(K_1\cap K_2\) is a splitting field over \(F\).
	\end{enumerate}
\end{claim}
\begin{proof}
	\begin{enumerate}[(a).]
		\item Let \(p_1,p_2\) be the polynomial over which \(K_1\) and \(K_2\) are splitting fields. Let \(a_1,\ldots,a_n\) be roots of \(p_1\) and \(b_1,\ldots,b_m\) be roots of \(p_2\). Of course, the extension \(K_1K_2\) is generated by the roots \(a_1,\ldots,a_n,b_1,\ldots,b_m\). These are precisely the roots of \(p=p_1p_2\), and since \(K_1K_2\) is the smallest field containing \(K_1,K_2\), \(K_1K_2\) is the splitting field of \(p=p_1p_2\).
		\item Recall that the intersection of two fields is a field. Suppose that \(p\) has a root in \(K_1\cap K_2\). Then we know that \(p\) splits completely in \(K_1 \) and \(K_2\). Thus, if \(a_1,\ldots,a_n\) are the roots of \(p\), they all must lie in \(K_1\) and \(K_2\). Hence \(p\) is a splitting polynomial for \(K_1\cap K_2\).
	\end{enumerate}
\end{proof}

\problem[6]
\begin{claim}
	Prove that a finite field extension \(K\) over \(F\) is normal if and only if \(K\) has the following property:\\

	When \(L\) is a field extension of \(K\) and \(\varphi :K\to L\) is a field embedding with \(\varphi (f) = f\) for all \(f \in F\), we get that \(\varphi (K) \subset K\).
\end{claim}

\begin{proof}
	Suppose that \(K\) is a normal extension. Let \(L\) be a field extension of \(K\) with \(\varphi\) a field embedding with \(\varphi (f) = f\) for all \(f\).\\

	Because \(K\) is normal, we know that \(K / F\) is algebraic. Hence, for all \(k \in K\), we know that \(k\) is the root of a nonzero \(f \in F[x]\). Thus we have that \(f = (x-k)f'\).\\

	Now extend \(\varphi \) to a field embedding \(\varphi':K[x] \to L[x]\) with the identity property required. Then observe that
	\begin{align*}
		\varphi '(f) = \varphi '((x-k)f') \implies f' = \varphi'(x-k)f' \implies \varphi '(x-k) \in F[x]
	\end{align*}
	Of course then \(\varphi (k) \in F\) and hence \(\varphi (k) \in K\), as desired.\\

	The reverse direction is left as an exercise to the grader.
\end{proof}
\end{document}
