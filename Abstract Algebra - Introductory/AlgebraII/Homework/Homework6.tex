\documentclass[num=6,duedate=03-17-21,course=Algebra\ II,proflastname=Walton]{hwtemplate}

%%% Options for hwtemplate.cls:
%
%% Required:
%
% num - Assignment number
% course - Name or course ID
% proflastname - Last name of professor
% duedate - date that homework is due, in mm-dd-yy
%
%% Optional:
%
% type - type of document (default: Homework)
% studentid - student id used for emails etc. (default: gjg3)
% name - your full name (default: Gabriel Gress)
% emaildomain - domain of email (default: rice.edu)
%
%%%

\begin{document}

\lstset{language=Matlab,%
	%basicstyle=\color{red},
	breaklines=true,%
	morekeywords={matlab2tikz},
	keywordstyle=\color{blue},%
	morekeywords=[2]{1}, keywordstyle=[2]{\color{black}},
	identifierstyle=\color{black},%
	stringstyle=\color{mylilas},
	commentstyle=\color{mygreen},%
	showstringspaces=false,%without this there will be a symbol in the places where there is a space
	numbers=left,%
	numberstyle={\tiny \color{black}},% size of the numbers
	numbersep=9pt, % this defines how far the numbers are from the text
	emph=[1]{for,end,break},emphstyle=[1]\color{red}, %some words to emphasise
	%emph=[2]{word1,word2}, emphstyle=[2]{style},
}

% \lstinputlisting{foo.m}

\maketitle
\pagebreak
\problem[1]
\begin{claim}
	Show that \(x^3+x+1\) is irreducible over \(\mathbb{F}_2\) and let \(\theta \) be a root. Compute the powers of \(\theta \) in \(\mathbb{F}_2(\theta )\).
\end{claim}

\begin{proof}

\end{proof}

\problem[2]
\begin{claim}
	Determine the minimal polynomial over \(\Q\) for the element \(1+i\).
\end{claim}
\begin{proof}
	
\end{proof}

\problem[3]
\begin{claim}
	Let \(\mathbb{F}\) be a finite field of characteristic \(p\). Prove that \(\left| \mathbb{F} \right| =p^{n}\) for some positive integer \(n\).
\end{claim}
\begin{proof}
	
\end{proof}

\problem[4]
\begin{claim}
	Determine the degree over \(\Q\) of \(2+\sqrt{3} \) and of \(1+\sqrt[3]{2} + \sqrt[3]{4} \).
\end{claim}
\begin{proof}
	
\end{proof}

\separator

\problem[5]
\begin{claim}
	Prove that \(x^{5}-ax-1 \in Z[x]\) is irreducible unless \(a=0,2,-1\). The first two correspond to linear factors, the third corresponds to the factorization \((x^2-x+1)(x^3+x^2-1)\).
\end{claim}
\begin{proof}
	
\end{proof}

\problem[6]
\begin{claim}
	Prove that \(\Q(\sqrt{2} +\sqrt{3} ) = \Q(\sqrt{2} ,\sqrt{3} )\). Conclude that \([\Q(\sqrt{2} +\sqrt{3} ) : Q ] = 4\). Find an irreducible polynomial satisfied by \(\sqrt{2} +\sqrt{3} \). % For the first part, consider (sqrt(2) + sqrt(3))^2 etc
\end{claim}
\begin{proof}
	
\end{proof}

\problem[7]
\begin{claim}
	Suppose the degree of the extension \(K / F\) is a prime \(p\). Show that any subfield \(E\) of \(K\) containing \(F\) is either \(K\) or \(F\).
\end{claim}
\begin{proof}
	
\end{proof}

\problem[8]
\begin{claim}
	Prove that if \([F(\alpha ):F]\) is odd then \(F(\alpha ) = F(\alpha ^2)\).
\end{claim}
\begin{proof}
	
\end{proof}

\end{document}
