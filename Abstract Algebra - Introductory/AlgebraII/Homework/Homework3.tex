\documentclass[num=3,duedate=02-24-21,course=Algebra\ II,proflastname=Walton]{hwtemplate}

%%% Options for hwtemplate.cls:
%
%% Required:
%
% num - Assignment number
% course - Name or course ID
% proflastname - Last name of professor
% duedate - date that homework is due, in mm-dd-yy
%
%% Optional:
%
% type - type of document (default: Homework)
% studentid - student id used for emails etc. (default: gjg3)
% name - your full name (default: Gabriel Gress)
% emaildomain - domain of email (default: rice.edu)
%
%%%

\begin{document}

\lstset{language=Matlab,%
	%basicstyle=\color{red},
	breaklines=true,%
	morekeywords={matlab2tikz},
	keywordstyle=\color{blue},%
	morekeywords=[2]{1}, keywordstyle=[2]{\color{black}},
	identifierstyle=\color{black},%
	stringstyle=\color{mylilas},
	commentstyle=\color{mygreen},%
	showstringspaces=false,%without this there will be a symbol in the places where there is a space
	numbers=left,%
	numberstyle={\tiny \color{black}},% size of the numbers
	numbersep=9pt, % this defines how far the numbers are from the text
	emph=[1]{for,end,break},emphstyle=[1]\color{red}, %some words to emphasise
	%emph=[2]{word1,word2}, emphstyle=[2]{style},
}

% \lstinputlisting{foo.m}

\maketitle
\pagebreak
\problem[1]
\begin{claim}
	Verify the module axioms for \(O = \textrm{Hom}_R(M,N)\) and conclude that \(O\) is an \(R\)-module. Furthermore, show that \(E = \textrm{End}_R(M)\) is a ring.
\end{claim}

\begin{proof}

\end{proof}

\problem[2]
\begin{claim}
	Read and prove the proposition on the iff on generating sets and bases.
\end{claim}

\problem[3]
\begin{claim}
	Let \(C_m = \langle g \mid g^{m}=e \rangle \) be the cyclic group of order \(m\). For an element \(A \in GL\)
\end{claim}

\separator

\end{document}
