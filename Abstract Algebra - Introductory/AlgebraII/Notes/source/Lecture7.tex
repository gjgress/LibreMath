\documentclass{memoir}
\usepackage{notestemplate}

%\logo{./resources/pdf/logo.pdf}
%\institute{Rice University}
%\faculty{Faculty of Whatever Sciences}
%\department{Department of Mathematics}
%\title{Class Notes}
%\subtitle{Based on MATH xxx}
%\author{\textit{Author}\\Gabriel \textsc{Gress}}
%\supervisor{Linus \textsc{Torvalds}}
%\context{Well, I was bored...}
%\date{\today}

\begin{document}

% \maketitle

% Notes taken on 02/08/21

\section{Quotient Modules and Isomorphism Theorems}
\label{sec:quotient_modules_and_isomorphism_theorems}

Given \(R\)-modules \(M\) and \(N\), an \(R\)-module homomorphism from \(M\) to \(N\) is a group homomorphism \(\varphi:M\to N\) 
\begin{align*}
	[\varphi(m +_M m') = \varphi(m) +_N \varphi(m') \quad \forall m,m' \in M]
\end{align*}
The set of all such \(R\)-module homomorphisms is denoted by \( \textrm{Hom}_R(M,N)\)

\begin{hw}
	If \(\varphi \in \textrm{Hom}_R(M,N)\) and \(\psi \in \textrm{Hom}_R(N,P)\) then \(\psi \circ \varphi \in \textrm{Hom}_R(M,P)\).
\end{hw}

\begin{prop}
	Let \(R\) be a ring, and take \(R\)-modules \(M\) and \(N\).
	\begin{enumerate}
		\item If \(\varphi,\psi \in \textrm{Hom}_R(M,N)\), then define
			\begin{align*}
				(\varphi+\psi)(m) := \varphi(m) + \psi(m) \quad \forall m \in M\\
				(r \varphi) (m) := r \varphi(m) \quad \forall r \in R, m \in M
			\end{align*}
			This gives \( \textrm{Hom}_R(M,N)\) the structure of an \(R\)-module.
		\item Denote \( \textrm{Hom}_R(M,M)\) by \( \textrm{End}_R(M)\). We get that \( \textrm{End}_R(M)\) is a ring with addition defined as above, and multiplication as defined in the exercise.
	\end{enumerate}
\end{prop}
There is a structure that combines being an \(R\)-module and a ring into \(R\)-algebras, but we will not discuss them in detail.

\begin{defn}
	Let \(R\) be a commutative ring with \(1_R\). A \textbf{(unital) \(R\)-algebra} is a unital ring \(A\) equipped with a unital ring homomorphism \(f:R\to A\) such that the subring \(f(R)\) ...
\end{defn}

\subsection{Quotient Modules}
\begin{prop}
	Take \(R\) as a ring, and \(M\) an \(R\)-module with \(R\)-submodule \(N\).
	\begin{enumerate}
		\item The quotient group \(M / N\) can be made into an \(R\)-module via
			\begin{align*}
				R\times M / N \to M / N\\
				(r,m+N) \mapsto rm + N \quad \forall r \in R, m+N \in M / N
			\end{align*}
		\item The canonical projection map of groups
			\begin{align*}
				\pi:M\to M / N\\
				m\mapsto m+N \quad \forall m \in M
			\end{align*}
			is a subjective \(R\)-module map, with \( \textrm{Ker}\pi = N\).
	\end{enumerate}
\end{prop}

\begin{thm}[Module Isomorphism Theorems]
	Take a ring \(R\), and \(M,N\) \(R\)-modules.
	\begin{enumerate}
		\item If \(\varphi \in \textrm{Hom}_R(M,N)\), then \( \textrm{ker}\varphi\) is a \(R\)-submodule of \(M\), and \(M / \textrm{ker}\varphi \cong \varphi(M)\) as \(R\)-modules.
		\item If \(A,B\) are \(R\)-submodules of \(M\), then \((A+B) / B \cong A / (A\cap B)\) as \(R\)-modules.
		\item If \(A\subset B\) are submodules of \(M\), then
			\begin{align*}
				(M / A) / (B / A) \cong M / B
			\end{align*}
			as \(R\)-modules.
		\item Suppose that \(N\) is an \(R\)-submodule of \(M\). Then \(\exists \) a bijection:
			\begin{align*}
				\left\{ \text{submodules } A \text{ of \(M\) containing \(N\)} \right\} \iff \left\{ \text{submodules \(A / N\) of \(M/N\)} \right\} .
			\end{align*}
	\end{enumerate}
\end{thm}

\end{document}
