\documentclass{memoir}
\usepackage{notestemplate}

%\logo{./resources/pdf/logo.pdf}
%\institute{Rice University}
%\faculty{Faculty of Whatever Sciences}
%\department{Department of Mathematics}
%\title{Class Notes}
%\subtitle{Based on MATH xxx}
%\author{\textit{Author}\\Gabriel \textsc{Gress}}
%\supervisor{Linus \textsc{Torvalds}}
%\context{Well, I was bored...}
%\date{\today}

\begin{document}

% \maketitle

% Notes taken on 01/25/21

\chapter{Modules}
\label{cha:modules}

\section{Ring Review}
\label{sec:ring_review}



Recall the definition of a ring.

\begin{defn}[Ring]
	A \textbf{ring} is a nonempty set \(R\) with two operations: addition and multiplication, which has the properties
	\begin{itemize}
		\item \(R^{+}\) is an abelian group
		\item \(R^{\times }\) is a semigroup
		\item \(\times \) distributes over \(+\) :
			\begin{align*}
				a(b+c) = ab+ac \quad (b+c)a = ba+ca
			\end{align*}
	\end{itemize}
\end{defn}

\begin{defn}[Types of Rings]
	A ring \(R\) is \textbf{commutative} if \(\times \) is commutative. Furthermore, we say a ring is \textbf{unital} if there exists an element \(1_R \in R\) so that \(1_R a = a_{1_R} = a\) for all \(a \in R\).\\

The ring \(R\) is \textbf{(in)finite} if it is (in)finite as a set.\\

A \textbf{subring} of \(R\) is an additive subgroup \(S\) of \(R\) that is closed under the multiplication of \(R\). Furthermore, if it contains \(1_R\) then it is a \textbf{unital subring}.\\

A \textbf{unit} of \(R\) is an element \(a \in R\) that has a multiplicative inverse, i.e. \(\exists b \in R\) so that \(ab = ba = 1_R\). The set of units of \(R\) is sometimes denoted by \(R^{\times }\).
\end{defn}

In order to better classify commutative rings, we introduce a few more terminology.

\begin{defn}[Irreducibles and Primes]
	We say that \(r \in R\) is \textbf{irreducible} if 
	\begin{align*}
		r = ab \implies a \text{ or }b \text{ is a unit}.
	\end{align*}
	We impose a stronger condition that \(r \in R\) is \textbf{prime} if 
	\begin{align*}
		r \mid ab \implies r\mid a \text{ or }r \mid b.
	\end{align*}

\end{defn}
Note that every prime is irreducible, but not vice versa.\\

\begin{figure}[ht]
    \centering
     \def\svgwidth{1\linewidth}
     \input{./figures/hierarchy-of-commutative-rings.pdf_tex}
    \caption{Hierarchy of Commutative Rings}
    \label{fig:hierarchy-of-commutative-rings}
\end{figure}

\end{document}
