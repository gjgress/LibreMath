\documentclass{memoir}
\usepackage{notestemplate}

%\logo{~/School-Work/Auxiliary-Files/resources/png/logo.png}
%\institute{Rice University}
%\faculty{Faculty of Whatever Sciences}
%\department{Department of Mathematics}
%\title{Class Notes}
%\subtitle{Based on MATH xxx}
%\author{\textit{Author}\\Gabriel \textsc{Gress}}
%\supervisor{Linus \textsc{Torvalds}}
%\context{Well, I was bored...}
%\date{\today}

\begin{document}

% \maketitle

% Notes taken on ??

\section{Minimal Polynomials}
\label{sec:minimal_polynomials}

\begin{prop}
	Let \(\alpha \) be an algebraic element over \(F\).
	\begin{enumerate}[(a).]
		\item Then there exists a monic irreducible polynomial of minimal degree \(m_{\alpha ,F}(x) \in F[x]\) which has \(\alpha \) as a root.
		\item A polynomial \(f(x) \in F[x]\) has \(\alpha \) as a root if and only if \(m_{\alpha ,F}(x) \mid f(x)\) in \(F[x]\).
		\item The polynomial \(m_{\alpha ,F}(x)\) with the property in (a) is unique.
	\end{enumerate}
\end{prop}
We can see the minimal polynomial must be irreducible, because otherwise one of its factors would have \(\alpha \) as a root and hence has degree smaller than \(m_{\alpha ,F}(x)\), contradicting our hypothesis. The divisibility \(m_{\alpha ,F}(x) \mid f(x)\) follows from the division algorithm in \(F[x]\). The divisibility and minimality conditions together give uniqueness.
\begin{cor}
	If \(K / F\) is a field extension, and \(\alpha \) is algebraic over both \(F\) and \(K\), then \(m_{\alpha ,K}(x)\) divides \(m_{\alpha ,F}(x)\) in \(K[x]\).
\end{cor}
This directly follows as \(m_{\alpha ,F}(x)\) has a root \(\alpha \) in \(K\) and hence (b) gives us divisibility.
\begin{defn}
	The polynomial \(m_{\alpha ,F}(x)\) is called the \textbf{minimal polynomial of \(\alpha \) over \(F\)}. The degree of \(m_\alpha (x)\) is called the \textbf{degree of \(\alpha \)}.\\

In other words, the minimal polynomial of \(\alpha \) over \(F\) is a monic irreducible polynomial over \(F\) that has \(\alpha \) as a root. Alternatively, it is a monic polynomial over \(F\) of minimal degree with \(\alpha \) as a root-- both imply the other.
\end{defn}

\begin{prop}
	Let \(\alpha \) be algebraic over \(F\). Then
	\begin{align*}
		F(\alpha ) \cong F[x] / (m_{\alpha }(x))
	\end{align*}
	So that \([F(\alpha ):F] = \textrm{deg}m_{\alpha }(x) \equiv \textrm{deg}\alpha\).
\end{prop}

\begin{prop}
	An element \(\alpha  \in F\) is algebraic over \(F\) if and only if the simple extension \(F(\alpha ) / F\) is finite.\\

	If \(\alpha \in K\) with \([K:F] = n\), then \(\textrm{deg}(\alpha) \leq n\).
\end{prop}
This follows by applying linear dependence to powers \(\alpha^i\) with \(i = 0,1,\ldots,n\).
\begin{cor}
	If \(K / F\) is finite, then \(K / F\) is algebraic.
\end{cor}

\begin{exmp}
	Take \(F\) to be a field with \(\textrm{char}(F) \neq 2\). Consider \(K/F\) of degree 2, which is hence algebraic. Let \(\alpha  \in K / F\) so that \(\alpha \) is a root of a polynomial over \(F\) of degree 1 or 2. Because \(\alpha \not\in F\), the polynomial must has degree 2.\\

	This implies that \(m_{\alpha ,F}(x) = x^2+bx+c\) for \(b,c \in F\). This implies that \(F(\alpha )\) has the same dimension of \(K\) and hence \(K = F(\alpha )\) (as \(K\) is a field extension of \(F(\alpha )\). This implies that \(K = F(\sqrt{b^2-4ac} )\) and so any degree 2 extension of a field \(F\) with characteristic not equal to \(2\) is of the form \(F(\sqrt{D} )\) for \(D\) a non-square element of \(F\).\\

	Conversely, for such a field, \([F(\sqrt{D} ) : F] = 2\) and hence extensions of the form \(F(\sqrt{D} ) / F\) are called \textbf{quadratic extensions of \(F\)}.
\end{exmp}
\end{document}
