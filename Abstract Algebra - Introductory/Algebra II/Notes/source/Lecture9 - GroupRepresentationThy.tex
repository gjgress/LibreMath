\documentclass{memoir}
\usepackage{notestemplate}

%\logo{~/School-Work/Auxiliary-Files/resources/png/logo.png}
%\institute{Rice University}
%\faculty{Faculty of Whatever Sciences}
%\department{Department of Mathematics}
%\title{Class Notes}
%\subtitle{Based on MATH xxx}
%\author{\textit{Author}\\Gabriel \textsc{Gress}}
%\supervisor{Linus \textsc{Torvalds}}
%\context{Well, I was bored...}
%\date{\today}

\begin{document}

% \maketitle

% Notes taken on 02/12/21


If, when taking an \(R\)-module \(M\), we may work over a field \(K\) and modify \(M = V\) to be a \(K\)-vector space by \(K\times V\to V\). This then gives us that an \(R\)-module over \(V\) is a pair \(V\) with \(R\times V\to V\).

\begin{defn}[Group Module]
	Let \(G\) be a group. We say that a \(K\)-vector space \(V\) is a \textbf{\(G\)-module} if it comes equipped with a \(G\)-action map
	\begin{align*}
		G\times V \to V \quad (g,v) \mapsto g*v :=gv
	\end{align*}
	compatible with operations of \(G\) and \(V\) :
	\begin{enumerate}[(a).]
		\item \(e_G v = v\) 
		\item \((gh)v = g(hv)\) 
		\item \(g(v+w) = gv + gw\) 
		\item \(g(\lambda v) = \lambda (gv)\)
	\end{enumerate}
	for all \(g,h \in G\), \(v \in V\), \(\lambda \in K\).
\end{defn}
If one is given a \(G\)-module \(V\), then there is a natural group homomorphism
\begin{align*}
	\rho:G\to \textrm{End}_K(V)\\
	g\mapsto \left[ \rho g : V \to V , \; v\mapsto g * v := gv \right] 
\end{align*}
The image of \(\rho\) is inside of \(GL(V)\).

\begin{defn}
	A \textbf{\(K\)-linear representation of a group \(G\)} is a \(K\)-vector space \(V\) equipped with a group homomorphism \(\rho:G \to GL(V)\).
\end{defn}

% Save recap

\begin{defn}
	Given a representation of a group \(G\), \((V:\rho)\), its \textbf{degree} is \(\textrm{dim}V\).
\end{defn}
Note that when \(\textrm{dim}_KV=n\), we get that
\begin{align*}
	GL(V) \cong GL_n(K)
\end{align*}
as groups.

% Something about V basis and maps between them?

Representations of \(G\) of degree \(n\) over a field \(K\) are congruent to group homomorphisms \(\rho:G\to GL_n(K)\).

\begin{defn}
	For any group \(G\), the \textbf{trivial representation of \(G\) over \(K\)} is \((V=K, \rho:G\to GL_1(K)=K^{\times })\) given by \(g\mapsto 1_K\) for all \(g \in G\).
\end{defn}

% Examples

\begin{defn}
	Let \(\rho:G\to GL(V)\) and \(\rho':G\to GL(V')\) be two representatives of a group \(G\). We say that \(p\) and \(p'\) are \textbf{equivalent} or isomorphic if there exists an invertible linear transformation
	\begin{align*}
		\tau:V\to V'
	\end{align*}
	so that \(\tau\) \textbf{intertwines} with action of \(G\) :
	\begin{align*}
		\tau(\rho g(v)) = \rho'g(\tau(v)) \quad \forall g \in G, v \in V
	\end{align*}
\end{defn}

\begin{rmrk}
	\(\rho:G\to GL_n(K)\) and \(\rho':G\to GL_{n'}(K)\) are equivalent if and only if \(n = n'\) and \(\exists T \in GL_n(K)\) such that \(T\rho(g)T^{-1} = \rho'(g)\) for all \(g \in G\).
\end{rmrk}
This notion will be captured more clearly later with homomorphism/isomorphisms.
\end{document}
