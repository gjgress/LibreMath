\documentclass{memoir}
\usepackage{notestemplate}

%\logo{~/School-Work/Auxiliary-Files/resources/png/logo.png}
%\institute{Rice University}
%\faculty{Faculty of Whatever Sciences}
%\department{Department of Mathematics}
%\title{Class Notes}
%\subtitle{Based on MATH xxx}
%\author{\textit{Author}\\Gabriel \textsc{Gress}}
%\supervisor{Linus \textsc{Torvalds}}
%\context{Well, I was bored...}
%\date{\today}

\begin{document}

% \maketitle

% Notes taken on 05/10/21

\chapter{Galois Theory}
\label{cha:galois_theory}

Galois theory studies the connection between finite field extensions via roots of polynomials and the structures of groups that permute those roots.\\

Let \(F,K\) be fields, and \(K / F\) a field extension.

\begin{defn}[Field Automorphism]
	We say that \(\sigma :K\to K\) is a \textbf{field automorphism} if \(\sigma \) is a bijective unital ring homomorphism. We denote the collection of field automorphisms of \(K\) by \(\textrm{Aut}(K)\).\\

	An automorphism \(\sigma  \in \textrm{Aut}(K)\) \textbf{fixes an element} \(\alpha  \in K\) if \(\sigma (\alpha )=\alpha \).\\

	An automorphism \(\sigma \in \textrm{Aut}(K)\) \textbf{fixes a subset \(E\) of \(K\)} if \(\sigma (\alpha ) = \alpha \) for all \(\alpha  \in E\).\\

	For \(\sigma  \in \textrm{Aut}(K)\) and \(E\subset K\), \(\sigma (E)\) denotes the subset \(\left\{\sigma (\alpha ) \mid \alpha \in E \right\} \)
\end{defn}
Recall that the prime subfield of a field \(K\) is given by
\begin{align*}
	K_{\textrm{prime}} = \begin{cases}
		\Q & K \text{ has characteristic 0}\\
		\Z_p & \text{\(p\) prime}
	\end{cases}
\end{align*}
because \(\sigma \in \textrm{Aut}(K)\) fixes \(1_K\), it must hold that \(\sigma \) fixes \(K_{\textrm{prime}}\) and hence prime subfields are fixed by any automorphism of a field.

\section{Automorphisms fixing subfields}
\label{sec:automorphisms_fixing_subfields}

\begin{defn}
	We define \(\textrm{Aut}(K / F)\) to be the collection of automorphisms of \(K\) that fix \(F\).
\end{defn}

\begin{prop}
	\(\textrm{Aut}(K)\) is a group under composition, and \(\textrm{Aut}(K / F)\) is a subgroup of \(\textrm{Aut}(K)\).
\end{prop}

\begin{prop}
	Let \(\alpha  \in K\) be an algebraic element over \(F\). Then for any \(\alpha \in \textrm{Aut}(K / F)\), we get that \(m_{\alpha ,F}(\sigma (\alpha )) = 0\).
\end{prop}
In other words, automorphisms permute roots of minimal polynomials.

\section{Subfields and Subgroups}
\label{sec:subfields_and_subgroups}

\begin{prop}
	Let \(H\) be a subgroup of \(\textrm{Aut}(K)\). Then 
	\begin{align*}
		\left\{\alpha  \in K \mid \sigma (\alpha ) = \alpha \quad \forall \sigma \in H \right\} 
	\end{align*}
	is a subfield of \(K\). We call this subfield the \textbf{fixed field of \(H\)} denoted by \(K^{H}\).
\end{prop}

In fact, this structure induces a correspondence between field extensions and chains of subgroups.
\begin{prop}
	Let \(F_1 \subset F_2 \subset K\) be a sequence of field extensions. Then \(\textrm{Aut}(K / K) = \textrm{Id}_{\textrm{Aut}(K)} \leq \textrm{Aut}(K / F_2) \leq \textrm{Aut}(K / F_1)\).\\

	Conversely, let \(H_1 \leq H_2 \leq \textrm{Aut}(K)\) be a chain of subgroups. Then \(K^{\textrm{Aut}(K)} = K_{\textrm{prime}} \subset K^{H_2} \subset K^{H_1}\)
\end{prop}
\end{document}
