\documentclass{memoir}
\usepackage{notestemplate}

%\logo{~/School-Work/Auxiliary-Files/resources/png/logo.png}
%\institute{Rice University}
%\faculty{Faculty of Whatever Sciences}
%\department{Department of Mathematics}
%\title{Class Notes}
%\subtitle{Based on MATH xxx}
%\author{\textit{Author}\\Gabriel \textsc{Gress}}
%\supervisor{Linus \textsc{Torvalds}}
%\context{Well, I was bored...}
%\date{\today}

\begin{document}

% \maketitle

% Notes taken on 02/10/21

\section{Generating Sets}
\label{sec:generating_sets}

\begin{defn}
	Take an \(R\)-module \(\prescript{}{R}M\) and a subset \(X \subset M\).
	\begin{enumerate}
		\item The \textbf{\(R\)-submodule generated by \(X\)} is
			\begin{align*}
				RX = \left\{r_1x_1+\ldots+r_mx_m \mid r_i \in R, \; x_i \in X, \; m \in \Z>0 \right\} .
			\end{align*}
			The set \(X\) is called the \textbf{generating set} of \(RX\).
		\item A \(R\)-submodule \(\prescript{}{R}N\) of \(\prescript{}{R}M\) is \textbf{finitely generated} if \(N = RX\) for \(\left| X \right| <\infty\) and \textbf{cyclic} if \(N=RX\) for \(\left| X \right| =1\).
	\end{enumerate}
\end{defn}

\subsection{Free Modules}
\label{sub:free_modules}

\begin{defn}[Linear independence by \(R\)-modules]
	We say that \(X = \left\{x_1,\ldots,x_n \right\} \) is \textbf{\(R\)-linearly independent} if
	\begin{align*}
		r_1x_1+\ldots+r_nx_n = 0 \implies r_i = 0 \quad \forall i = 1,\ldots,n
	\end{align*}
\end{defn}

\begin{defn}
	We say that an \(R\)-module \(\prescript{}{R}M\) is \textbf{free on the subset \(X\)} of \(M\) if
	\begin{align*}
		M = RX\\
		X \text{ is \(R\)-linearly independent}
	\end{align*}
In this case, we call \(X\) the \textbf{basis} of \(\prescript{}{R}M\), and sometimes denote \(\prescript{}{R}M\) by \(F_R(X)\).\\

If \(R\) is commutative, then we call \(\left| X \right| \) the \textbf{rank} of \(\prescript{}{R}M\).
\end{defn}
This illustrates a key difference between vector spaces and modules-- vector spaces are always free, while modules need not be.

\begin{exmp}[Free and non-free modules]
Most modules have no basis! A free \(\Z\)-module is also called a \textbf{free abelian group}; lattices in \(\R^2\) are free abelian groups, while finite, non-zero abelian groups are not free.
\end{exmp}

\begin{defn}[R-Matrix]
	Let \(R\) be a ring. An \textbf{\(R\)-matrix} is a matrix whose entries are in \(R\). An \textbf{invertible \(R\)-matrix} is an \(R\)-matrix that has an inverse that is also an \(R\)-matrix. The \(n \times n\) invertible \(R\)-matrices form a group called the \textbf{general linear group over \(R\)}:
	\begin{align*}
		GL_n(R) = \left\{n\times n \text{ invertible \(R\)-matrices} \right\} .
	\end{align*}
	The \textbf{determinant} of an \(R\)-matrix \(A = (a_{ij})\) is defined in the usual way
	\begin{align*}
		\textrm{det}(A) = \sum_{p} \pm a_{1,p 1}\ldots a_{n, pn}.
	\end{align*}
	or the sum over all permutations of the indices and the sign being the sign of the permutation. Of course, all the usual properties of determinants hold for \(R\)-matrices.
\end{defn}

\begin{lemma}
	Le \(R\) be a non-zero ring. Then a square \(R\)-matrix \(A\) is invertible if and only if it has either a left inverse or a right inverse, and only if its determinant is a unit of the ring. Furthermore, an invertible \(R\)-matrix is square.
\end{lemma}

\begin{prop}[Free modules and \(R\)-matrices]
	Let \(R\) be a non-zero ring. Then the matrix \(P\) of a change of basis in a free module is an invertible \(R\)-matrix. Furthermore, any two bases of the same free module over \(R\) have the same cardinality.
\end{prop}
Every homomorphism \(f\) between two free modules is given by left multiplication by an \(R\)-matrix.

\begin{thm}[Universal Property of Free Modules]
	For any set \(A\) there is a free \(R\)-module \(F_R(A)\) on the set \(A\) and \(F_R(A)\) satisfies the \textit{universal property}: if \(\prescript{}{R}M\) is any \(R\)-module and \(\varphi :A\to M\) is any map of sets, then there is a unique \(R\)-module homomorphism \(\Phi:F_R(A) \to M\) such that \(\Phi (a) = \varphi (a)\) for all \(a \in A\). In other words, the following diagram commutes:
\begin{center}
			\begin{tikzpicture}
  \matrix (m)
    [
      matrix of math nodes,
      row sep    = 3em,
      column sep = 4em
    ]
    {
	    A & F_R(A) \\
	     & M            \\
    };
  \path
  (m-1-2) edge [->] node [right] {\(\Phi \)} (m-2-2)
    (m-1-1.east |- m-1-2)
    edge [->] node [above] {inclusion} (m-1-2)
      (m-1-1) edge [->] node [below] {$\varphi$} (m-2-2);
\end{tikzpicture}
\end{center}
Furthermore, if \(A = \left\{ a_1,\ldots,a_n \right\} \), then
\begin{align*}
	F_R(A) = Ra_1 \oplus Ra_2 \oplus \ldots \oplus Ra_n \stackrel{R}{\cong} R^{n}
\end{align*}
\end{thm}
This corresponds to the notion of free groups from group theory.

\begin{hw}
	If \(F_R,F_R'\) are free modules on the same set \(A\), there is a unique isomorphism between \(F_R\) and \(F_R'\) which is the identity map on \(A\).\\

	If \(\prescript{}{R}F\) is any free \(R\)-module with basis \(A\), then \(\prescript{}{R}F \cong F_R(A)\).
\end{hw}
If we have a free \(R\)-module with a basis \(A\), the above statement says that we can define \(R\)-module homomorphisms from the free module into other \(R\)-modules by simply specifying how the homomorphism acts on elements of \(A\).\\

The free module \(F_\Z(A)\) is called the \textbf{free abelian group on \(A\)}. If \(A\) is finite, then we say it is of \textbf{rank \(\left| A \right| \)} and is isomorphic to
\begin{align*}
	\Z \oplus \ldots^{n} \oplus \Z.
\end{align*}

%---
%
%diagonalizaton of modules etc
%
%---

\end{document}
