\documentclass{memoir}
\usepackage{notestemplate}

%\logo{./resources/pdf/logo.pdf}
%\institute{Rice University}
%\faculty{Faculty of Whatever Sciences}
%\department{Department of Mathematics}
%\title{Class Notes}
%\subtitle{Based on MATH xxx}
%\author{\textit{Author}\\Gabriel \textsc{Gress}}
%\supervisor{Linus \textsc{Torvalds}}
%\context{Well, I was bored...}
%\date{\today}

\begin{document}

% \maketitle

% Notes taken on 02/05/21

Recall that for a ring \(R\), a \textbf{(left) \(R\)-module} \(M\) is a pair
\begin{align*}
	M:= (M, R\times M\to M)\\
	(r,m) \mapsto rm
\end{align*}
where \(R\)-action map is compatible with \(+_M, +_R, \times_R\).

\section{Substructures of Modules}
\label{sec:substructures_of_modules}

\begin{defn}[Submodule]
	Take a ring \(R\) and a left \(R\)-module \(M\). A \textbf{\(R\)-submodule of \(M\)} is a subgroup \(N\) of \(M\) so that we get a left \(R\)-action on \(N\) via \(R\times M\to M\).\\

	In other words, it is a subgroup with closure under the \(R\)-action.
\end{defn}

\begin{prop}[Submodule Criterion]
	Take a ring \(R\) with \(1_R\), and left \(R\)-module \(M\). A subset \(N\) of \(M\) is a \(R\)-submodule of \(M\) if and only if
	\begin{itemize}
		\item \(N \neq \emptyset\) and
		\item \(n+rn' \in N\) for all \(r \in R\), \(n,n' \in N\).
	\end{itemize}
\end{prop}

\begin{prop}
	Let \(M\) be an \(R\)-module, and let \(N_{i}\) with \(i \in I\) be \(R\)-submodules of \(M\). Then
	\begin{enumerate}
		\item \(\bigcap_{i \in  I} N_i\) is an \(R\)-submodule of \(M\) 
		\item \(\bigcup_{i \in  I} N_i\) is not necessarily an \(R\)-submodule of \(M\)
		\item If \(N_1\subset N_2\subset N_3\subset \ldots\) is an increasing chain of \(R\)-submodules of \(M\), then \(\bigcup_{i \in \N} N_i\) is an \(R\)-submodule of \(M\)
		\item Let \(N_1+N_2 = \left\{n_1+n_2 \mid n_1 \in N_1, n_2 \in N_2 \right\} \) be the sum of \(N_1\) and \(N_2\). Then \(N_1+N_2\) is an \(R\)-submodule of \(M\)
	\end{enumerate}
\end{prop}

\section{R-module homomorphisms}
\label{sec:r_module_homomorphisms}

\begin{defn}
	Let \(R\) be a ring, and let \(M\) and \(N\) be \(R\)-modules. An \textbf{\(R\)-module homomorphism} is a group homomorphism
	\begin{align*}
		\varphi:M\to N \quad [\varphi(m+m') = \varphi(m) + \varphi(m')]
	\end{align*}
	so that \(R\)-action is preserved
	\begin{align*}
		[\varphi(rm) = r\varphi(m)]
	\end{align*}
	for all \(r \in R\), \(m,m' \in M\).
\end{defn}
The set of \(R\)-module homomorphisms from \(M\) to \(N\) is denoted by \( \textrm{Hom}_R(M,N)\).\\

An \(R\)-module isomorphism is a bijective \(R\)-module homomorphism.\\

If \(\varphi \in \textrm{Hom}_R(M,N)\), then the kernel of \(\varphi\) is \( \textrm{ker}\varphi = \left\{m \in M \mid \varphi(m) = 0 \right\} \). The image of \(\varphi\) is \( \textrm{Im}(\varphi) = \left\{\varphi(m) \mid m \in M \right\} \).\\

Recall that any group homomorphism between abelian groups can be represented as a \(\Z\)-module homomorphism. \(\left\{  \right\} \)
\end{document}
