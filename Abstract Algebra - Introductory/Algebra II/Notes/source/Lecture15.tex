\documentclass{memoir}
\usepackage{notestemplate}

%\logo{~/School-Work/Auxiliary-Files/resources/png/logo.png}
%\institute{Rice University}
%\faculty{Faculty of Whatever Sciences}
%\department{Department of Mathematics}
%\title{Class Notes}
%\subtitle{Based on MATH xxx}
%\author{\textit{Author}\\Gabriel \textsc{Gress}}
%\supervisor{Linus \textsc{Torvalds}}
%\context{Well, I was bored...}
%\date{\today}

\begin{document}

% \maketitle

% Notes taken on 03/08/21
% Recall that we denote by \(F(\theta )\) the field \(F\) adjoined by the root \(\theta \), which is spanned by powers of \(\theta \) as an \(F\)-vector space.
\begin{thm}
	Let \(F\) be a field and \(p(x) \in F[x]\) be an irreducible polynomial. Then \(\exists \) a field extension \(K\) of \(F\) in which \(p(x)\) has a root.
\end{thm}
This field is given by \(K := F[x] / (p(x))\), but we will show this more formally later.

\begin{thm}
	Let \(p(x) \in F[x]\) be an irreducible polynomial of degree over \(F\), and let \(K\) be the field \(F[x] / (p(x))\). Take \(\theta := x + (p(x))\) (root of \(p(x)\) ). Then
	\begin{enumerate}
		\item The elements \(\left\{ 1_F, \theta , \theta ^2, \ldots, \theta ^{n-1} \right\} \) are an \(F\)-vector space basis of the \(F\)-vector space \(K\).
		\item \([K:F] = n\) 
		\item \(K = \left\{a_0 + a_1\theta + a_2\theta^2 + \ldots + a_{n-1}\theta ^{n-1} \mid a_0,\ldots,a_{n-1} \in F \right\} \) as an \(F\)-vector space.
	\end{enumerate}
\end{thm}
Another nice example to be familiar with is \(K = \mathbb{F}_2[x] / (x^2+x+1)\). This is a field extension of \(\mathbb{F}_2\) as \(x^2+x+1\) is irreducible in \(\mathbb{F}_2\). We can see that \([\mathbb{F}_2[x] / (x^2+x+1) : \mathbb{F}_2[x]] = 2\) simply because the degree of the polynomial is \(2\), but we can also directly count elements in the set and see that it has twice the elements of \(\mathbb{F}_2[x]\).\\

Now let's define fields formed by adjoining roots more formally.

\begin{defn}
	Let \(K / F\) be a field extension, and let \(\alpha_1,\alpha_2,\ldots \in K\) be elements. The smallest subfield of \(K\) containing both \(F\) and the elements \(\alpha_1,\alpha_2,\ldots,\) denoted \(F(\alpha_1,\alpha_2,\ldots)\) is called the \textbf{field generated by \(\alpha_1, \alpha_2,\ldots\) over \(F\)}.
\end{defn}

\begin{defn}
	The field \(F(\alpha )\) generated by a single element \(\alpha \) over \(F\) is called a \textbf{simple extension of \(F\)}, and the element \(\alpha \) in this case is called \textbf{primitive}.
\end{defn}

\begin{thm}
	Let \(F\) be a field and let \(p(x) \in F[x]\) be an irreducible polynomial. Suppose \(K\) is an extension of \(F\) containing a root \(\alpha \) of \(p(x)\). Then \(F[x] / (p(x)) \cong F(\alpha )\).
\end{thm}
\end{document}
