\documentclass{memoir}
\usepackage{notestemplate}

% \begin{figure}[ht]
%     \centering
%     \incfig{riemmans-theorem}
%     \caption{Riemmans theorem}
%     \label{fig:riemmans-theorem}
% \end{figure}

\begin{document}
\section{Subring, Ideals, Factor rings, Ring homomorphisms}	
\begin{defn}[Subring]
	A subring is a subset \(S\) of a ring \(R\) which is a ring under the restriction of the operations in \(R\). Notationally, we say that \(S\leq R\).
\end{defn}
\(\emptyset \neq S\) is a subring if and only if \(a,b \in S \implies a+b, ab, -a \in S\) which is equivalent to \(a,b \in S \implies a-b, ab \in S\).
\begin{defn}[Ideal]
	An ideal is a subring \(I \leq R\) which is closed under multiplication with elements of \(R\). Notationally, we say that \(I \triangleleft R\).
\end{defn}
\(\emptyset\neq I\) is an ideal if and only if \(a,b \in I \implies a-b \in I\), \(a \in I, r \in R \implies ar, ra \in I\).\\
Note: a field has only trivial ideals.\\

\begin{defn}[Principal Ideal]
	If \(R\) is commutative and has an identity, then the principal ideal generated by \(c\) is the ideal \((c) = \left\{rc \mid r \in R \right\} \). 
\end{defn}
This is the smallest ideal containing \(c\).
\end{document}
