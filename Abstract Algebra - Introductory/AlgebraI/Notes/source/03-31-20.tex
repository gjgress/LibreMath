\documentclass{memoir}
\usepackage{notestemplate}

% \begin{figure}[ht]
%     \centering
%     \incfig{riemmans-theorem}
%     \caption{Riemmans theorem}
%     \label{fig:riemmans-theorem}
% \end{figure}

\begin{document}

\begin{defn}[Normal Subgroup]
	A subgroup \(N\) of \(G\) is \textbf{normal} if the left and right cosets are the same; i.e.
\begin{align*}
	N\leq G \text{ and }gN = Ng
\end{align*}
for every \(g \in G\). We denote this by \(N \triangleleft G\).
\end{defn}
We can also think of this as \(Ng \subset gN\) and \(gN \subset Ng\), or in other words, \(g^{-1}ng \in N\) and \(gng^{-1} \in N\).
\begin{align*}
	N \triangleleft G \iff N \leq G \text{ and } g^{-1}ng \in N \text{for every }g \in G, n \in N
\end{align*}
We call \(g^{-1}ng\) a \textbf{conjugate} of \(n\).\\

Observe that, because the trivial subgroups are trivially normal, then all subgroups \(H\) of index 2 are normal.

\end{document}
