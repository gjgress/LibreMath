\documentclass{memoir}
\usepackage{notestemplate}

% \begin{figure}[ht]
%     \centering
%     \incfig{riemmans-theorem}
%     \caption{Riemmans theorem}
%     \label{fig:riemmans-theorem}
% \end{figure}

\begin{document}
\section{Direct Sums of Rings}
\label{sec:direct_sums_of_rings}
\begin{defn}[Direct Sum]
	The \textbf{direct sum} of two rings \(R_1\bigoplus R_2\) (or direct product \(R_1\times R_2\)) is the ring of all ordered pairs \((r_1,r_2)\), with \(r_i \in R_i\), with addition and multiplication defined by adding and multiplying the components:
	\begin{align*}
		(r_1,r_2) + (s_1,s_2) = (r_1+s_1,r_2+s_2) \quad (r_1,r_2)(s_1,s_2) = (r_1s_1,r_2s_2).
	\end{align*}
\end{defn}
One can check that this indeed satisfies the properties of a ring.
\begin{defn}[Projection]
	The \textbf{projection} subsets are defined by
	\begin{align*}
		R_1^* := \left\{(r_1,0) \mid r_1 \in R_1 \right\} \\
		R_2^* := \left\{(0,r_2) \mid r_2 \in R_2 \right\} 
	\end{align*}
\end{defn}
These projection subsets are isomorphic to \(R_1,R_2\) respectively. Furthermore, \(R_1^{*},R_2^{*}\) are ideals in \(R_1\bigoplus R_2\), and every \(c \in R_1 \bigoplus R_2\) has a unique decomposition \(c = c_1 + c_2\) with \(c_i \in R_i\).
\begin{prop}[Unique Decompositions]
	If \(I,J\) are ideals in a ring \(R\) such that every \(r \in R\) can be uniquely written as \(r = i + j\), with \(i \in I\), \(j \in J\), then \(R \cong I\bigoplus J\)
\end{prop}

\end{document}
