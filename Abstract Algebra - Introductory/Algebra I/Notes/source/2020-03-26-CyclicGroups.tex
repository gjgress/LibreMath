\documentclass{memoir}
\usepackage{notestemplate}

%\logo{~/School-Work/Auxiliary-Files/resources/png/logo.png}
%\institute{Rice University}
%\faculty{Faculty of Whatever Sciences}
%\department{Department of Mathematics}
%\title{Class Notes}
%\subtitle{Based on MATH xxx}
%\author{\textit{Author}\\Gabriel \textsc{Gress}}
%\supervisor{Linus \textsc{Torvalds}}
%\context{Well, I was bored...}
%\date{\today}

%\makeindex

\begin{document}

% \maketitle

% Notes taken on 

\section{Cyclic Groups}
\label{sec:cyclic_groups}
One particular important class of groups are cyclic groups. 

\begin{defn}[Cyclic Group]
	A group \(G\) is \textbf{cyclic} if \(G\) can be generated by a single element. That is, there is some element \(g \in G\) such that
	\begin{align*}
		G = \left\{ g^{n} \mid n \in \Z \right\} .
	\end{align*}
	We write a cyclic group by \(G = \langle g \rangle \).
\end{defn}
It is not necessarily true that all powers of \(g\) are distinct (to see this, observe that in a finite group, if \(g^{n}=1\), then \(g^{n+k}= g^{k}\)). However, it is true that all cyclic groups are abelian (due to the law of exponents).

\begin{prop}
	Let \(G = \langle g \rangle \). Then \(\left| G \right| = \left| g \right| \). That is, the order of the group is the order of the generator-- if the order is infinite, then the group must also be infinite.
\end{prop}
Ideas from cyclic groups are useful in the general case for groups as well. For example, if \(G\) is any group, and \(g^{n}=g^{m}=1\) for some \(m,n \in \Z\), then \(g^{(m,n)}=1\). This actually gives us the following theorem:

\begin{thm}
	Any two cyclic groups of the same order are isomorphic. In particular, if \(\langle g \rangle \) and \(\langle h \rangle \) are finite cyclic groups of order \(n\), we have an isomorphism
	\begin{align*}
		\varphi :\langle g \rangle \to \langle h \rangle \\
		x^{k}\mapsto h^{k}
	\end{align*}
	and if \(\langle g \rangle \) has infinite order, then
	\begin{align*}
		\varphi :\Z\to \langle g \rangle \\
		k \mapsto g^{k}
	\end{align*}
	is an isomorphism.
\end{thm}

A finite group \(G\) is cyclic if and only if \(\left| G \right| = o(g)\) for some \(g \in G\). Lagrange's theorem implies that \(o(g) \mid \left| G \right| \) for \(\left| G \right| <\infty\).\\

Of course, a generator for a cyclic group is not necessarily unique. For example, if \(g\) generates a group with order \(n\), then \(g^{a}\) will also generate the group provided that \((n,a) = 1\). Combining this with previous results allow us to completely classify cyclic groups.

\begin{thm}[Classification of cyclic groups]
	Let \(G = \langle g \rangle \) be a cyclic group.
	\begin{enumerate}[(i).]
		\item Every subgroup \(H\leq G\) is cyclic. In particular,  \(H = \left\{ 1 \right\} \) or \(H = \langle g^{d} \rangle \), where \(d\) is the smallest positive integer such that \(g^{d} \in K\).
		\item If \(\left| G \right| = \infty\), then \(\langle g^{a} \rangle \neq \langle g^{b} \rangle  \) for distinct nonnegative integers \(a,b\). Equality only holds if \(\left| a \right| = \left| b \right| \). This implies that subgroups of infinite cyclic groups are in bijection with \(\Z_+\).
		\item If \(\left| G \right|=n <\infty\), then for each \(a\mid n\) with \(a>0\), there is a unique subgroup \(H\leq G\) with \(\left| H \right| =a\). This subgroups is exactly the cyclic group
			\begin{align*}
				H = \langle x^{\sfrac{n}{a}} \rangle .
			\end{align*}
			In general, for every integer \(m\),
			\begin{align*}
				\langle x^{m} \rangle = \langle x^{(n,m)} \rangle .
			\end{align*}
	\end{enumerate}
\end{thm}

% \printindex
\end{document}
