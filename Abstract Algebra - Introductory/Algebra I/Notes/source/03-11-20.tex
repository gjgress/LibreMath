\documentclass{memoir}
\usepackage{notestemplate}

% \begin{figure}[ht]
%     \centering
%     \incfig{riemmans-theorem}
%     \caption{Riemmans theorem}
%     \label{fig:riemmans-theorem}
% \end{figure}

\begin{document}
\section{Fermat's Last Theorem}
\label{sec:fermat_s_last_theorem}
\begin{thm}[Fermat's Last Theorem]
	Let \(n\geq 3\). Does \(x^{n}+y^{n}=z^{n}\) have positive integer solutions?
\end{thm}
It is clear that if it is true for \(n  = 4\), \(n= p\) prime, then it holds, as of course it will hold for any multiples. We can rewrite this as
\begin{align*}
	x^{p} = z^{p}-y^{p}
\end{align*}
\(y\) is a parameter, so the roots are
\begin{align*}
	z^{p} = y^{p} \implies z = (y^{p})^{\frac{1}{p}} = z, z\rho, z\rho^2,\ldots,z\rho^{p-1} \text{ where } \rho = \cos \frac{\pi}{p} + i \sin^2 \frac{2\pi}{p}
\end{align*}
So we can rewrite this as
\begin{align*}
	x^{p} = z^{p}-y^{p} = (z-y)(z-\rho y)\ldots(z-\rho y^{p-1})
\end{align*}
Observe that each prime must be a \(p\)-th power as they cannot share factors. Let
\begin{align*}
	H_p = \left\{ a_0 + a_1\rho + \ldots + a_{p-2}\rho^{p-2} \right\}, a_j \in \Z \\
	\rho^{p-1}+\rho^{p-2} + \ldots + \rho + 1 = 0
\end{align*}
If the factors on the RHS are pairwise coprime, then \(z-y = \varepsilon_0 \theta_0^{p}\), \(z-y\rho = e_1 \theta_1^{p}\)
BUT the units are non-trivial for these types of coefficients, AND we don't have UFT.\\

Then Kummer did a different direction. Note that ~"if there is a gcd then it is UFT" (not exactly, but dw about it). Then consider for \(a,b \in \Z\)
\begin{align*}
	\textrm{gcd}(a,b) = d \implies d = au + bv
\end{align*}
Consider the set
\begin{align*}
	\left\{ak + bl \mid k,l \in \Z \right\} = \left\{dn \mid n \in \Z \right\} 
\end{align*}
Now consider
\begin{align*}
	\left\{ak + bl \mid k,l \in H_p \right\} 
\end{align*}
If \( \exists \textrm{gcd}(a,b) \implies\) this set is the set of multiples of gcd. So for "ideal numbers" UFT holds and the proof holds. So for "ideal numbers" UFT holds and the proof holds.

\section{Finite Fields}
\label{sec:finite_fields}
\begin{thm}
	Let \(F\) be a finite field. \(\left| F \right| = p^{k}\), \(p\) prime. Conversely, for every \(p^{k}\) there exists exactly one \(F\) such that \(\left| F \right| = p^{k}\)
\end{thm}
%We will show that the \( \textrm{dim} = k\); namely that there exists a basis \(b_1,\ldots,b_k\) that generates all elements in \(F\) uniquely by 
%\begin{align*}
%	a = \lambda_1 b_1 + \ldots + \lambda_k b_k
%\end{align*}
%where \(\lambda_1,\ldots,\lambda_k \in \Z_p\), which by our other theorem is a subfield. This is isomorphic to \(\Z_p[x] / (g)\) which is \(F\). Observe that \( \textrm{deg}g = k\)These are the residues mod \(g\), or
%\begin{align*}
%	a_0 + a_1x + \ldots + a_{k-1}x^{k-1}
%\end{align*}
%Each \(a_i\) can be chosen \(p\) ways, so then \(\left| F \right| = p^{k}\), as desired.
\begin{thm}
	For any element \(a \in F\) a finite field,
	\begin{align*}
		a + \ldots + a = 0
	\end{align*}
	where \(a\) is added exactly \(p\) times.
\end{thm}
%For all \(a\) there exists an \(n\) such that
%\begin{align*}
%	a+\ldots+a = 0
%\end{align*}
%when \(a\) is added \(n\) times, because eventually the sequence \(a, a+a, a+a+a, \ldots\) must repeat, and then you'll have your \(n\). There are infinite solutions, but finitely many elements in \(F\). This implies that
%\begin{align*}
%	a + \ldots + a = a + \ldots + a
%\end{align*}
%where on the LHS, \(a\) is added \(i\) times, and on the RHS added \(j\) times.\\
%Assume that \(i < j\). Observe that
%\begin{align*}
%	a + \ldots + a = 0
%\end{align*}
%for \(j-i\) additions. Now assume that \(a \neq 0\). Then multiply both sides by \(b\) implies
%\begin{align*}
%	\forall b, b + \ldots + b = 0
%\end{align*}
%because we can factor. Thus the assumption holds. Now we want to find the smallest \(n>0\). Assume for the sake of contradiction that \(n = kl\), both greater than \(1\) (not prime). Then we can separate the sum into blocks of \(k\) and blocks of \(l\). Then
%\begin{align*}
%	\underbrace{a+\ldots+a}_k + \underbrace{\ldots}_{l} + \underbrace{a+\ldots+a}_k = 0
%\end{align*}
%By assumption  \(a+\ldots+a\) is non-zero, so it is some \(b \in F\). But then \(\underbrace{b+\ldots+b}_l = 0\), which is a contradiction. So it has to be prime, as desired.
\begin{thm}
	Let \(F\) be a finite field. Then
	\begin{align*}
		F = \left\{ 0,1,\alpha,\alpha^2,\ldots,\alpha ^{\left| F \right| -2}\right\} 
	\end{align*}
	and \(\alpha ^{\left| F \right| -1} = 1\)
\end{thm}
\begin{thm}[Finite Field Extensions]
	For all finite fields \(F\),there exists an \(H\leq F\) such that \(H \cong \Z_p\) for some \(p\). Furthermore, \(F\) is a vector space over \(\Z^{p}\).
\end{thm}
\(F\) is then a field extension of \(H\).
%Consider
%\begin{align*}
%	H = \left\{ 0,1,1+1,\ldots,\underbrace{1+\ldots+1}_{p-1} \right\} 
%\end{align*}
%Observe that adding and multiplying elements works like traditional addition and subtraction; namely that
%\begin{align*}
%(t\cdot 1)+(s\cdot 1) = (s+t) \pmod p \cdot 1
%\end{align*}
%for \(0\leq t\leq p-1\) and
%\begin{align*}
%	(\underbrace{1+\ldots+1}_t)\cdot (\underbrace{1+\ldots+1}_s) = ( ts) \pmod p \cdot 1
%\end{align*}
%and so the isomorphism to \(\Z_p\) holds, as all the elements can then be written by multiplication of \(0\leq t\leq p-1\) and \(1\).
\begin{thm}
	For any finite field \(F\), \(F \cong \Z_{p[x]} / (g)\) where \(g\) is a polynomial over \(\Z_p\), \( \textrm{deg }g = k\), \(g\) irreducible over \(\Z_p\)
\end{thm}
\end{document}
