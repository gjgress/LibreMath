\documentclass{memoir}
\usepackage{notestemplate}

% \begin{figure}[ht]
%     \centering
%     \incfig{riemmans-theorem}
%     \caption{Riemmans theorem}
%     \label{fig:riemmans-theorem}
% \end{figure}

\begin{document}
Every group also has a special normal subgroup called the \textbf{center}, denoted by \(Z(G)\).
\begin{align*}
	Z(G) = \left\{c \in G \mid cg = gc \text{ for every }g \in G \right\} .
\end{align*}
Furthermore, every subgroup of the center is normal.

\begin{defn}[Factor Group]
	We denote by \(G / N\) the \textbf{factor group}, whose elements are the cosets of the normal subgroup \(N\), with operations defined by \((aN)(bN) = (abN)\)
\end{defn}
\begin{defn}[Group Homomorphism]
	We call a map \(\varphi:G_1\to G_2\) a \textbf{group homomorphism} if it preserves the operation; \(\varphi(gh) = \varphi(g)\varphi(h)\) for every \(g,h \in G_1\).
\end{defn}
Naturally, \(\varphi(e_1) = \varphi(e_2)\), \(\varphi(g^{-1}) = (\varphi(g))^{-1}\), \( \textrm{Ker}(\varphi) = \left\{g \in G_1 \mid \varphi(g) = e_2 \right\} \triangleleft G_1\), and \( \textrm{Image}(\varphi) = \left\{\varphi(g) \mid g \in G_1 \right\} \leq G_2\).\\

If the homomorphism is bijective, then it is an \textbf{isomorphism}. A homomorphism \(\varphi\) is an isomorphism if and only if \( \textrm{Ker}\varphi = e_1\) and \( \textrm{Im}\varphi = G_2\).
\begin{thm}[First Isomorphism Theorem]
If \(\varphi:G_1\to G_2\) is a homomorphism, then
\begin{align*}
	\textrm{Im}\varphi \cong G_1 / \textrm{Ker}\varphi.
\end{align*}
\end{thm}
\begin{defn}[Natural Homomorphism]
	If \(N \triangleleft G\), then \(\psi:G \to G / N\) defined by \(\psi(g) = gN\) is the natural homomorphism with \( \textrm{Ker}\psi = N\) and \( \textrm{Im}\psi = G / N\).
\end{defn}

\end{document}
