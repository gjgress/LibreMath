\documentclass{memoir}
\usepackage{notestemplate}

% \begin{figure}[ht]
%     \centering
%     \incfig{riemmans-theorem}
%     \caption{Riemmans theorem}
%     \label{fig:riemmans-theorem}
% \end{figure}

\begin{document}

\subsection{Quotient Groups}
\label{subsec:quotient_groups}


\begin{defn}[Group Homomorphism]
	We call a map \(\varphi:G_1\to G_2\) a \textbf{group homomorphism} if it preserves the operation; \(\varphi(gh) = \varphi(g)\varphi(h)\) for every \(g,h \in G_1\).
\end{defn}
If the homomorphism is bijective, then it is an \textbf{isomorphism}. A homomorphism \(\varphi\) is an isomorphism if and only if \( \textrm{Ker}\varphi = e_1\) and \( \textrm{Im}\varphi = G_2\).

\begin{prop}
	Let \(G\) and \(H\) be groups and let \(\varphi :G\to H\) be a homomorphism.
	\begin{itemize}
		\item \(\varphi (1_G) = 1_H\) 
		\item \(\varphi (g^{-1}) = \varphi (g)^{-1}\)
		\item \(\varphi (g^{n}) = \varphi (g)^{n}\) 
		\item \(\textrm{Ker}\varphi \triangleleft G\)
		\item \(\textrm{Im}\varphi \leq H\)
	\end{itemize}
\end{prop}
In fact, a subgroup \(N \triangleleft G\) is normal if and only if it is the kernel of a group homomorphism.

\begin{defn}[Quotient Group]
	We denote by \(G / N\) the \textbf{quotient group}, whose elements are the cosets of the normal subgroup \(N\), with operations defined by \((aN)(bN) = (abN)\)
\end{defn}
\begin{thm}[First Isomorphism Theorem]
If \(\varphi:G_1\to G_2\) is a homomorphism, then
\begin{align*}
	\textrm{Im}\varphi \cong G_1 / \textrm{Ker}\varphi.
\end{align*}
\end{thm}
This tells us that \(\varphi \) is injective if and only if the kernel is trivial, and we can also see that \(\left| G : \textrm{Ker}\varphi  \right|  = \left| \varphi (G) \right| \).\\

This key theorem allows us to fully connect the notion of normal subgroups partitioning a group.

\begin{defn}[Natural Homomorphism]
	If \(N \triangleleft G\), then \(\psi:G \to G / N\) defined by \(\psi(g) = gN\) is the natural homomorphism with \( \textrm{Ker}\psi = N\) and \( \textrm{Im}\psi = G / N\).
\end{defn}


\begin{thm}[Third Isomorphism Theorem]
	Let \(G\) be a group, and let \(H,K \triangleleft G\) with \(H\leq K\). Then \(K / H \triangleleft G / H\) and
	\begin{align*}
		(G / H) / (K / H) \cong G /K
	\end{align*}
\end{thm}

The point of this theorem is that quotients of quotient groups provide little additional information.

\begin{thm}[Fourth Isomorphism Theorem]
	Let \(N \triangleleft G\) be a normal subgroup of \(G\). There is a bijection from the set of subgroups A satisfying \(N \leq A\leq G\) onto the set of subgroups \(A/N \leq G / N\).\\

	That is, every subgroup of \(G / N\) can be viewed as some \(A / N\) for some subgroup \(A\) containing \(N\). Furthermore, for all \(A,B \leq G\) with \(N\leq A,B\),
	\begin{itemize}[(i).]
		\item \(A\leq B \iff A / N \leq B / N\) 
		\item \(A\leq B \implies \left| B : A \right| = \left| B / N : A / N \right| \) 
		\item \(\langle A,B \rangle / N = \langle A / N, B / N \rangle \) 
		\item \(A \triangleleft G \iff A / N \triangleleft G / N\)
	\end{itemize}
\end{thm}
This theorem really just tells us that we can get isomorphisms beween structures via lattices-- if two group structures have a certain lattice structure, there is a natural isomorphism between each other.

% Include some example of lattices here

\end{document}
