\documentclass{memoir}
\usepackage{notestemplate}

% \begin{figure}[ht]
%     \centering
%     \incfig{riemmans-theorem}
%     \caption{Riemmans theorem}
%     \label{fig:riemmans-theorem}
% \end{figure}

\begin{document}
\section{Permutation Groups and Group Actions}	

\begin{defn}[Action]
	An \textbf{action} of a group \(G\) on a set \(\Omega\) is a function \(\mu:\Omega \times G \to \Omega\) with the following two properties:
	\begin{itemize}
		\item \(\mu(\mu(x,g),h) = \mu(x,gh)\) for all \(x \in \Omega\), \(g,h \in G\).
		\item \(\mu(x,e) = x\) for all \(x \in \Omega\).
	\end{itemize}
\end{defn}
It immediately foolow that \(\mu(\mu(x,g),g^{-1}) = \mu(\mu(x,g^{-1}),g) = x\) for all \(x \in \Omega, g \in G\).

\begin{prop}
	\begin{itemize}
		\item For any \(g \in G\), the map \(\pi_g:\Omega\to \Omega\) defined by \(x\pi_g = \mu(x,g)\) is a permutation.
		\item The map \(\theta:G\to S_n\) defined by \(g\theta = \pi_g\) is a homomorphism (where \(S_n\) is the set of permutations of \(\Omega\), so \(n = \left| \Omega \right| \) )
		\item Conversely, given a homomorphism \(\theta:G\to S_n\), there is an action \(\mu\) of \(G\) on \(\Omega\) given by \(\mu(x,g) = x(g\theta)\).
	\end{itemize}
\end{prop}

\begin{exmp}
	\begin{itemize}
		\item Let \(H\) be a subgroup of \(G\). Let \(\Omega\) be the set of all right cosets of \(H\) in \(G\). Define an action by \(\mu(Ha,g) = H(ag)\). This is the action of \textbf{right multiplication}.
		\item Define an action of \(G\) on itself (\(\Omega = G\) ) by the rule \(\mu(x,g) = g^{-1}xg\). This is the action of \textbf{conjugation}.
		\item Let \(\Omega\) be the set of all subgroups of \(G\). Then \(G\) acts on \(\Omega\) by \textbf{conjugation}: \(\mu(H,g) = g^{-1}Hg\).
	\end{itemize}
\end{exmp}
An equivalence relation on \(\Omega\) is formed by a group action via the rule that \(x \sim y\) if there exists \(g \in G\) with \(\mu(x,g) = y\). The equivalence classes are called \textbf{orbits}. The set \(\Omega\) decomposes into a disjoint union of orbits.
\begin{defn}[Transitivity]
	We say that an action is \textbf{transitive} if there is just one orbit, and \textbf{intransitive} otherwise.
\end{defn}
Right multiplication is transitive, but conjugation is in general not. The orbits for conjugation of \(G\) onto itself are the \textbf{conjugacy classes} of \(G\).
\begin{defn}[Stabilizer]
	The \textbf{stabilizer} of an element \(x \in \Omega\) is the set
	\begin{align*}
		\left\{g \in G \mid \mu(x,g) = x \right\} 
	\end{align*}
	of elements of \(G\) for which the corresponding permutation fixes \(x\). It is denoted \(G_x\).
\end{defn}

\begin{thm}[Orbit-Stabilizer Theorem]
	Given an action of \(G\) onto \(\Omega\), and \(x \in \Omega\), the stabilizers \(G_x\) form a subgroup of \(G\). Furthermore, there is a bijection between the orbit of \(x\) and the set of right cosets of \(G_x\) in \(G\).
\end{thm}
If \(G\) is finite, the size of the orbit of \(x\) is equal to \( \left| G:G_x \right| = \left| G \right| / \left| G_x \right| \).\\

Note that in the action of \(G\) by conjugation, we call the stabilizer of \(x\) its \textbf{centralizer} \(C_G(x)\). When considering conjugation of \(G\) by conjugation on subgroups, the stabilizer of a subgroup \(H\) is its \textbf{normalizer}
\begin{align*}
	N_G(H) = \left\{g \in G \mid g^{-1} H g = H \right\} .
\end{align*}

\begin{cor}
	Every transitive action is isomprohic to an action by right multiplication on the right cosets of a subgroup. Furthermore, the actions on the right cosets of two subgroups \(H,K\) are isomorphic if and only if \(H,K\) are conjugate.
\end{cor}

Note: Let \( \textrm{fix}(g)\) denote the number of elements in \(\Omega\) that are mapped to themselves when \(g\) is applied to them as an action.

\begin{thm}[Orbit-Counting Lemma]
	The number of orbits of \(G\) on \(\Omega\) is given by
	\begin{align*}
	\frac{1}{\left| G \right| } \sum_{g\in G} \textrm{fix}(g).	
\end{align*}
\end{thm}

\end{document}
