\documentclass{memoir}
\usepackage{notestemplate}

% \begin{figure}[ht]
%     \centering
%     \incfig{riemmans-theorem}
%     \caption{Riemmans theorem}
%     \label{fig:riemmans-theorem}
% \end{figure}

\begin{document}
\section{Direct Sums of Rings}
\label{sec:direct_sums_of_rings}
\begin{defn}[Direct Sum]
	The \textbf{direct sum} of two rings \(R_1\oplus R_2\) (or direct product \(R_1\times R_2\)) is the ring of all ordered pairs \((r_1,r_2)\), with \(r_i \in R_i\), with addition and multiplication defined by adding and multiplying the components:
	\begin{align*}
		(r_1,r_2) + (s_1,s_2) = (r_1+s_1,r_2+s_2) \quad (r_1,r_2)(s_1,s_2) = (r_1s_1,r_2s_2).
	\end{align*}
\end{defn}
One can check that this indeed satisfies the properties of a ring.
\begin{defn}[Projection]
	The \textbf{projection} subsets are defined by
	\begin{align*}
		R_1^* := \left\{(r_1,0) \mid r_1 \in R_1 \right\} \\
		R_2^* := \left\{(0,r_2) \mid r_2 \in R_2 \right\} 
	\end{align*}
\end{defn}
These projection subsets are isomorphic to \(R_1,R_2\) respectively. Furthermore, \(R_1^{*},R_2^{*}\) are ideals in \(R_1\oplus R_2\), and every \(c \in R_1 \oplus R_2\) has a unique decomposition \(c = c_1 + c_2\) with \(c_i \in R_i\).
\begin{prop}[Unique Decompositions]
	If \(I,J\) are ideals in a ring \(R\) such that every \(r \in R\) can be uniquely written as \(r = i + j\), with \(i \in I\), \(j \in J\), then \(R \cong I\oplus J\)
\end{prop}

\begin{thm}[More Isomorphism Theorems]
	Let \(R\) be a ring.
	\begin{itemize}
		\item \textit{(Second Isomorphism Theorem for Rings)} Let \(A\leq R\) and \(B \triangleleft R\). Then
			\begin{align*}
				A+B \leq R\\
				A\cap B \triangleleft A\\
				(A+B) / B \cong A / (A\cap B)
			\end{align*}
	\item \textit{(Third Isomorphism Theorem for Rings} Let \(I,J \triangleleft R\) with \(I\subset J\). Then 
		\begin{align*}
			J / I \triangleleft R / I\\
			(R / I) / (J / I) \cong R / J
		\end{align*}
	\item Let \(I \triangleleft R\). Then there is a correspondence between subrings \(A<R\) that contain an ideal \(I\) and the subrings of \(R / I\) by \(A \leftrightarrow A / I\). Furthermore, if \(A\leq R\) contains \(I\), then \(A \triangleleft R\) if and only if \(A / I \triangleleft R / I\).
	\end{itemize}
\end{thm}

\end{document}
