\documentclass{memoir}
\usepackage{notestemplate}

%\logo{~/School-Work/Auxiliary-Files/resources/png/logo.png}
%\institute{Rice University}
%\faculty{Faculty of Whatever Sciences}
%\department{Department of Mathematics}
%\title{Class Notes}
%\subtitle{Based on MATH xxx}
%\author{\textit{Author}\\Gabriel \textsc{Gress}}
%\supervisor{Linus \textsc{Torvalds}}
%\context{Well, I was bored...}
%\date{\today}

%\makeindex

\begin{document}

% \maketitle

% Notes taken on 

\subsection{Quadratic Integer Rings}
\label{sub:quadratic_integer_rings}

Let \(D\) be a rational number that is not a perfect square in \(\Q\) and define
\begin{align*}
	\Q(\sqrt{D} ) = \left\{a+b\sqrt{D}  \mid a,b \in \Q \right\} 
\end{align*}
as a subset of \(\C\). One can check that the set is closed under subtraction and under the multiplication defined by
\begin{align*}
	(a+b\sqrt{D})(c+d\sqrt{D} ) = (ac+bdD) + (ad+bc)\sqrt{D} 
\end{align*}
shows that it is closed under multiplication. Hence \(\Q(\sqrt{D} ) \leq \C\) as a subring (and of \(\R\) if \(D>0\)) and hence is commutative with identity.\\

The assumption that \(D\) is not a square allows us to write every element of \(\Q(\sqrt{D} )\) uniquely in the form \(a+b\sqrt{D} \). Furthermore, if \(a,b\) are not both zero, then \(a^2-Db^2\neq 0\), and because
\begin{align*}
	(a+b\sqrt{D} )(a-b\sqrt{D} ) = a^2-Db^2
\end{align*}
then if \(a+b\sqrt{D} \neq 0\) we have
\begin{align*}
	(a+b\sqrt{D} )^{-1} = \dfrac{a-b\sqrt{D} }{a^2-Db^2}.
\end{align*}
This shows that every nonzero element in the commutative ring is a unit and hence \(\Q(\sqrt{D} )\) is a field called a \textbf{quadratic field}.\\

The rational number \(D\) can be written by \(D = q^2D'\) for some rational number \(q\) and a unique integer \(D'\), where \(z^2\not\mid D'\) for all \(z \in \Z_+\) greater than \(1\). We call \(D'\) the \textbf{squarefree part} of \(D\). Because \(\sqrt{D} =q\sqrt{D'} \) it holds that \(\Q(\sqrt{D} )=\Q(\sqrt{D'} )\), and hence we can use squarefree integers instead in the definition of the quadratic field.\\

Let \(D\) be a squarefree integer. One can check that
\begin{align*}
	\Z[\sqrt{D} ] = \left\{a + b\sqrt{D}  \mid a,b \in \Z \right\} 
\end{align*}
forms a subring of the quadratic field \(\Q(\sqrt{D} )\). In the case where \(D \equiv 1 \pmod{4}\), then we can form a slightly larger subring by
\begin{align*}
	\Z\left[ \frac{1+\sqrt{D}}{2} \right] = \left\{ a + b \frac{1+\sqrt{D} }{2} \mid a,b \in \Z \right\} .
\end{align*}
Now define
\begin{align*}
	\mathcal{O} = \mathcal{O}_{\Q(\sqrt{D} )} = \Z[\omega ] = \left\{a + b \omega  \mid a,b \in \Z \right\} \\
	\omega = \begin{cases}
		\sqrt{D}  & D \equiv 2, 3 \pmod{4}\\
		\frac{1+\sqrt{D} }{2} & D \equiv 1 \pmod{4}
	\end{cases}.
\end{align*}
We call \(\mathcal{O}\) the \textbf{ring of integers} in the quadratic field \(\Q(\sqrt{D} )\)-- despite the fact that elements are not actually integers. This terminology arises because the properties of \(\mathcal{O}\) are similar to those of \(\Z\leq \Q\) as subrings. In fact, we will later see that it is the \textit{integral closure} of \(\Z\) in \(\Q(\sqrt{D} )\).\\

In the special case where \(D = -1\), we obtain the ring \(\Z[i]\) of \textbf{Gaussian integers}. We will cover the Gaussian integers later, as they have important ties to number theory.

\begin{defn}[Field Norm]
	Define the \textbf{field norm} \(N:\Q(\sqrt{D} ) \to \Q\) by
	\begin{align*}
		N(a+b\sqrt{D} ) = (a+b\sqrt{D} )(a-b\sqrt{D} ) = a^2 - Db^2 \in \Q
	\end{align*}
	This norm gives a notion of size in the field. For example, when \(D=-1\), the norm of \(a+bi\) is \(a^2+b^2\).
\end{defn}
One can check that \(N\) is \textbf{multiplicative}-- that is, \(N(\alpha \beta ) = N(\alpha )N(\beta )\). We can also see that on the subring \(\mathcal{O}\) the field norm is given by
\begin{align*}
	N(a+b\omega ) = (a+b\omega )(a+b \overline{\omega }) = \begin{cases}
		a^2-Db^2 & D \equiv 2,3 \pmod{4}\\
		a^2+ab + \frac{1-D}{4}b^2 & D \equiv 1 \pmod{4}
	\end{cases}\\
	\overline{\omega } = \begin{cases}
		-\sqrt{D} & D \equiv 2,3 \pmod{4}\\
		\frac{1-\sqrt{D} }{2} & D \equiv 1 \pmod{4}
	\end{cases}
\end{align*}
And hence \(N(\alpha )\) is in fact an integer for every \(\alpha  \in \mathcal{O}\). This in fact characterizes the units of \(\mathcal{O}\)-- if \(\alpha \in \mathcal{O}\) has field norm \(N(\alpha ) = \pm 1\), then
\begin{align*}
	(a + b\omega )^{-1} = \pm (a + b \overline{\omega })
\end{align*}
and hence \(\alpha \) is a unit. The multiplicative property directly tells us that the converse holds, and hence \(\alpha \in \mathcal{O}\) is a unit if and only if \(N(\alpha ) = \pm 1\).\\

In number theory, finding solutions to the equation \(x^2-Dy^2 = \pm 1\) is equivalent to the determination of units in \(\mathcal{O}\).

\begin{hw}
	Show that if \(D>0\) then the group of units \(\mathcal{O}^{\times }\) is always infinite. Find a class of units in \(\mathcal{O} = \Z[\sqrt{2} ]\) that exemplifies this.\\

	Show that \(\mathcal{O}_{\Q(\sqrt{-3} )}\) has only a finite number of elements, and list them. Are there other values of \(D\) with more units than \(\left\{ \pm 1 \right\} \)?
\end{hw}

\subsection{Polynomial Rings}
\label{sub:polynomial_rings}

\begin{defn}[Polynomial Ring]
	Let \(R\) be a commutative ring. We define a \textbf{polynomial} in \(x\) to be the formal sum
	\begin{align*}
		a_n x^{n} + a_{n-1}x^{n-1} + \ldots + a_1 x + a_0
	\end{align*}
	where \(n\geq 0\) and \(a_i \in R\). If \(a_n \neq 0\), then the polynomial is of \textbf{degree \(n\)}, and \(a_nx^{n}\) is te \textbf{leading term} (\(a_n\) is the \textbf{leading coefficient}). Furthermore, we say the polnyomial is \textbf{monic} if \(a_n=1\).\\

	The set of all such polynomials is called the \textbf{ring of polynomials in \(\R\)} and will be denoted \(R[x]\). We define addition and multiplication by the standard version from algebra:
	\begin{align*}
		(a_nx^{n} + \ldots + a_1x + a_0) + (b_nx^{n} + \ldots + b_1x + b_0) = (a_n+b_n)x^{n} + \ldots + (a_1+b_1)x + (a_0 + b_0)\\
		(a_0+a_1x + a_2x^2+\ldots) \times (b_0 + b_1x + b_2x^2 + \ldots) = a_0b_0 + (a_0b_1 + a_1b_0)x + (a_0b_2 + a_1b_1 + a_2b_0)x^2 + \ldots
	\end{align*}
	That is, the coefficient in the product of \(x^{k}\) is \(\sum_{i=0}^{k} a_i b_{k-i}\).
\end{defn}
We can see that \(R\leq R[x]\) as the \textbf{constant polynomials}. Notice further that \(R[x]\) is also a commutative ring.

\begin{prop}
	Let \(R\) be an integral domain and let \(p(x),q(x) \in R[x]\) be nonzero polynomials. Then the degree of \(p(x)q(x) = \textrm{deg}p(x) + \textrm{deg}q(x)\).\\

	The units of \(R[x]\) are exactly the same units of \(R\), and \(R[x]\) is an integral domain.
\end{prop}
If \(R\) has zero divisors, then \(R[x]\) does as well. We can also see that \(S\leq R \implies S[x] \leq R[x]\).

\begin{exmp}
	Consider the polynomial ring \((\Z_3[x]\). This ring consists of nonnegative powers of \(x\) with coefficients in \(\left\{ 0,1,2 \right\} \) with calculations being done in modulus \(3\). For example, let
	\begin{align*}
		p(x) = x^2+2x+1 \quad q(x) = x^3+x +2.
	\end{align*}
	Then
	\begin{align*}
		p(x) + q(x) = x^3+x^2\\
		p(x)q(x) = x^{5}+2x^{4}+2x^3+x^2+2x+2
	\end{align*}
	Polynomials behave very differently even under simple modulus structures.
\end{exmp}
We will see more of polynomial rings after building more theory.

\subsection{Matrix Rings}
\label{sub:matrix_rings}

\begin{defn}[Matrix Rings]
	Let \(R\) be a ring and \(n\in \Z_+\). We define \(M_n(R)\) to be the set of all \(n\times n\) matrices with entries in \(R\). The element \(A \in M_n(R)\) is an \(n\times n\) square array of elements of \(R\) whose entry in row \(i\) and column \(j\) is \(A_{ij} \in R\). We see that this set of matices becomes a ring under the usual matrix addition and multiplication, called the \textbf{matrix ring of rank \(n\)}.
\end{defn}
Notice that if \(n\geq 2\), then \(M_n(R)\) is not commutative, regardless of the commutativity of \(R\). Furthermore, it will also have zero divisors.\\

We say \(A \in M_n(R)\) is a \textbf{scalar matrix} if \(a_{ii}=a\) for all \(i \in \left\{ 1,\ldots,n \right\} \), and \(a_{ij}=0\) if \(i\neq j\). This forms a subgring of \(M_n(R)\), and is in fact isomorphic to \(R\). If \(R\) is commutative, the scalar matrices commute with all elements of \(M_n(R)\).\\

Note that the units of \(M_n(R)\) are the invertible \(n\times n\) matrices-- this forms a subgroup called the \textbf{general linear group of degree \(n\) over \(R\)} (written by \(GL_n(R)\)).\\

Similar to polynomial rings, if \(S\leq R\), then \(M_n(S) \leq M_n(R)\). Another subring is the set of upper triangular matrices.

\subsection{Group Rings}
\label{sub:group_rings}

\begin{defn}[Group Rings]
Let \(R\) be a commutative ring and \(G \) a finite group. The \textbf{group ring \(RG\) of \(G\)} is the set of all sums
\begin{align*}
	RG = \left\{ a_1g_1 + a_2g_2 + \ldots + a_ng_n \mid a_i\in R, g_i \in G\right\} 
\end{align*}
We define addition and multiplication by
\begin{align*}
	(a_1g_1+ a_2g_2 + \ldots + a_ng_n) + (b_1g_1 + b_2g_2 + \ldots + b_ng_n) = (a_1+b_1)g_1 + (a_2+b_2)g_2 + \ldots + (a_n+b_n)g_n\\
		(a_1g_1+ \ldots + a_ng_n)(b_1g_1 + \ldots + b_ng_n) = \sum_{g_ig_j = g_k} (a_ib_j)g_k 
\end{align*}
where the multiplication is the natural construction derived from defining \((ag_i)(bg_j) = (ab)(g_ig_j)\).
\end{defn}
\(RG\) is commutative if and only if \(G\) is commutative. We can see \(R \leq RG\) by the "constant" sums of \(a_ie_G\). In fact, \(G\leq RG\) as well by taking \(a_i = e_R\), and because elements of \(G\) has inverses, \(G\) is a subgroup of the group of units of \(RG\).\\

If \(\left| G \right| >1\) then \(RG\) has zero divisors, given by
\begin{align*}
	(1-g)(1+g+\ldots+g^{m-1}) = 1-g^{m} = 1-1 = 0
\end{align*}
where \(g\) is an element with order \(m>1\).\\

If \(S\) is a subring of \(R\) then \(SG \leq RG\). Similarly, if \(H\leq G\), then \(RH \leq RG\).

\begin{exmp}[\(\Z D_8\)]
	Let \(G = D_8\) be the dihedral group of order \(8\) and \(R = \Z\). Some example of elements in \(\Z D_8\) could be \(\alpha  = r + r^2 - 2s\) and \(\beta = -3r^2 + rs\), and one can see that
	\begin{align*}
		\alpha + \beta = r - 2r^2 -2s +rs\\
		\alpha \beta = (r+r^2-2s)(-3r^2+rs) = r(-3r^2+rs) + r^2(-3r^2+rs) - 2s(-3r^2+rs) = -3 - 5r^3 + 7r^2s + r^3s
	\end{align*}
\end{exmp}

\begin{exmp}[\(\R Q_8\)]
	An interesting example is the group ring \(\R Q_8\). This ring is distinct from the Hamilton quaternions \(\mathbb{H}\) even though \(Q_8 \subset \mathbb{H}\). The unique element of order \(2\) in \(Q_8\) is NOT the additive inverse of \(1 \) in \(\R Q_8\), even though it is in \(\mathbb{H}\). It also contains zero divisors and hence is not a division ring.\\

	However, if one takes the quotient \(\R Q_8 / \left( 1 + (-1), i + (-i), j + (-j), k + (-k) \right) \), then it is isomorphic to \(\mathbb{H}\).
\end{exmp}
In other words, we only apply the group operation between elements of the group ring when multiplying two elements. The group elements hence serve as sort of a basis, that interacts multiplicatively via the group action.

\subsection{Rings of Fractions}
\label{sub:rings_of_fractions}

Let \(R\) be a commutative ring. Recall that if we have a non-zero non-zero divisor element \(a \in R\), then \(ab = ac \implies b = c\). This property is similar to division even if \(a\) is not a unit. Our goal will be to define a larger ring \(Q\geq R\) so that elements like \(a\) are units. This becomes particularly useful when \(R\) is an integral domain, as then \(Q\) becomes a field known as the \textbf{field of fractions} or \textbf{quotient field}.

\begin{thm}
	Let \(R\) be a commutative ring and \(D\subset R\) a subset without \(0\), zero divisors, and is closed under multiplication. Then there is a commutative ring \(Q\) such that \(R\leq Q\) and for all \(d \in D\), \(d \in Q\) is a unit.\\

	Furthermore, every element of \(Q\) is of the form \(rd^{-1}\) for some \(r \in R\) and \(d \in D\). If \(D = R - \left\{ 0 \right\} \), then \(Q\) is a field.\\

	The ring \(Q\) is the smallest ring containing \(R\) in which all elements of \(D\) become units. That is, let \(S\) be a commutative ring and \(\varphi :R\to S\) be an injective ring homomorphism such that \(\varphi (d)\) is a unit in \(S\) for every \(d \in D\). Then there is an injective homomorphism \(\Phi :Q\to S\) such that \(\Phi\mid_R = \varphi \). In other words, any other ring that makes \(D\) into units must contain an isomorphic copy of \(Q\).\\

	We call the ring \(Q\) the \textbf{ring of fractions of \(D\)} and denote it by \(D^{-1}R\). If \(R\) is integral, then \(Q\) is the \textbf{field of fractions} of \(R\).
\end{thm}
What does this actually look like? Caution must be exercised, as when dealing with fractions we are dealing with equivalence classes. Hence we will define an equivalence class on \((r,d)\) with \(r \in R\) and \(d \in D\) by
\begin{align*}
	\dfrac{r}{d} = \left\{(a,b) \mid a \in R,\, b \in D,\, rb = ad \right\} 
\end{align*}
Then \(Q\) is the set of equivalence classes \(\dfrac{r}{d}\). Then we define addition and multiplication by
\begin{align*}
	\dfrac{a}{b}+ \dfrac{c}{d}= \dfrac{ad+bc}{bd} \quad \dfrac{a}{b} \times \dfrac{c}{d} = \dfrac{ac}{bd}.
\end{align*}
We leave it as an exercise to verify that this indeed gives \(Q\) the structure of a commutative ring. We embed \(R\) into \(Q\) by defining
\begin{align*}
	\iota : R\to Q \quad \iota:r\mapsto \dfrac{rd}{d}
\end{align*}
for any \(d \in D\). This is obviously in the equivalence class of \(\dfrac{re}{e}\), so the choice of \(d\) does not matter. This is a ring homomorphism and is in fact injective because \(d\) is not a zero divisor, and so this tells us that \(\iota(R)\leq Q\) is isomorphic to \(R\).\\

We can also see that \(d \in D\) has a multiplicative inverse in \(Q\) as desired. That is,
\begin{align*}
	\left( \dfrac{de}{e} \right)^{-1} = \dfrac{e}{de}
\end{align*}
and one can see that every element of \(Q\) can be written by \(r \cdot d^{-1}\) for some \(r \in R\) and \(d \in D\).

\begin{rmrk}
	Recall that if \(A\subset F\) is a subset of a field, then the intersection of all the subfields of \(F\) containing \(A\) is a subfield of \(F\) called the \textbf{subfield generated by \(A\)}.\\

	This subfield is the smallest subfield of \(F\) containing \(A\).
\end{rmrk}

\begin{cor}
	Let \(R\) be an integra domain and \(Q\) be the field of fractions of \(R\). If a field \(F\) contains a subring \(R'\) isomorphic to \(R\) then the subfield of \(F\) generated by \(R'\) is isomorphic to \(Q\).
\end{cor}
% \printindex
\end{document}
