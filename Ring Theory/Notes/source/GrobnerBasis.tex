\documentclass{memoir}
\usepackage{notestemplate}

%\logo{~/School-Work/Auxiliary-Files/resources/png/logo.png}
%\institute{Rice University}
%\faculty{Faculty of Whatever Sciences}
%\department{Department of Mathematics}
%\title{Class Notes}
%\subtitle{Based on MATH xxx}
%\author{\textit{Author}\\Gabriel \textsc{Gress}}
%\supervisor{Linus \textsc{Torvalds}}
%\context{Well, I was bored...}
%\date{\today}

%\makeindex

\begin{document}

% \maketitle

% Notes taken on 06/06/21

Because a field is clearly Noetherian, we get an interesting result:
\begin{cor}
	Every ideal in the polynomial ring \(F[x_1,x_2,\ldots,x_n]\) is finitely generated.
\end{cor}
The colection of leading coefficients of polynomials in \(I\triangleleft R[x]\) form a useful ideal in \(R\) that characterizes \(I\). We can utilize this to study \(F[x_1,x_2,\ldots,x_n]\), but we need to exercise caution-- we need an ordering on the monomials in order to determine the leading term of a polynomial.

\begin{defn}[Monomial Ordering]
	A \textbf{monomial ordering} is a well ordering on the set of polynomials that satisfies \(mm_1\geq mm_2\) whenever \(m_1\geq m_2\) for monomials \(m,m_1,m_2\).
\end{defn}

\begin{defn}[Leading Terms]
	Fix a monomial ordering on the polynomial ring \(F[x_1,x_2,\ldots,x_n]\). The \textbf{leading term} of a nonzero polynomial \(f \in F[x_1,x_2,\ldots,x_n]\), denoted \(LT(f)\), is the monomial term of maximal order in \(f\). The \textbf{multidegree of \(f\)}, denoted \(\partial(f)\), is the multidegree of the leading term of \(f\).\\

	If \(I\) is an ideal in \(F[x_1,x_2,\ldots,x_n]\), the \textbf{ideal of leading terms}, denoted \(LT(I)\), is the ideal generated by the leading terms of all the elements in the ideal:
	\begin{align*}
		LT(I) = (LT(f) \mid f \in I).
	\end{align*}
\end{defn}
It is clear that the choice of ordering affects the leading term and hence the multidegree of a polynomial. Notice that \(\partial(fg) = \partial f + \partial g\) when \(f\) and \(g\) are nonzero, and hence \(LT(fg) = LT(f) + LT(g)\).\\

By construction, \(LT(I)\) is generated by monomials. We refer to these kinds of ideals as \textbf{monomial ideals}.

\begin{hw}
	Show that a polynomial \(p(x)\) is contained in the monomial ideal \((f_1(x),f_2(x),\ldots,f_n(x))\) (where \(f_i(x)\) is a monomial) if and only if each of the monomial terms of \(p(x)\) is a multiple of one of the generators \(f_i(x)\).
\end{hw}
If \(I = (f_1,\ldots,f_m)\), notice that \(LT(I)\) contains the ideals of each of the leading terms:
\begin{align*}
	(LT(f_1),\ldots,LT(f_m)) \subset LT(I).
\end{align*}
This inequality can in fact be strict.

\begin{exmp}
	
\end{exmp}
The intuition behind this strange fact is that while a smaller polynomial may not be in \((LT(f_1),LT(f_2),\ldots,LT(f_n))\), one could potentially generate polynomials by canceling out the leading terms of \(f_1,f_2\). Hence it is natural to look for a basis that generates these smaller polynomials as well.

\begin{defn}[Grobner Basis]
	A \textbf{Grobner basis} for an ideal \(I\) in \(F[x_1,\ldots,x_n]\) is a finite set of generators \(\left\{ g_1,\ldots,g_m \right\} \) for \(I\) whose leading terms generate the ideal of all leading terms in \(I\):
	\begin{align*}
		I = (g_1,\ldots,g_m) \quad LT(I) = (LT(g_1),\ldots,LT(g_m)).
	\end{align*}
\end{defn}
Notice that it is a basis in the sense that it is a set of generators-- it is not a basis in the sense of vector spaces.\\

A Grobner basis allows for every polynomial to be written uniquely as a sum of elements in \(I\) and remainder \(r\). Hence, it is a tool that allows us to give back more structure to multivariable polynomial rings.

\begin{general}[General Polynomial Division]
	Let \(F[x_1,\ldots,x_n]\) be a monomial ordering and suppose \(g_1,\ldots,g_m \in F[x_1,\ldots,x_n]\) is a set of nonzero polynomials. If \(f \in F[x_1,\ldots,x_n]\) then we construct a set of quotients \(q_1,\ldots,q_m\) and a remainder \(r\) (initially all zero) and test if \(LT(f)\) is divisible by \(LT(g_i)\) in order.\\

	If \(LT(g_i)\mid LT(f)\), i.e \(LT\left(  = a_iLT(g_i) \right) \), then \(q_i' = q_i+a_i\) and \(f' = f-a_ig_i\), and reiterate.\\

	Once the leading term  \(LT(f)\) is not divisible by any \(LT(g_1),\ldots,LT(g_m)\), set \(r' = r+LT(f)\) and \(f' = f-LT(f)\) and reiterate.\\

	The process terminates when \(f = 0\) and results in a set of quotients and remainder such that:
	\begin{align*}
		f = q_1g_1 + \ldots + q_mg_m + r.
	\end{align*}
	Notice that \(q_ig_i\) has multidegree less than or equal to the multidegree of \(f\). Furthermore, no nonzero term in \(r\) is divisible by any \(LT(g_i)\).
\end{general}
Unfortunately, the remainder is dependent on the choices of \(g_i\), and is nonunique. In fact, sometimes the remainder is zero for the right choice of \(g_i\), but not others, making it difficult to utilize the factorization. Of course, our Grobner basis is the right tool to fix this.

\begin{thm}
	Fix a monomial ordering on \(R = F[x_1,\ldots,x_n]\) and suppose \(g_1,\ldots,g_m\) is a Grobner basis for the nonzero ideal \(I\triangleleft R\). Then every polynomial \(f\) can be written uniquely as
	\begin{align*}
		f = f_I + 4
	\end{align*}
	where \(f_I \in I\) and no nonzero monomial term of \(r\) is divisible by \(LT(g_i)\).\\

	Both \(f_I\) and \(r\) are computed by general polynomial division by \(g_1,\ldots,g_m\) and are independent of the order of \(g_i\).\\

	Finally, the remainder \(r\) provides a unique representative for the coset of \(f\) in \(F[x_1,\ldots,x_n] / I\). This also tells us that \(f \in I\) if and only if \(r=0\).
\end{thm}
\begin{proof}
	
\end{proof}
Now we will see that if we have a well-defined division algorithm for some set of polynomials, that they must be a Grobner basis.

\begin{prop}
	Fix a monomial ordering on \(R = F[x_1,\ldots,x_n]\) and let \(I \triangleleft R\) be nonzero. If \(g_1,\ldots,g_m \in I\) such that
	\begin{align*}
		LT(I) = (LT(g_1),\ldots,LT(g_m))
	\end{align*}
	then \(g_1,\ldots,g_m \in I\) is a Grobner basis for \(I\).\\

	For any ideal \(I\triangleleft R\) nonzero, the ideal has a Grobner basis.
\end{prop}

Now our goal is to form some criterion to determine whether a set of generators forms a Grobner basis. Recall that our main difficulty in forming a nice basis is that it is possible to cancel leading terms of two polynomials. One way this is done is via the equation below:
\begin{align*}
	S(f_1,f_2) := \frac{M}{LT(f_1)}f_1 - \frac{M}{LT(f_2)}f_2.
\end{align*}
where \(M\) is the monic lcm of \(LT(f_1),LT(f_2)\). One can see that this in fact causes the leading terms of \(f_1,f_2\) to cancel.

\begin{lemma}
	Suppose \(f_1,\ldots,f_m \in F[x_1,\ldots,x_n]\) are polynomials with multidegree \(\alpha \) and determined so that
	\begin{align*}
		h = a_1f_1 + \ldots + a_mf_m
	\end{align*}
	with constants \(a_i \in F\) has strictly smaller multidegree. Then
	\begin{align*}
		h = \sum_{i=2}^{m} b_i S(f_{i-1},f_i)
	\end{align*}
	for some constants \(b_i \in F\).
\end{lemma}
We can use this to show that a set of generators is Grobner if there are no new leading terms among \(S(g_i,g_j)\).

\begin{rmrk}
	Let \(R = F[x_1,\ldots,x_n]\) and \(G = \left\{ g_1,\ldots,g_m \right\} \). We write \(f \equiv r \pmod{G}\) if \(r\) is the remainder obtained by general polynomial division of \(f\) by polynomials \(g1,\ldots,g_m\).
\end{rmrk}

\begin{prop}[Buchberger's Criterion]
	Let \(R = F[x_1,\ldots,x_n]\) and fix a monomial ordering on \(R\). If \(I = (g_1,\ldots,g_m)\) is a nonzero ideal in \(R\), then \(G = \left\{ g_1,\ldots,g)m \right\} \) is a Grobner basis for \(I\) if and only if \(S(g_i,g_j) \equiv 0 \pmod{G}\) for \(1\leq i<j\leq m\).
\end{prop}
\begin{proof}
	
\end{proof}

We can use this to form a new algorithm to find a Grobner basis.

\begin{general}[Buchberger's Algorithm]
	Let \(I = (g_1,\ldots,g_m)\). If \(S(g_i,g_j)\) leaves a remainder of \(0\) when divided by  \(G = \left\{ g_1,\ldots,g_m \right\} \) using general polynomial division, then \(G\) is a Grobner basis. Otherwise, we proceed with the algorithm.\\

	If \(G\) is not a Grobner basis, then \(S(g_i,g_j)\) has a nonzero remainder \(r\). We increase \(G\) by appending the polynomial \(g_{m+1} =r\) to form
	 \begin{align*}
		G' = \left\{g_1,\ldots,g_m,g_{m+1}  \right\} 
	\end{align*}
	and repeat general polynomial division.\\

	This procedure terminates after a finite number of steps, and will result in a generating set \(G\) that satisfies Buchberger's Criterion.\\
\end{general}
If \(\left\{ g_1,\ldots,g_m \right\} \) is a Grobner basis for the ideal \(I\), and \(LT(g_i)\mid LT(g_j)\) for some \(i\neq j\), then \(LT(g_j)\) can be dropped from the list without affecting the basis.\\

We can assume without loss of generality that the leading term of each \(g_i\) is monic. A Grobner basis \(G = \left\{g_1,\ldots,g_m \right\} \) for \(I\) where \(LT(g_i)\) is monic and no leading temr divides the other is called a \textbf{minimal Grobner basis}.\\

A minimal Grobner basis is not unique, but the number of elements and their leading terms are unique. We can impose additional restrictions to make a basis unique.

\begin{defn}[Reduced Grobner Basis]
	Fix a monomial ordering on \(R = F[x_1,\ldots,x_n]\). A Grobner basis \(\left\{ g_1,\ldots,g_m \right\} \) for a nonzero ideal \(I\triangleleft R\) is called a \textbf{reduced Grobner basis} if each \(LT(g_i)\) is monic and \(LT(g_i)\not\mid g_j\) for \(i\neq j\).
\end{defn}
This is of course a stronger type of minimal Grobner basis. We can turn a minimal Grobner basis into a reduced Grobner basis by replacing each \(g_i\) by its remainder after division by \(g_j\).

\begin{thm}
	Fix a monomial ordering on \(R = F[x_1,\ldots,x_n]\). Then there is a unique reduced Grobner basis for every nonzero ideal \(I \triangleleft R\).
\end{thm}
This is really useful to distinguish ideals in a polynomial ring.

\begin{cor}
	Let \(I,J \triangleleft F[x_1,\ldots,x_n]\). Then \(I=J\) if and only if \(I,J\) have the same reduced Grobner basis with respect to any fixed monomial ordering on \(F[x_1,\ldots,x_n]\).
\end{cor}

\begin{exmp}
	
\end{exmp}

\subsection{Solving Algebraic Equations}
\label{sub:solving_algebraic_equations}

We can use Grobner bases to help solve systems of algebraic equations. Suppose \(S = \left\{ f_1,\ldots,f_m \right\} \) is a collection of polynomials in \(n\) variables \(x_1,\ldots,x_n\), and we are trying to find a solution to the system of equations \(f_i = 0\) for all \(i \in \left\{ 1,\ldots,m \right\} \). Notice that if \((a_1,\ldots,a_n)\) is a solution to this system, then every element \(f \in I\) where \(I\) is generated by \(S\) also satisfies \(f_1(a_1,\ldots,a_n) = 0\).\\

The ideas of Grobner bases allow us to expand the theory of linear polynomial equations to nonlinear polynomial equations. The process of finding elements of the ideal \(I\) independent of variables (so as to reduce the equation similar to Gauss-Jordan elimination) is part of \textit{elimination theory}.

\begin{defn}[Elimination Ideal]
	If \(I\triangleleft F[x_1,\ldots,x_n]\), then
	\begin{align*}
		I_i = I \cap F[x_{i+1},\ldots,x_n]
	\end{align*}
	is called the \(i\)-th \textbf{elimination ideal} of \(I\) with respect to the ordering \(x_1>\ldots>x_n\).
\end{defn}

\begin{prop}[Elimination]
	Suppose \(G = \left\{ g_1,\ldots,g_m \right\} \) is a Grobner basis for the nonzero ideal \(I \triangleleft F[x_1,\ldots,x_n]\) with respect to the lexicographic monomial ordering \(x_1>\ldots>x_n\). Then
	\begin{align*}
		G \cap F[x_{i+1},\ldots,x_n]
	\end{align*}
	is a Grobner basis of the \(i\)-th elimination ideal
	\begin{align*}
		I_i = I \cap F[x_{i+1},\ldots,x_n] \subset I.
	\end{align*}
	In particular, \(I_i = 0\) if and only if \(G \cap F[x_{i+1},\ldots,x_n] = \emptyset\).
\end{prop}
\begin{proof}
	
\end{proof}

\begin{exmp}
		
\end{exmp}

\begin{prop}
	If \(I,J \triangleleft F[x_1,\ldots,x_n]\), then
	\begin{align*}
		tI + (1-t)J \triangleleft F[t,x_1,\ldots,x_n]
	\end{align*}
	and
	\begin{align*}
		I \cap J = (tI + (1-t)J) \cap F[x_1,\ldots,x_n].
	\end{align*}
	Hence \(I\cap J\) is the first elimination ideal of \(tI + (1-t)J\) with respect to the ordering \(t > x_1 > \ldots > x_n\).
\end{prop}

% \printindex
\end{document}
