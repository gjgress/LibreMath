\documentclass{memoir}
\usepackage{notestemplate}

% \begin{figure}[ht]
%     \centering
%     \incfig{riemmans-theorem}
%     \caption{Riemmans theorem}
%     \label{fig:riemmans-theorem}
% \end{figure}

\begin{document}
\begin{defn}[Converging to a point]
	Let \(\left\{ z_n \right\}  \subset \C\) be a sequence of complex numbers. Let \(z \in \C\) be a complex number. We say that \((z_n)\) \textbf{converges to \(z\)} if \(\lim_{n \to \infty} \left| z_n-z \right| =0\).
\end{defn}
Of course this is identical to the coordinate-wise convergence in \(\R^2\). In other words, a complex sequence converges if and only if the real part converges to the real part of \(z\) and the imaginary part converges to the imaginary part of \(z\).\\

Note that \(\C\) is Hausdorff and thus if a sequence converges it only converges to one point in the complex plane.
\begin{defn}[Convergent and Cauchy Sequences]
	Let \(\left\{ z_n \right\} \subset \C\). Then
	\begin{itemize}
		\item \((z_n)\) is \textbf{convergent} if it converges to some \(z \in \C\).
		\item \((z_n)\) is a \textbf{Cauchy sequence} if for all \(\varepsilon>0, \exists n\geq N\) such that \(\left| z_n-z_m \right|\leq \varepsilon\) for all \(n,m \geq N\).
	\end{itemize}
\end{defn}

\begin{anki}
TARGET DECK
Complex Qual::Complex Analysis
START
MathJaxCloze
Text: Let \(\left\{ z_n \right\} \subset \C\). Then

* \((z_n)\) is **convergent** if
{{c1::\(\begin{align*}
        	\lim_{n \to \infty} \left| z_n - z \right| = 0
        \end{align*}\)}} 
for some \(z \in \C\).
* \((z_n)\) is a **Cauchy sequence** if for all \(\varepsilon>0\) there exists \( n\geq N\) such that
 {{c2::\(\begin{align*}
        	\left| z_n-z_m \right|\leq \varepsilon
        \end{align*}\)}} 
for all \(n,m \geq N\).
Tags: analysis complex_analysis complex_convergence
<!--ID: 1624245565291-->
END
\end{anki}


\begin{thm}[Completeness of \(\C\) ]
	\(\C\) is complete. That is, a sequence is convergent if and only if it is Cauchy.
\end{thm}

\begin{anki}
START
MathJaxCloze
Text: \(\C\) is {{c1::complete}}. That is, a sequence is convergent if and only if {{c1::it is Cauchy}}.
Tags: analysis complex_analysis complex_convergence
<!--ID: 1624245565361-->
END
\end{anki}


\begin{prop}
	Let \(z_n\to z,w_n\to w\) be two convergent sequences of complex numbers. Then:
	\begin{itemize}
		\item \(z_n+w_n \to z+w\) 
		\item \(z_n - w_n = z-w\)
		\item \(z_nw_n \to zw\) 
		\item \(\frac{z_n}{w_n} \to \frac{z}{w}\) when \(w_n\neq 0\)
	\end{itemize}
	In other words, \(\C\) is a topological field.
\end{prop}

\begin{prop}
	Let \(\Omega\subset \C\). Let \(z \in \C\). Then \(z \in \overline{\Omega}\) if and only if there is a convergent sequence in \(\Omega\) that converges to \(z\).
\end{prop}

\begin{cor}
	Let \(\Omega\subset \C\). Then \(\Omega\) is closed if and only if, for every convergent sequence in \(\Omega\), the limit also lies in \(\Omega\).
\end{cor}

\begin{anki}
START
MathJaxCloze
Text: Let \(\Omega\subset \C\). Let \(z \in \C\). Then \(z \in \overline{\Omega}\) if and only if there is {{c1::a convergent sequence in \(\Omega\) that converges to \(z\)::sequence}} .
Extra: \(\Omega\) is closed if and only if, for every convergent sequence in \(\Omega\), the limit also lies in \(\Omega\).
Tags: analysis complex_analysis complex_convergence
<!--ID: 1624245565405-->
END
\end{anki}


\begin{defn}[Diameter and Boundedness]
	Let \(\Omega\subset \C\). The \textbf{diameter} of \(\Omega\) is
	\begin{align*}
		\sup \left\{\left| w-z \right|  \mid w,z \in \Omega \right\} .
	\end{align*}
	We say that \(\Omega\) is \textbf{bounded} if the diameter is finite.
\end{defn}

\begin{anki}
START
MathJaxCloze
Text: Let \(\Omega\subset \C\). The **diameter** of \(\Omega\) is
 {{c1::\(\begin{align*}
         	\sup \left\{\left| w-z \right|  \mid w,z \in \Omega \right\} .
         \end{align*}\)}} 
We say that \(\Omega\) is **bounded** if {{c1::the diameter is finite}}.
Tags: analysis complex_analysis complex_convergence
<!--ID: 1624245565441-->
END
\end{anki}

\begin{thm}
	Let \(F_1 \supset F_2 \supset F_3 \supset \ldots\) be a decreasing chain of non-empty closed subsets of \(\C\). Assume that the diameter of \(F_n\to 0\). Then \(\bigcap_{n=1}^{\infty}F_n\) is a singular set \(\left\{ z \right\} \).
\end{thm}
This sequence converges if even one \(F_n\) is bounded, but the convergent set may not be singular.

\begin{anki}
START
MathJaxCloze
Text: Let \(F_1 \supset F_2 \supset F_3 \supset \ldots\) be a decreasing chain of non-empty closed subsets of \(\C\). Assume that the diameter of \(F_n\to 0\). Then {{c1::\(\bigcap_{n=1}^{\infty}F_n\) is a singular set \(\left\{ z \right\} \)}}.
Extra: The diameter converges if even one \(F_n\) is bounded, but the convergent set may not be singular.
Tags: analysis complex_analysis complex_convergence
<!--ID: 1624245565476-->
END
\end{anki}

\begin{defn}[Compactness]
	A set \(\Omega\subset \C\) is \textbf{compact} if every open cover of \(\Omega\) has a finite subcover.
\end{defn}

This is inherited from the topological compactness via the basis of open discs of \(\C\).

\begin{thm}
	Let \(\Omega \subset \C\). The following are equivalent:
	\begin{itemize}
		\item \(\Omega\) is compact
		\item Every sequence in \(\Omega\) has a subsequence converging to a point in \(\Omega\)
		\item \(\Omega\) is closed and bounded
	\end{itemize}
\end{thm}

\begin{anki}
START
MathJaxCloze
Text: 
Let \(\Omega \subset \C\). The following are equivalent:
* {{c1::\(\Omega\) is compact}}
* {{c2::Every sequence in \(\Omega\) has a subsequence converging to a point in \(\Omega\)}}
* {{c3::\(\Omega\) is closed and bounded}} 
Tags: analysis complex_analysis complex_convergence
<!--ID: 1624245565513-->
END
\end{anki}

\begin{defn}[Connected Set]
	Let \(\Omega\subset \C\) be an open set. \(\Omega\) is \textbf{connected} if there does not exist two disjoint non-empty open sets \(\Omega_1,\Omega_2\subset \C\) such that
	\begin{align*}
		\Omega = \Omega_1\cup \Omega_2.
	\end{align*}
	A connected open set in \(\C\) is often referred to as a \textbf{region}.
\end{defn}

\begin{thm}
	Let \(K\subset G\subset \C\), where \(K\) is compact and \(G\) is open. Then there is an \(r>0\) such that, for any \(z \in K\),
	\begin{align*}
		D(z,r) \subset G.
	\end{align*}
\end{thm}

\begin{anki}
START
MathJaxCloze
Text: Let \(K\subset G\subset \C\), where \(K\) is compact and \(G\) is open. Then there is an \(r>0\) such that, for any \(z \in K\),
 {{c1::\(\begin{align*}
        	D(z,r) \subset G.
        \end{align*}\)}}
Tags: analysis complex_analysis complex_topology
<!--ID: 1624845437350-->
END
\end{anki}


\end{document}
