\documentclass{memoir}
\usepackage{notestemplate}

% \begin{figure}[ht]
%     \centering
%     \incfig{riemmans-theorem}
%     \caption{Riemmans theorem}
%     \label{fig:riemmans-theorem}
% \end{figure}

\begin{document}
% \section{}	

Now we consider functions on the complex field. Namely, we are characterizing functions \(f:\C\to \C\). It is immediate that any function \(f\) can be parametrized by
\begin{align*}
	f:\C\to \C\\
	(x,y) \mapsto (u(x,y),v(x,y))
\end{align*}
for \(u,v \) real-valued functions. Sometimes we might write \(f = u + iv\) for shorthand of the above parametrization. This notation arises because
\begin{align*}
	f(x,y) = (u(x,y),v(x,y)) &\implies\\
	f &= (u(x,y),0) + (0,v(x,y)) \implies\\
	f &= u + iv.
\end{align*}
Now we construct the equivalent analytic structures on complex functions.
\begin{defn}[Complex Continuity]
	Let \(\Omega\subset \C\). A function \(f:\Omega\to \C\) is \textbf{continuous} if for all \(\varepsilon >0\), for all \(z \in \Omega\), there exists a \(\delta>0\) such that, for all \(w \in \Omega\),
	\begin{align*}
		|w-z| < \delta \implies \left| f(w)-f(z) \right| < \varepsilon
	\end{align*}
Equivalently, if \((z_n) \subset \Omega\) is an arbitrary sequence in \(\Omega\) converging to \(z \in \C\), then \(f(z_n) \to f(z)\).\\

We denote the space of complex continuous functions on \(\Omega\) by \(C(\Omega)\).
\end{defn}

\begin{anki}
TARGET DECK
Complex Qual::Complex Analysis

START
MathJaxCloze
Text: Let \(\Omega\subset \C\). A function \(f:\Omega\to \C\) is **continuous** if {{c2::for all \(\varepsilon >0\)}}, {{c2::for all \(z \in \Omega\)}}, {{c2::there exists a \(\delta>0\)}} such that, {{c2::for all \(w \in \Omega\)}},
{{c1::\(\begin{align*}
        	|w-z| < \delta \implies \left| f(w)-f(z) \right| < \varepsilon
        \end{align*}\)}} 
Equivalently, if \((z_n) \subset \Omega\) is an arbitrary sequence in \(\Omega\) converging to \(z \in \C\), then {{c1::\(f(z_n) \to f(z)\)}}.
Extra: Of course, we can alternatively define continuity of a function at a point, then refer to a continuous function as one that's continuous at all points.
Tags: analysis complex_analysis complex_analyticity defn
<!--ID: 1624399037000-->
END
\end{anki}

Of course, we can alternatively define continuity of a function at a point, then refer to a continuous function as one that's continuous at all points.\\

The sum, difference, product, ratio, and composition of continuous functions is continuous. We leave the proof of this as an exercise to the reader.
\begin{defn}[Uniform Continuity]
	Let \(\Omega \subset \C\). A function \(f:\Omega\to \C\) is \textbf{uniformly continuous} if for all \(\varepsilon\), there exists a \(\delta>0\) such that for all \(z,w \in \Omega\),
	\begin{align*}
		|w-z| < \delta \implies \left| f(w)-f(z) \right| < \varepsilon
	\end{align*}
\end{defn}

\begin{anki}
START
MathJaxCloze
Text: Let \(\Omega \subset \C\). A function \(f:\Omega\to \C\) is **uniformly continuous** if {{c1::for all \(\varepsilon\)}}, {{c1::there exists a \(\delta>0\)}} such that {{c1::for all \(z,w \in \Omega\)}},
{{c2::\(\begin{align*}
        	|w-z| < \delta \implies \left| f(w)-f(z) \right| < \varepsilon
        \end{align*}\)}}
Extra: In other words, we choose \(\delta \) first before being given the point \(z \in \Omega\). On unbounded domains, this is a much stronger property than continuity. 
Tags: analysis complex_analysis complex_analyticity defn
<!--ID: 1624399037047-->
END
\end{anki}

In other words, we choose \(\delta \) first before being given the point \(z \in \Omega\). On unbounded domains, this is a much stronger property than continuity. However, under many circumstances the notions are equivalent:
\begin{thm}
	On a compact set in \(\C\), every continuous function is uniformly continuous.
\end{thm}
\begin{anki}
START
MathJaxCloze
Text: On a compact set in \(\C\), every continuous function is {{c1::uniformly continuous}}.
Tags: analysis complex_analysis complex_analyticity
<!--ID: 1624399037095-->
END
\end{anki}

Many of the theorems here are equivalent to the real-valued version, and hence unless the proof differs, proofs will be omitted as they can be found in a standard real analysis textbook.\\

\begin{thm}
	The continuous image of a compact set in \(\C\) is compact.
\end{thm}
This is actually a corollary via point-set topology.

\begin{cor}
	A continuous real-valued function on a compact set has a maximum and a minimum.
\end{cor}
This doesn't seem very relevant immediately, as the range of a complex function is not well-ordered. However, by considering the modulus of complex functions, we can obtain a similar result.

\begin{defn}[Maximum and Minimum of Complex Functions]
	Let \(f:\C\to \C\) be a complex-valued function. We say that \(f\) attains a \textbf{maximum} at \(z_0 \in \Omega \) if
	\begin{align*}
		\left| f(z) \right| \leq \left| f(z_0) \right| 
	\end{align*}for all \(z \in \Omega \).\\

	If instead
	\begin{align*}
		\left| f(z) \right| \geq \left| f(z_0) \right| 
	\end{align*} for all \(z\in \Omega \) then we say \(z_0\) is a \textbf{minimum} for \(f\).
\end{defn}
Notice that if \(f\) is continuous, then \(z\mapsto \left| f(z) \right| \) is continuous (by the triangle inequality).

\begin{anki}
START
MathJaxCloze
Text: Let \(f:\C\to \C\) be a complex-valued function. We say that \(f\) attains a **maximum** at \(z_0 \in \Omega \) if
{{c1::\(\begin{align*}
        	\left| f(z) \right| \leq \left| f(z_0) \right| 
        \end{align*}\)}}
for all \(z \in \Omega \).

If instead
{{c1::\(\begin{align*}
      \left| f(z) \right| \geq \left| f(z_0) \right| 
      \end{align*}\)}}
for all \(z\in \Omega \) then we say \(z_0\) is a **minimum** for \(f\).
Extra: Notice that if \(f\) is continuous, then \(z\mapsto \left| f(z) \right| \) is continuous (by the triangle inequality).
Tags: analysis complex_analysis complex_analyticity defn
<!--ID: 1624399037133-->
END
\end{anki}

\begin{cor}
	A continuous (real or complex-valued) function on a compact set \(\Omega\) is bounded and attains a maximum and minimum on \(\Omega\).
\end{cor}

\begin{anki}
START
MathJaxCloze
Text: A continuous (real or complex-valued) function on a compact set \(\Omega \) is {{c1::bounded}} and {{c1::attains a maximum and minimum on \(\Omega\)}}.
Tags: analysis complex_analysis complex_analyticity
<!--ID: 1624399037173-->
END
\end{anki}

\subsection{Curves in \(\C\)}
\label{sub:curves_in_c}

\begin{defn}[Continuous Curve]
	Let \(\Omega\subset \C\). A \textbf{continuous curve} in \(\Omega\) is a continuous function \(\gamma :[t_0,t_1]\to \Omega\) where \([t_0,t_1]\subset \R\).
\end{defn}

\begin{anki}
% Up to 5 consequences
START
Definition
Name: Continuous Curve in \(\C\)
Premise 1: \(\Omega\subset \C\), \([t_0,t_1] \subset \R\)
Consequence 1: \(\gamma \) is a continuous curve if \(\gamma:[t_0,t_1]\to \Omega\) is a continous function
Tags: analysis complex_analysis complex_analyticity defn
<!--ID: 1624399037216-->
END
\end{anki}

Because every complex function can be parametrized by \(F(x,y) = (u(x,y),v(x,y))\), it follows that we can write a curve \(\gamma \) by
\begin{align*}
	\gamma (t) = (x(t),y(t))
\end{align*}
for real-valued curves \(x,y\). This will prove useful later when we introduce complex differentiation.

\begin{defn}[Path-Connected]
	Let \(\Omega\subset \C\). Then \(\Omega\) is path-connected if every two points in  \(\Omega\) can be joined by a continuous curve in \(\Omega\).
\end{defn}

\begin{anki}
% Up to 5 consequences
START
Definition
Name: Path-Connected Set in \(\C\)
Premise 1: \(\Omega\subset \C\)
Consequence 1: \(\Omega\) is path-connected if \(\forall z_0,z_1 \in \Omega\), \(\exists \gamma:[t_0,t_1]\to \Omega\) with \(\gamma(t_i)=z_i\)
Tags: analysis complex_analysis complex_topology defn
<!--ID: 1624399037262-->
END
\end{anki}


\begin{thm}
	Let \(\Omega\subset \C\) be open, Then the following are equivalent:
	\begin{itemize}
		\item \(\Omega\) is connected
		\item Every two points in \(\Omega\) can be joined by a broken line consisting of (finitely many) horizontal and vertical line segments within \(\Omega\)
		\item \(\Omega\) is path-connected
	\end{itemize}
\end{thm}

\begin{anki}
% Up to 4 premises
% Up to 5 equivalences
START
Equivalence
Name: Connectedness of open \(\Omega\subset \C\)
Premise 1:  \(\Omega\subset \C\) open set
Equivalence 1: \(\Omega\) is connected
Equivalence 2: Every \(z_0,z_1 \in \Omega\) is joined by line consisting of (finitely many) horizontal and vertical line segments in \(\Omega\)
Equivalence 3: \(\Omega\) is path-connected
Tags: analysis complex_analysis complex_topology
<!--ID: 1624399037300-->
END
\end{anki}


\end{document}
