\documentclass{memoir}
\usepackage{notestemplate}

%\logo{~/School-Work/Auxiliary-Files/resources/png/logo.png}
%\institute{Rice University}
%\faculty{Faculty of Whatever Sciences}
%\department{Department of Mathematics}
%\title{Class Notes}
%\subtitle{Based on MATH xxx}
%\author{\textit{Author}\\Gabriel \textsc{Gress}}
%\supervisor{Linus \textsc{Torvalds}}
%\context{Well, I was bored...}
%\date{\today}

%\makeindex

\begin{document}

% \maketitle

% Notes taken on 

\begin{defn}[Convergence of Infinite Products]
	Let \(\left\{ a_n \right\}_{n=1}^{\infty}\) be a sequence of non-zero complex numbers. We say that the **infinite produc**t
	\begin{align*}
		\prod_{n=1}^{\infty} a_n 
	\end{align*}
	\textbf{converges absolutely} if
	 \begin{align*}
		\lim_{n \to \infty} a_n = 1
	\end{align*}
	and if the corresponding series
	\begin{align*}
		\sum_{n=1}^{\infty} \ln(a_n)
	\end{align*}
	converges absolutely.
\end{defn}
The transformation between the products of \(a_n\) and the sum of \(\ln(a_n)\) is natural, as exponentiation gives equality for partial sums/products.
\begin{anki}
TARGET DECK
Complex Qual::Complex Analysis
START
MathJaxCloze
Text: Let \(\left\{ a_n \right\}_{n=1}^{\infty}\) be a sequence of non-zero complex numbers. We say that the **infinite product**
\(\begin{align*}
  	\prod_{n=1}^{\infty} a_n 
  \end{align*}\)
**converges absolutely** if
{{c1::\( \begin{align*}
      	\lim_{n \to \infty} a_n = 1
        \end{align*}\)::coefficients}}
and if the corresponding power series
{{c1::\(\begin{align*}
        	\sum_{n=1}^{\infty} \ln(a_n)
        \end{align*}\)}} 
converges absolutely.
Extra: The transformation between the products of \(a_n\) and the sum of \(\ln(a_n)\) is natural, as exponentiation gives equality for partial sums/products.
Tags: analysis complex_analysis entire_meromorphic
<!--ID: 1626995479520-->
END
\end{anki}
An astute reader will notice that we need to be careful about our determination of \(\ln(a_n)\).
For finitely many \(\ln(a_n)\), we can take any determination without concern-- but as \(a_n\) approaches 1, we will face some issues.
Fortunately, there exists an \(N\) so that for all \(n\geq N\), we can express \(a_n = 1-\alpha_n\) for some \(\left| \alpha_n \right| <1\), for which the logarithm will remain well-defined for the rest of the sequence.
Of course, we ought to verify this transformation won't affect convergence.

\begin{lemma}
	Let \(\left\{ a_n \right\}_{n=1}^{\infty}\) be a sequence of complex numbers with \(a_n\neq 1\) for all \(n\). Suppose that \(\left\{ a_n \right\} \) is absolutely convergent:
	\begin{align*}
		\sum_{n=1}^{\infty} \left| a_n \right| 
	\end{align*}
	Then
	\begin{align*}
		\prod_{n=1}^{\infty} (1-a_n) 
	\end{align*}
	converges absolutely.
\end{lemma}
This lemma makes our study of infinite products convenient, as it allows us to reduce infinite products to infinite sums, a case which we are already familiar with.\\

If we consider infinite products of functions, we see something rather interesting.
The lemma above already gives us conditions for uniform convergence of
\begin{align*}
	\prod_{n=1}^{\infty} (1-g_n(z)) .
\end{align*}
Furthermore, we can leverage our knowledge of the logarithmic derivative to obtain more information about uniformly convergent products of functions.
\begin{lemma}
	Let \(\Omega \subset \C\) be an open set and \(\left\{ f_n \right\}_{n=1}^{\infty}\) a sequence of holomorphic functions on \(\Omega \).
	Consider the corresponding sequence \(\left\{ g_n \right\}_{n=1}^{\infty}\) determined so that \(f_n(z) = 1 + g_n(z)\), and suppose that
	\begin{align*}
		\sum_{n=1}^{\infty} g_n(z)
	\end{align*}
	converges uniformly and absolutely on \(\Omega \).
	Let \(K\subset \Omega \) be a compact subset so that \(f_n^{-1}(0)\cap K = \emptyset\) for all \(n\).
	Then the infinite product of \(f_n\) converges to a holomorphic function on \(\Omega \):
	\begin{align*}
		\prod_{n=1}^{\infty} f_n = f 
	\end{align*}
	and we have absolute and uniform convergence on \(K\) for the following sum:
	\begin{align*}
		\frac{f'}{f} = \sum_{n=1}^{\infty} \frac{f_n'}{f_n}.
	\end{align*}
\end{lemma}

\begin{anki}
START
MathJaxCloze
Text: Let \(\Omega \subset \C\) be an open set and \(\left\{ f_n \right\}_{n=1}^{\infty}\) a sequence of holomorphic functions on \(\Omega \).
Consider the corresponding sequence \(\left\{ g_n \right\}_{n=1}^{\infty}\) determined so that \(f_n(z) = 1 + g_n(z)\), and suppose that
\(\begin{align*}
  	\sum_{n=1}^{\infty} g_n(z)
  \end{align*}\)
converges {{c1::uniformly and absolutely on \(\Omega \)}}.
Let \(K\subset \Omega \) be a compact subset so that {{c1::\(f_n^{-1}(0)\cap K = \emptyset\)}} for all \(n\).
Then the infinite product of \(f_n\) {{c1::converges to a holomorphic function}} on \(\Omega \):
{{c1::\(\begin{align*}
        	\prod_{n=1}^{\infty} f_n = f 
        \end{align*}\)}}
and we have absolute and uniform convergence on \(K\) for the following sum:
{{c1::\(\begin{align*}
        	\frac{f'}{f} = \sum_{n=1}^{\infty} \frac{f_n'}{f_n}.
        \end{align*}\)::logarithmic derivative}} 
Tags: analysis complex_analysis entire_meromorphic
<!--ID: 1626995479539-->
END
\end{anki}


\begin{hw}[Blaschke Products]
	Let \(\left\{ a_n \right\} \subset D_1 \) be a sequence in the unit disc such that \(a_n\neq 0\) for all \(n\) and
	\begin{align*}
		\sum_{n=1}^{\infty} (1-\left| a_n \right| )
	\end{align*}
	converges.
	Show that the \textbf{Blaschke product}
	\begin{align*}
		f(z) = \prod_{n=1}^{\infty} \frac{\left| a_n \right| }{a_n}\cdot \frac{a_n - z}{1 - \overline{a}_n z} 
	\end{align*}
	converges uniformly on \(\left| z \right| \leq r<1\) for some fixed \(r\) and defines a holomorphic function on \(D_1\) having only the zeros \(\left\{ a_n \right\} \).
	Furthermore, show that  \(\left| f(z) \right| \leq 1\).\\

	(Hint: prove that for \(0<\left| a \right| <1\) and for some fixed \(r\) and  \(\left| z \right| \leq r<1\), the inequality
	\begin{align*}
		\left| \frac{a + \left| a \right| z}{(1-\overline{a}z)a} \right| \leq \frac{1+r}{1-r}
	\end{align*}
	holds)
\end{hw}
One can use Blaschke products to construct some unusual functions.
For example, if we choose \(a_n = 1-\sfrac{1}{n^2}\), then our resulting function is holomorphic on the unit disc with a zero at 1.
Modifying this construction allows us to construct a bounded holomorphic function \(f\) on \(D_1\) for which each point of the unit circle is a singularity.
(Note that this is a useful example that demonstrates that non-isolated singularities need not conform to our standard types of singularities-- we refer to this form of non-isolated singularity as a \textbf{natural boundary})

\begin{anki}
START
MathJaxCloze
Text: Let \(\left\{ a_n \right\} \subset D_1 \) be a sequence in the unit disc such that \(a_n\neq 0\) for all \(n\) and
\(\begin{align*}
  	\sum_{n=1}^{\infty} (1-\left| a_n \right| )
  \end{align*}\)
converges.
Then the **Blaschke product**
\(\begin{align*}
  	f(z) = \prod_{n=1}^{\infty} \frac{\left| a_n \right| }{a_n}\cdot \frac{a_n - z}{1 - \overline{a}_n z} 
  \end{align*}\)
  {{c1::converges uniformly}} on \(\left| z \right| \leq r<1\) for some fixed \(r\) and defines a {{c1::holomorphic function on \(D_1\) having only the zeros \(\left\{ a_n \right\} \)}}.
Furthermore,  \(\left| f(z) \right| \leq 1\).
Extra: One can use Blaschke products to construct some unusual functions. For example, if we choose \(a_n = 1-\sfrac{1}{n^2}\), then our resulting function is holomorphic on the unit disc with a zero at 1. Modifying this construction allows us to construct a bounded holomorphic function \(f\) on \(D_1\) for which each point of the unit circle is a singularity.
Tags: analysis complex_analysis entire_meromorphic
<!--ID: 1626995479562-->
END
\end{anki}

\subsection{Weierstrass Products}
\label{sub:weierstrass_products}

Our goal will be to show that we can use infinite products to classify entire functions.
We will classify a restricted class of entire functions, then utilize this to extend to the general case.

\begin{thm}[Non-vanishing Entire Functions]
	Let \(f\) be a non-vanishing entire function. Then there exists a second entire function \(g\) so that
	\begin{align*}
		f(z) = e^{g(z)}.
	\end{align*}
	Furthermore, \(g\) is unique up to an additive constant. That is, if
	\begin{align*}
		f(z) = \lambda e^{h(z)}
	\end{align*}
	for some \(\lambda \in \C\setminus\left\{ 1 \right\} \), then
	\begin{align*}
		h(z) = g(z) + \ln(\lambda ).
	\end{align*}
\end{thm}
This follows from our logarithmic derivatives directly.
We leave the verification as an exercise to the reader.

\begin{anki}
START
MathJaxCloze
Text: Let \(f\) be a non-vanishing entire function. Then there exists a second entire function \(g\) so that
{{c1::\(\begin{align*}
        	f(z) = e^{g(z)}.
        \end{align*}\)}}
Furthermore, \(g\) is unique up to an additive constant. That is, if
{{c1::\(\begin{align*}
      	f(z) = \lambda e^{h(z)}
        \end{align*}\)}}
for some \(\lambda \in \C\setminus\left\{ 1 \right\} \) and distinct entire function \(h\), then
{{c1::\(\begin{align*}
      	h(z) = g(z) + \ln(\lambda ).
        \end{align*}\)}}
Extra: This follows from our logarithmic derivatives directly.
Suppose \(f,g\) are two entire functions with the same zeros of equal multiplicity. Then it follows that
\(\begin{align*}
  	f(z) = g(z)e^{h(z)}
  \end{align*}\)
for some entire function \(h(z)\) (uniquely determined up to an additive constant).
It also follows that for \(h\) entire,
\(\begin{align*}
  	g(z) = 0 \iff g(z)e^{h(z)} = 0
  \end{align*}\)
with the same multiplicities.
Tags: analysis complex_analysis entire_meromorphic
<!--ID: 1626995479580-->
END
\end{anki}

Suppose \(f,g\) are two entire functions with the same zeros of equal multiplicity. Then it follows that
\begin{align*}
	f(z) = g(z)e^{h(z)}
\end{align*}
for some entire function \(h(z)\) (uniquely determined up to an additive constant).
It also follows that for \(h\) entire,
\begin{align*}
	g(z) = 0 \iff g(z)e^{h(z)} = 0
\end{align*}
with the same multiplicities.
Hence, we can construct a canonical entire function for a set of zeros of fixed multiplicity-- and thus all entire functions with those zeros of fixed multiplicity can be expressed in terms of the canonical form.\\

We will now give the intuition for the canonical form.
First, we should order our zeros by increasing absolute value, so that our zeros \(\left\{ z_n \right\} \) satisfy
\begin{align*}
	\left| z_1 \right| \leq \left| z_2 \right| \leq \ldots
\end{align*}
We'd like to define our function by the infinite product
\begin{align*}
	\prod_{n=1}^{\infty} \left( 1 - \frac{z}{z_n} \right)  
\end{align*}
but this product may not converge.
To resolve this, we introduce a convergence factor which does not introduce any zeros-- an exponential.
Our exponent in this term ought to be a polynomial whose degree is dependent on the term of the sequence (to ensure independence between terms).
In other words, our convergence factor will be of the form
\begin{align*}
	e^{w_n + \frac{1}{2}w_n^2 + \ldots + \frac{1}{n-1}w_n^{n-1}}
\end{align*}
where \(w_n = \sfrac{z}{z_n}\). We combine the convergence term with our original terms and write
\begin{align*}
	E_n(w) = (1-w) e^{w_n + \frac{1}{2}w_n^2 + \ldots + \frac{1}{n-1}w_n^{n-1}}.
\end{align*}
The polynomial in the exponent is chosen because
\begin{align*}
	\ln(1-z) = \sum_{n=1}^{\infty} -\frac{z^{n}}{n}
\end{align*}
and so
\begin{align*}
	\ln \left( \prod_{n=1}^{\infty} E_n\left(\sfrac{z}{z_n}\right)  \right)\\
	= \sum_{n=1}^{\infty} \ln \left(E_n\left(\sfrac{z}{z_n}\right) \right)\\
	= \sum_{n=1}^{\infty} \left( \frac{z}{z_n} + \ldots + \frac{1}{n-1}\frac{z}{z_{n-1}} \right) + \ln \left( 1 - \frac{z}{z_n}\right) \\
	= \sum_{n=1}^{\infty} \sum_{k=n}^{\infty} - \frac{1}{k} \left( \frac{z}{z_n} \right)^{k} 
\end{align*}
Of course, this identity is desirable as we want to show our infinite product converges absolutely.\\

Now we formally justify the work we've shown thus far.
First, we verify that convergence will occur as we expect.

\begin{lemma}
	If \(\left| w \right| \leq \frac{1}{2}\) then
	\begin{align*}
		\frac{\left| \ln E_n(w) \right| }{\left| w \right|^{n}} \leq 2.
	\end{align*}
	Furthermore, let a sequence of complex numbers \(\left\{ z_n \right\} \) be given with \(\left| z_1 \right| \leq \left| z_2 \right| \leq \ldots\). There exists a corresponding increasing sequence of positive integers \(\left\{ k_n \right\} \) so that, for all positive real \(a>0\) 
	\begin{align*}
		\sum_{n=1}^{\infty} \left( \frac{a}{\left| z_n \right| } \right)^{k_n} 
	\end{align*}
	converges. In fact, for every \(a>0\) there is a corresponding integer \(N_a>0\) so that for all \(k_n \geq N_a\):
	\begin{align*}
		\left( \frac{a}{\left| z_n \right| } \right)^{k_n} \leq \frac{1}{2^{k_n}}.
	\end{align*}
\end{lemma}

Now that we have introduced this sequence \(\left\{ k_n \right\} \), we can finally connect the various ideas we've constructed together into a well-defined product with the properties desired.

\begin{thm}[Weierstrass Product Theorem]
	Let \(\left\{ z_n \right\}_{n=1}^{\infty} \subset \C\setminus\left\{ 0 \right\} \) be a sequence of complex numbers in the complex plane with
	\begin{align*}
		\left| z_1 \right| \leq \left| z_2 \right| \leq \ldots
	\end{align*}
	and let \(\left\{ k_n \right\} \subset \N\) be the corresponding smallest sequence of positive integers so that for all positive real \(a>0\)
	\begin{align*}
		\sum_{n=1}^{\infty} \left( \frac{a}{\left| z_n \right| } \right)^{k_n} 
	\end{align*}
	converges. Define
	\begin{align*}
		P_n(z) &= \sum_{k=1}^{k_n-1} \frac{z^{k}}{k}\\
		E_n( \sfrac{z}{z_n}) &= \left( 1- \frac{z}{z_n} \right) e^{P_n(z / z_n)} .
	\end{align*}
	Then
	\begin{align*}
		\prod_{n=1}^{\infty} E_n( \sfrac{z }{z_n})   
	\end{align*}
	converges uniformly and absolutely on every disc \(D_a\), and hence defines an entire function with zeros exclusively at \(\left\{ z_n \right\} \).\\

	If \(\sup_{n} k_n = k< \infty\), then we take the canonical sequence to be \(k_n = \sup_{n} k_n\). In this case, we refer to \(E_n(\sfrac{z}{z_n})\) as the \textbf{elementary form} and the product
	 \begin{align*}
		 z^{m} \prod_{n=1}^{\infty} E_n \left( \sfrac{z}{z_n} \right)
	\end{align*}
	as the \textbf{Weierstrass product} and take it to be the canonical form for a set of zeros \(\left\{ z_n \right\} \subset \C\setminus\left\{ 0 \right\} \).
\end{thm}

\begin{anki}
START
MathJaxCloze
Text: **Weierstrass Product Theorem**
Let \(\left\{ z_n \right\}_{n=1}^{\infty} \subset \C\setminus\left\{ 0 \right\} \) be a sequence of complex numbers in the complex plane with
\(\begin{align*}
  	\left| z_1 \right| \leq \left| z_2 \right| \leq \ldots
  \end{align*}\)
and let \(\left\{ k_n \right\} \subset \N\) be the corresponding smallest sequence of positive integers so that for all positive real \(a>0\)
\(\begin{align*}
  	\sum_{n=1}^{\infty} \left( \frac{a}{\left| z_n \right| } \right)^{k_n} 
  \end{align*}\)
converges. Define
\(\begin{align*}
  	P_n(z) &= \sum_{k=1}^{k_n-1} \frac{z^{k}}{k}\\
  	E_n( \sfrac{z}{z_n}) &= \left( 1- \frac{z}{z_n} \right) e^{P_n(z / z_n)} .
  \end{align*}\)
Then
 {{c1::\(\begin{align*}
        	\prod_{n=1}^{\infty} E_n( \sfrac{z }{z_n})   
        \end{align*}\)}} 
converges {{c1::uniformly}} and {{c1::absolutely}} on every disc \(D_a\), and hence defines an entire function with zeros {{c1::exclusively at \(\left\{ z_n \right\} \)}}.

If \(\sup_{n} k_n = k< \infty\), then we take the canonical sequence to be \(k_n = \sup_{n} k_n\). In this case, we refer to \(E_n(\sfrac{z}{z_n})\) as the **elementary form** and the product
{{c1::\( \begin{align*}
        	 z^{m} \prod_{n=1}^{\infty} E_n \left( \sfrac{z}{z_n} \right)
        \end{align*}\)}}
as the \textbf{Weierstrass product} and take it to be the canonical form for a set of zeros \(\left\{ z_n \right\} \subset \C\setminus\left\{ 0 \right\} \).
Tags: analysis complex_analysis entire_meromorphic defn
<!--ID: 1626995479602-->
END
\end{anki}


\begin{cor}[Hadamard's Theorem]
	Every entire function \(f\) with zeros exactly at \(\left\{ z_n \right\}_{n=1}^{\infty}\subset \C\setminus\left\{ 0 \right\}\) and possibly at zero with order \(m\) can be written uniquely in the form
	\begin{align*}
	  	f(z) = e^{g(z)} z^{m} \prod_{n=1}^{\infty} E_n( \sfrac{z}{z_n})
	  \end{align*}
	where \(g\) is a polynomial of fixed degree \(\leq \sup_{n} k_n\) uniquely determined up to an additive constant.
\end{cor}

\begin{anki}
START
MathJaxCloze
Text: **Hadamard's Theorem**
Every entire function \(f\) with zeros exactly at \(\left\{ z_n \right\}_{n=1}^{\infty}\subset \C\setminus\left\{ 0 \right\}\) and possibly at zero with order \(m\) can be written uniquely in the form
{{c1::\(\begin{align*}
      	f(z) = e^{g(z)} z^{m} \prod_{n=1}^{\infty} E_n( \sfrac{z}{z_n})
        \end{align*}\)}}
where \(g\) is a polynomial of fixed degree \(\leq \sup_{n} k_n\) uniquely determined up to an additive constant.
Tags: analysis complex_analysis entire_meromorphic
<!--ID: 1626995479618-->
END
\end{anki}


\begin{proof}[Proof of Weierstrass Product Theorem]
	
\end{proof}

Of course, we can immediately leverage our classification of entire functions to classify meromorphic functions on \(\C\).

\begin{cor}[Classification of Meromorphic Functions on \(\C\)]
	Every function \(F\) which is meromorphic in the whole plane can be expressed uniquely by:
	\begin{align*}
		F(z) = f(z)\frac{g(z)}{h(z)}
	\end{align*}
	where \(f(z)\) is a non-vanishing entire function, \(g(z)\) is the canonical Weierstrass product corresponding to the zeros of \(F\) and \(h(z)\) is the canonical Weierstrass product corresponding to the poles of \(F\).
\end{cor}
We can use either this form or equivalently the form
\begin{align*}
	F(z) = e^{f(z)} \frac{g(z)}{h(z)}
\end{align*}
as our canonical choice depending on the context.

\begin{anki}
START
MathJaxCloze
Text: Every function \(F\) which is meromorphic in the whole plane can be expressed uniquely by:
{{c1::\(\begin{align*}
        	F(z) = f(z)\frac{g(z)}{h(z)}
        \end{align*}\)}} 
where \(f(z)\) is a non-vanishing entire function, \(g(z)\) is the canonical Weierstrass product corresponding to the {{c1::zeros of \(F\)}} and \(h(z)\) is the canonical Weierstrass product corresponding to the {{c1::poles of \(F\)}}.
Extra: We can use either this form or equivalently the form
\(\begin{align*}
  	F(z) = e^{f(z)} \frac{g(z)}{h(z)}
  \end{align*}\)
as our canonical choice depending on the context.
Tags: analysis complex_analysis entire_meromorphic
<!--ID: 1626995479636-->
END
\end{anki}
While this form is natural, we will explore other constructions for meromorphic functions later that will prove more fruitful.\\

First, we give an example of a few Weierstrass products and investigate the structure of \(\left\{ k_n \right\} \).

\begin{exmp}
	\begin{align*}
		\sin(\pi z) = \pi z \prod_{n=1}^{\infty} \left( 1- \frac{z^2}{n^2} \right)  .
	\end{align*}
	so
	\begin{align*}
		\frac{\pi ^2}{\sin^2(\pi z)} = \sum_{n=-\infty}^{\infty} \frac{1}{(z-n)^{2}}
	\end{align*}
\end{exmp}


% \printindex
\end{document}
