\documentclass{memoir}
\usepackage{notestemplate}

%\logo{~/School-Work/Auxiliary-Files/resources/png/logo.png}
%\institute{Rice University}
%\faculty{Faculty of Whatever Sciences}
%\department{Department of Mathematics}
%\title{Class Notes}
%\subtitle{Based on MATH xxx}
%\author{\textit{Author}\\Gabriel \textsc{Gress}}
%\supervisor{Linus \textsc{Torvalds}}
%\context{Well, I was bored...}
%\date{\today}

%\makeindex

\begin{document}

% \maketitle

% Notes taken on 

These notes provide material for a first course in complex analysis. It builds the complex field from the ground up, developing the main theory through Cauchy's theorem and calculus of residues. After this point, the material branches off into various topics which will provide many useful tools to aid in applying complex analysis to broader fields of mathematics. The textbook does assume the reader is familiar with the main theorems from real analysis, but it will nevertheless be stated when results from real analysis are utilized.\\

The lecture notes are based off a few main sources. The overall outline was developed by the author Gabriel Gress and was designed to be the most natural progression through the topics for someone aiming to develop a foundation to build upon. The completion of these notes will be sufficient preparation for a student to take a graduate-level course in complex analysis-- or to pass a graduate qualifying exam in complex analysis. The theorems and proofs are culminated from a few major sources: Ahlfors' \textit{Complex Analysis}, Stein and Shakarchi's \textit{Complex Analysis} and Lang's \textit{Complex Analysis}. My aim is to provide a more modern source on the deep results proven in these books, presented in a palatable and visual manner. I have also worked to adjust definitions and statements of theorems to better unify the underlying ideas that drive complex analysis and give a cleaner description.\\

These notes are still a work in progress. There are some notable sections and chapters at the end that have yet to be developed. Furthermore, many proofs and examples are omitted at the moment. Finally, I have not developed visual aids to best depict the main ideas.

% \printindex
\end{document}
