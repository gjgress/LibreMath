\documentclass{memoir}
\usepackage{notestemplate}

% \begin{figure}[ht]
%     \centering
%     \incfig{riemmans-theorem}
%     \caption{Riemmans theorem}
%     \label{fig:riemmans-theorem}
% \end{figure}

\begin{document}
The topology of \(\C\) is based on the metric \((\C,d)\), where \(d(z,z') = \left| z-z' \right| \).  Equivalently, the topology of \(\C\) is based on the notion of \textit{open discs}.
\begin{defn}[Open Disc]
	Let \(z_0 \in \C\) and \(r>0\). The \textbf{open disc \(D_r(z_0)\) of radius \(r\) centered at \(z_0\)} is the set
	\begin{align*}
		D_r(z_0) = \left\{z \in \C \mid \left| z-z_0 \right| <r \right\} .
	\end{align*}
The \textbf{closed disc \(\overline{D_r}(z_0)\) of radius \(r\) centered at \(z_0\)} is defined by
	\begin{align*}
	\overline{D_r}(z_0) = \left\{z \in \C \mid |z-z_0|\leq r \right\} .
	\end{align*}
\end{defn}
We may write \(D_r\) to denote the disc of radius \(r\) centered at \(0\).

\begin{anki}
TARGET DECK
Complex Qual::Complex Analysis
START
MathJaxCloze
Text: Let \(z_0 \in \C\) and \(r>0\). The **open disc \(D_r(z_0)\) of radius \(r\) centered at \(z_0\)** is the set
 {{c1::\(\begin{align*}
         	D_r(z_0) = \left\{z \in \C \mid \left| z-z_0 \right| <r \right\} .
         \end{align*}\)}}
The **closed disc \(\overline{D_r}(z_0)\) of radius \(r\) centered at \(z_0\)** is defined by
{{c1::\(\begin{align*}
        \overline{D_r}(z_0) = \left\{z \in \C \mid |z-z_0|\leq r \right\} .
        \end{align*}\)}}
Tags: analysis complex_analysis complex_topology defn
<!--ID: 1624844998803-->
END
\end{anki}

\begin{defn}[Isolated Points and Discrete Sets]
	Let \(\Omega \subset \C\) and let \(z_0 \in \Omega \). If there exists a disc \(D_{r}(z_0)\) for some \(r>0\) so that
	 \begin{align*}
		 D_r(z_0) \cap \Omega = \left\{ z_0 \right\}
	\end{align*}
then we say that \(z_0\) is \textbf{isolated}.\\

If every \(z_0 \in \Omega \) is isolated, then we say that \(\Omega \) is \textbf{discrete}.
\end{defn}
Later, we will see that the set of zeroes of a non-constant holomorphic function is discrete.

\begin{anki}
START
MathJaxCloze
Text: Let \(\Omega \subset \C\) and let \(z_0 \in \Omega \). If there exists a disc \(D_{r}(z_0)\) for some \(r>0\) so that
 {{c1::\( \begin{align*}
        	 D_r(z_0) \cap \Omega = \left\{ z_0 \right\}
        \end{align*}\)}}
then we say that \(z_0\) is **isolated**.

If every \(z_0 \in \Omega \) is isolated, then we say that \(\Omega \) is **discrete**.
Extra: If \(f\) is non-constant holomorphic on \(\Omega \), then its zeroes form a discrete set in \(\C\). Furthermore, if \(f=g\) on a non-discrete subset \(U\subset \Omega \) for \(f,g\) holomorphic functions, then \(f=g\) on \(\Omega \).
Tags: analysis complex_analysis complex_topology defn
<!--ID: 1625188420444-->
END
\end{anki}


\begin{defn}[Interior, Exterior, and Boundary Points]
	Let \(\Omega \subset \C\), and let \(z \in \C\). 
	\begin{itemize}
		\item \(z\) is an \textbf{interior point of \(\Omega \)} if there exists an \(r>0\) such that \(D_r(z) \subset \Omega \). The set of interior points of \(\Omega\) is called the \textbf{interior of \(\Omega\)}, denoted \textrm{int}\(\Omega\).
	\item \(z\) is an \textbf{exterior point of \(\Omega\)} if there exists an \(r>0\) such that \(D_r(z) \cap \Omega = \emptyset\). The set of exterior points of \(\Omega\) is called the \textbf{exterior of \(\Omega\)}, denoted \textrm{ext}\(\Omega\).
	\item \(z\) is a \textbf{boundary point of \(\Omega\)} if it is neither an interior point or exterior point. The set of boundary points of \(\Omega\) is called the \textbf{boundary of \(\Omega\)}, denoted \(\partial \Omega\).
	\end{itemize}
\end{defn}

\begin{anki}
START
MathJaxCloze
Text: Let \(\Omega\subset \C\), and let \(z \in \C\). 

* \(z\) is an **interior point of \(\Omega\)** if {{c1::there exists an \(r>0\) s.t. \(D_r(z) \subset \Omega\)}}. The {{c1::set of interior points of \(\Omega\)}} is called the **interior of \(\Omega\)**, denoted \textrm{int}\(\Omega\).
* \(z\) is an **exterior point of \(\Omega\)** if there exists an \(r>0\) s.t. \(D_r(z) \cap \Omega = \emptyset\). The set of exterior points of \(\Omega\) is called the **exterior of \(\Omega\)**, denoted \textrm{ext}\(\Omega\).
* \(z\) is a **boundary point of \(\Omega\)** if it is neither an interior point or exterior point. The set of boundary points of \(\Omega\) is called the **boundary of \(\Omega\)**, denoted \(\partial \Omega\).
Tags: analysis complex_analysis complex_topology defn
<!--ID: 1624844998844-->
END
\end{anki}

\begin{defn}[Closure]
	Let \(\Omega \subset \C\). The \textbf{closure} of \(\Omega\) is the union of the interior and the boundary of \(\Omega\). Alternatively, it is the complement of the exterior of \(\Omega\). We denote the closure of \(\Omega\) by \(\overline{\Omega}\).
\end{defn}

\begin{anki}
% Up to 5 consequences
START
Definition
Name: Closure
Premise 1: Let \(\Omega\subset \C\)
Consequence 1: The **closure** of \(\Omega\) is defined as \(\overline{\Omega} := \textrm{int} (\Omega) \cup \partial \Omega\).
Tags: analysis complex_analysis complex_topology defn
<!--ID: 1624844998884-->
END
\end{anki}

\begin{prop}
	Let \(\Omega\subset \C\), and let \(z \in \C\). Then \(z \in \overline{\Omega}\) if and only if for every \(r>0\), \(D(z,r) \cap \Omega \neq \emptyset\).
\end{prop}

\begin{anki}
START
MathJaxCloze
Text: Let \(\Omega\subset \C\), and let \(z \in \C\). Then \(z \in \overline{\Omega}\) if and only if {{c1::for every \(r>0\), \(D(z,r) \cap \Omega \neq \emptyset\)}}.
Extra: There is a partition of \(\C\) relative to any arbitrary \(\Omega\) given by
\begin{align*}
	\C = \textrm{int}\Omega \cup \textrm{ext}\Omega \cup \partial \Omega
\end{align*}
which in turn tells us that the interior of the complement is the exterior, and vice versa.
Tags: 
<!--ID: 1624844998921-->
END
\end{anki}


We have introduced a partition relative to an arbitrary \(\Omega \subset \C\), namely:
\begin{align*}
	\C = \textrm{int} \Omega \cup \partial \Omega \cup \textrm{ext} \Omega
\end{align*}
where each set is disjoint. Another partition:
\begin{align*}
	\C = \textrm{ext}\Omega^{c} \cup \partial \Omega^{c} \cup \textrm{int}\Omega^{c} 
\end{align*}
Therefore the interior of a set is the exterior of the complement, and the boundaries are equal.
\begin{defn}[Open and Closed Sets]
	Let \(\Omega\subset \C\). Then
	\begin{itemize}
		\item \(\Omega\) is \textbf{open} if \(\Omega\) contains none of its boundary points; \(\Omega \cap \partial \Omega = \emptyset\)
		\item \(\Omega\) is \textbf{closed} if \(\Omega\) contains all of its boundary points; \(\partial \Omega \subset \Omega\)
	\end{itemize}
\end{defn}

\begin{anki}
START
MathJaxCloze
Text: Let \(\Omega\subset \C\). Then
* \(\Omega\) is **open** if \(\Omega\) contains none of its boundary points; \(\Omega \cap \partial \Omega = \emptyset\)
* \(\Omega\) is **closed** if \(\Omega\) contains all of its boundary points; \(\partial \Omega \subset \Omega\)
Tags: analysis complex_analysis complex_topology defn
<!--ID: 1624844998958-->
END
\end{anki}

\begin{exmp}[Properties of Open/Closed Sets]
\begin{itemize}
	\item \(\Omega\) is open if and only if \( \textrm{int}\Omega = \Omega\) 
	\item \(\Omega\) is closed if and only if \(\overline{\Omega} = \Omega\) 
	\item The topology of \(\C\) is the set of all open sets
	\item Most sets in \(\C\) are neither open nor closed; there are only continuum many open/closed sets
	\item \(\emptyset,\C\) are both open and closed (the only two sets with this property). This implies that every other set in \(\C\) has a non-empty boundary
\end{itemize}
\end{exmp}

\begin{prop}
	Let \(\Omega\subset \C\). Then \(\Omega\) is open if and only if \(\Omega^{c}\) is closed.
\end{prop}

\begin{prop}[More Properties of Open/Closed Sets]
	\begin{itemize}
		\item The union of an arbitrary family of open sets is open
		\item The finite intersection of an arbitrary family of open sets is open
		\item The intersection of an arbitrary family of closed sets is closed
		\item The finite union of an arbitrary family of closed sets is closed
	\end{itemize}
\end{prop}

\begin{prop}
	Let \(A\subset B\subset \C\). Then 
	\begin{itemize}
		\item \( \textrm{int}A \subset \textrm{ int}B\) 
		\item \(\overline{A}\subset \overline{B}\)
	\end{itemize}
\end{prop}

\begin{anki}
START
MathJaxCloze
Text: Let \(A\subset B\subset \C\). Then 
* \( \textrm{int}A \subset \textrm{ int}B\
* \(\overline{A}\subset \overline{B}\)
Tags: analysis complex_analysis complex_topology
<!--ID: 1624844998996-->
END
\end{anki}

\begin{thm}
	Let \(\Omega\subset \C\). Then
	 \begin{itemize}
		\item \( \textrm{int}\Omega\) is open; it is the largest open subset of \(\Omega\)
		\item \( \overline{\Omega}\) is closed; it is the smallest closed superset of \(\Omega\).
	\end{itemize}
\end{thm}

\begin{cor}
	Let \(\Omega\subset \C\). Then
	\begin{itemize}
		\item \( \textrm{ext}\Omega\) is open
		\item \(\partial \Omega\) is closed
	\end{itemize}
\end{cor}

\begin{anki}
START
MathJaxCloze
Text: Let \(\Omega\subset \C\). Then
* \( \textrm{int}\Omega\) is open; it is the largest open subset of \(\Omega\)
* \( \overline{\Omega}\) is closed; it is the smallest closed superset of \(\Omega\).
* \( \textrm{ext}\Omega\) is open
* \(\partial \Omega\) is closed
Tags: analysis complex_analysis complex_topology
<!--ID: 1624844999032-->
END
\end{anki}


\end{document}
