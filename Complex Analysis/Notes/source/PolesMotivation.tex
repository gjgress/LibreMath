\documentclass{memoir}
\usepackage{notestemplate}

%\logo{~/School-Work/Auxiliary-Files/resources/png/logo.png}
%\institute{Rice University}
%\faculty{Faculty of Whatever Sciences}
%\department{Department of Mathematics}
%\title{Class Notes}
%\subtitle{Based on MATH xxx}
%\author{\textit{Author}\\Gabriel \textsc{Gress}}
%\supervisor{Linus \textsc{Torvalds}}
%\context{Well, I was bored...}
%\date{\today}

%\makeindex

\begin{document}

% \maketitle

% Notes taken on 

Cauchy's theorem gives us a rather detailed picture of holomorphic functions-- perhaps too detailed. While the properties we have shown are remarkable, they can be difficult to satisfy, and so holomorphic functions are too restrictive of a class. The problem is that many complex functions have isolated points-- singularities-- where the function is not well-defined. We can classify these singularities and deal with a larger class of complex functions-- meromorphic functions-- which behave nicely outside of these points, and are tameable near the singularities.

\subsection{Motivation}
\label{sub:motivation}

Before we discuss more generally singularities, it will be helpful to develop an intuition on a simpler class of complex functions.

\begin{defn}[Rational Functions]
	Let \(p,q \in \C[z]\) be two polynomials with no common factors. Then
	\begin{align*}
		r(z) = \frac{p(z)}{q(z)}
	\end{align*}
	is a \textbf{rational function} in the extended plane-- if \(q(z) = 0\), then we assign \(r(z) = \infty\). This function is continuous. We call the zeros of \(q\) the \textbf{poles} of \(r\), and the \textbf{order of a pole} is the order of the corresponding zero of \(q\).
\end{defn}
Note that the derivative
\begin{align*}
	r'(z) = \frac{p'(z)q(z) - q'(z)p(z)}{q(z)^2}
\end{align*}
exists only when \(q(z) \neq 0\), but as a rational function, it has the same poles as \(r(z)\), but with one additional order on the poles. Furthermore, note that if \(\lambda \in \C\) is a constant, then \(\lambda r\) has the same poles as \(r\) and hence the same order. Therefore, a rational function \(r\) of order \(p\) has \(p\) zeros and \(p\) poles, and every equation \(r(z) = \lambda \) has exactly \(p\) roots.\\

Finally, by partial fraction decomposition, there is a decomposition \(r(z) = g(z) + h(z)\), where \(g\) does not have a constant term, and \(h(z)\) is finite at infinity. The degree of \(g(z)\) is the order of the pole at \(\infty\) and so the polynomial is referred to the \textbf{singular part} of \(r(z)\) at \(\infty\). Another form that is useful is
\begin{align*}
	r(z) = g(z) + \sum_{j=1}^{q} g_j \left( \frac{1}{z-\beta_j} \right) 
\end{align*}
where \(\beta_j\) are the distinct finite poles of \(r(z)\).\\

With this definition alone, we can already introduce a powerful statement that follows from a theorem of Runge's that will give us uniform approximation of holomorphic functions:

\begin{thm}
	Let \(f\) be a holomorphic function on a neighborhood of a compact set \(K\). Then \(f\) can be approximated uniformly on \(K\) by rational functions with poles in \(K^{c}\).\\

	Furthermore, if \(K^{c}\) is connected, then \(f\) can be uniformly approximated by polynomials.
\end{thm}

\begin{anki}
TARGET DECK
Complex Qual::Complex Analysis
START
MathJaxCloze
Text: Let \(f\) be a holomorphic function on a neighborhood of a compact set \(K\). Then \(f\) can be {{c1::approximated uniformly}} on \(K\) by {{c1::rational functions with poles in \(K^{c}\)}}.

Furthermore, if \(K^{c}\) is connected, then \(f\) can be {{c1::uniformly approximated by polynomials}}.
Tags: analysis complex_analysis complex_analyticity
<!--ID: 1625608497996-->
END
\end{anki}


% \printindex
\end{document}
