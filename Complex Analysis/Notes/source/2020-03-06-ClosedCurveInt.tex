\documentclass{memoir}
\usepackage{notestemplate}

% \begin{figure}[ht]
%     \centering
%     \incfig{riemmans-theorem}
%     \caption{Riemmans theorem}
%     \label{fig:riemmans-theorem}
% \end{figure}

\begin{document}

\begin{defn}[Primitive]
	Let \(f\) be a function on the open set \(\Omega\). A \textbf{primitive} for \(f\) on \(\Omega\) is a function \(F\) that is holomorphic on \(\Omega\) such that \(F'(z) = f(z)\) for all \(z \in \Omega\).
\end{defn}
In other words, a primitive is an antiderivative. We use this nomenclature to prevent confusion with the real counterpart.

\begin{thm}[Newton-Leibniz Formula]
	If a continuous function \(f\) has a primitive \(F\) in \(\Omega\), and \(\gamma\) is a curve in \(\Omega\) that begins at \(z_0\) and ends at \(z_1\), then
	\begin{align*}
		\int_{\gamma} f(z) \,d z = F(z_1)-F(z_0)
	\end{align*}
\end{thm}
Note that while \(F\) and hence \(f\) is necessarily holomorphic along \(\gamma \), holomorphicity of \(f\) is not sufficient to get the statement-- not all holomorphic functions have primitives.

\begin{cor}
	If \(\gamma\) is a closed curve in an open set \(\Omega\), and \(f\) is continuous and has a primitive in \(\Omega\), then
	\begin{align*}
		\int_{\gamma} f(z) \,d z = 0 
	\end{align*}
\end{cor}
We can actually use this to show that a function does not have a primitive by integrating closed curves and getting non-zero values. In fact, if a function does integrate to \(0\) for every closed curve, then it must have an antiderivative.

\begin{exmp}
	Consider the function
	\begin{align*}
		f(z) = \frac{1}{z}
	\end{align*}
	on the domain \(\C\setminus \left\{ 0 \right\} \). If we take \(\gamma \) to be the boundary of the unit disc, that is, \(\gamma  = e^{it}\) on the interval \([0,2\pi ]\) then we observe that
	\begin{align*}
		\int_{\gamma } f \,d t = 2\pi i
	\end{align*}
	and hence \(f\) does not have a primitive in \(\C\setminus\left\{ 0 \right\} \). This exemplifies why we cannot define \(\log(z)\) on \(\C\setminus\left\{ 0 \right\} \)-- we have to specify a branch cut. If we integrate \(f\) over \(\gamma \) in the branch cut, we will indeed get zero once again, and hence \(f\) has a primitive within the branch cut.
\end{exmp}


\end{document}
