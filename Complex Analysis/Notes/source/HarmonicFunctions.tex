\documentclass{memoir}
\usepackage{notestemplate}

%\logo{~/School-Work/Auxiliary-Files/resources/png/logo.png}
%\institute{Rice University}
%\faculty{Faculty of Whatever Sciences}
%\department{Department of Mathematics}
%\title{Class Notes}
%\subtitle{Based on MATH xxx}
%\author{\textit{Author}\\Gabriel \textsc{Gress}}
%\supervisor{Linus \textsc{Torvalds}}
%\context{Well, I was bored...}
%\date{\today}

%\makeindex

\begin{document}

% \maketitle

% Notes taken on 07/14/21

\begin{defn}[Harmonic Functions]
	Let \(u \in C^2(\Omega)\) for \(\Omega\subset \R^{n}\) be a \textit{real-valued} multivariate function with continuous second-derivative. We say that \(u\) is \textbf{harmonic} if it satisfies
	\begin{align*}
		\Delta u := \sum_{i=1}^{n} \frac{\partial^2 u}{\partial x_i^2} = 0.
	\end{align*}
	We call the operator \(\Delta \) the \textbf{Laplacian} and the equation above \textbf{Laplace's equation}.
\end{defn}
Now suppose \(f = (u,v)\) is holomorphic on \(\Omega \subset \C\). We have shown that \(u,v \in C^{\infty}\) and the Cauchy-Riemann equations tell us that
\begin{align*}
	\frac{\partial u}{\partial x} = \frac{\partial v}{\partial y} \\
	\frac{\partial u}{\partial y} = - \frac{\partial v}{\partial x} .
\end{align*}
Applying \(\frac{\partial }{\partial x} \) or \(\frac{\partial }{\partial y} \) to both sides brings us to our first connection between holomorphic and harmonic functions.

\begin{thm}
	Let \(f=(u,v)\) be a holomorphic function. Then \(u,v\) are harmonic.
\end{thm}

Thus, we have seen that holomorphicity already implicitly requires two harmonic functions. But the Cauchy-Riemann equations give us additional structure on \(u,v\):

\begin{defn}[Conjugate Harmonic Function]
	If two harmonic functions \(u,v\) satisfy the Cauchy-Riemann equations
\begin{align*}
	\frac{\partial u}{\partial x} &= \frac{\partial v}{\partial y} \\
	\frac{\partial u}{\partial y} &= - \frac{\partial v}{\partial x} 
\end{align*}
	then \(v\) is said to be the \textbf{conjugate harmonic function} of \(u\). 
\end{defn}
In fact, the Cauchy-Riemann equations provide us with the most structure we could expect-- a conjugate harmonic function \(v\) is uniquely determined (up to an additive constant). But exercise caution, as harmonic conjugacy is symmetric, not antisymmetric-- if \(v\) is the conjugate harmonic function of \(u\), then \(u\) is the conjugate harmonic function of \(-v\).\\

\begin{anki}
TARGET DECK
Complex Qual::Complex Analysis
START
MathJaxCloze
Text: If two harmonic functions \(u,v\) satisfy {{c1::the Cauchy-Riemann equations
\(\begin{align*}
        	\frac{\partial u}{\partial x} &= \frac{\partial v}{\partial y} \\
        	\frac{\partial u}{\partial y} &= - \frac{\partial v}{\partial x} 
        \end{align*}\)}} 
	then \(v\) is said to be the **conjugate harmonic function** of \(u\). 
Extra: In fact, the Cauchy-Riemann equations provide us with the most structure we could expect-- a conjugate harmonic function \(v\) is uniquely determined (up to an additive constant). But exercise caution, as harmonic conjugacy is symmetric, not antisymmetric-- if \(v\) is the conjugate harmonic function of \(u\), then \(u\) is the conjugate harmonic function of \(-v\).
Tags: analysis complex_analysis complex_analyticity defn 
<!--ID: 1624560720227-->
END
\end{anki}

Recall the complex differential forms:
\begin{align*}
	dz = (dx, dy)\\
	d \overline{z} = (dx,-dy)
\end{align*}
defined so that
\begin{align*}
	dx = (\sfrac{1}{2},0) (dz + d \overline{z})\\
	dy = (0, \sfrac{1}{2}) (dz - d \overline{z})
\end{align*}
is satisfied. We also defined the complex differential operator by:
\begin{align*}
	\frac{\partial }{\partial z} &= \frac{1}{2}\left( \frac{\partial }{\partial x} , \frac{\partial }{\partial y}  \right)\\
	\frac{\partial }{\partial \overline{z}} &= \frac{1}{2} \left( \frac{\partial }{\partial x} , - \frac{\partial }{\partial y}  \right) 
\end{align*}
defined so that
\begin{align*}
	df = \frac{\partial f}{\partial z} \,d z + \frac{\partial f}{\partial \overline{z}} \,d \overline{z}
\end{align*}
is satisfied.

\begin{prop}
	\begin{align*}
		\Delta = 4 \frac{\partial }{\partial z} \frac{\partial }{\partial \overline{z}} = 4 \frac{\partial }{\partial \overline{z}} \frac{\partial }{\partial z} .
	\end{align*}
\end{prop}
This follows directly from computations.\\

\begin{anki}
START
MathJaxCloze
Text: Consider the complex differential forms:
\(\begin{align*}
  	dz &= (dx,dy) & \quad dx &= (\sfrac{1}{2},0)(dz + d\overline{z}) \\
  	d \overline{z} &= (dx,-dy) &\quad \,d y &= (0,\sfrac{1}{2})(dz - d \overline{z})
  \end{align*}\)
  and the corresponding operators
  \(\begin{align*}
    	\frac{\partial }{\partial z} &= \frac{1}{2}\left( \frac{\partial }{\partial x} , \frac{\partial }{\partial y}  \right)\\
    	\frac{\partial }{\partial \overline{z}} &= \frac{1}{2} \left( \frac{\partial }{\partial x} , - \frac{\partial }{\partial y}  \right) 
    \end{align*}\). Then we can write the Laplacian operator by:
    {{c1::\(\begin{align*}
	    	\Delta = 4 \frac{\partial }{\partial z} \frac{\partial }{\partial \overline{z}} = 4 \frac{\partial }{\partial \overline{z}} \frac{\partial }{\partial z} .
	    \end{align*}\)}} 
Tags: analysis complex_analysis harmonic_functions
<!--ID: 1626993604616-->
END
\end{anki}

Recall from real analysis that a harmonic function is defined by its value on the boundary:
\begin{thm}
	Let \(U\subset \R^n\) be a bounded open set. Let \(u:\overline{U}\to \R^{n}\) be a continuous function harmonic on \(U\). Suppose there exists a second continuous function \(v:\overline{U}\to \R^{n}\) harmonic on \(U\), so that \(u = v\) on \(\partial U\). Then \(u=v\) on \(\overline{U}\).
\end{thm}
This theorem has two main weaknesses. The first is that the set is required to be bounded. The second is that the functions must be continuous on the boundary.\\

Complex analysis provides us with tools that can resolve these weaknesses. The Riemann mapping theorem allows us to obtain an isomorphism from most unbounded sets into a bounded region-- although we must be cautious that the boundary curves correspond to each other properly. We will need to build more theory to handle discontinuities on the boundary.\\

One can show that if \(u\) is \textit{bounded everywhere on \(\partial U\)} and continuous on \(\partial U\)-- except for a finite number of points, then the theorem holds as expected. We omit the proof here.\\

\begin{anki}
START
MathJaxCloze
Text: Let \(U\subset \R^n\) be a bounded open set. Let \(u:\overline{U}\to \R^{n}\) be a continuous function harmonic on \(U\). Suppose there exists a second continuous function \(v:\overline{U}\to \R^{n}\) harmonic on \(U\), so that \(u = v\) on \(\partial U\). Then \(u=v\) on \(\overline{U}\).
Extra: If \(u\) is \textit{bounded everywhere on \(\partial U\)} and continuous on \(\partial U\)-- except for a finite number of points, then the theorem holds as expected.
Tags: analysis complex_analysis harmonic_functions
<!--ID: 1626993604635-->
END
\end{anki}

Recall that we can compute the complex derivative of a function \(f=(u,v)\) directly by
\begin{align*}
	f'(z) = 2 \frac{\partial u}{\partial z} .
\end{align*}
We have a converse provided that the underlying space is simply connected.
 \begin{prop}
	 Let \(\Omega \subset \C\) be a simply connected set, and let \(u\) be a real-valued harmonic function on \(U\). Then there exists a holomorphic function \(f \in \textrm{Hol}(\Omega )\) so that \(u= \textrm{Re}(f)\). Furthermore, this holomorphic function is unique up to an imaginary constant.
\end{prop}
To see that simply connectedness is necessary, consider \(\ln(\left| z \right| )\) on \(D_1 \setminus\left\{ 0 \right\} \).
\begin{anki}
START
MathJaxCloze
Text: Let \(\Omega \subset \C\) be a simply connected set, and let \(u\) be a real-valued harmonic function on \(U\). Then there exists a holomorphic function \(f \in \textrm{Hol}(\Omega )\) so that {{c1::\(u= \textrm{Re}(f)\)}}. Furthermore, this holomorphic function is {{c1::unique up to an imaginary constant}}.
Extra: To see that simply connectedness is necessary, consider \(\ln(\left| z \right| )\) on \(D_1 \setminus\left\{ 0 \right\} \).
Tags: analysis complex_analysis harmonic_functions
<!--ID: 1626993604652-->
END
\end{anki}
Uniqueness will follow from a slightly more general lemma:
\begin{lemma}
	Let \(\Omega \subset \C\) be a connected open set, and \(f,g\) holomorphic functions on \(\Omega \). If \(\textrm{Re}(f) = \textrm{Re}(g)\), then
	\begin{align*}
		f = (g, K)
	\end{align*}
	for some constant \(K \in \R\).
\end{lemma}

\begin{anki}
START
MathJaxCloze
Text: Let \(\Omega \subset \C\) be a connected open set, and \(f,g\) holomorphic functions on \(\Omega \). If \(\textrm{Re}(f) = \textrm{Re}(g)\), then
\(\begin{align*}
  	f = (g, K)
  \end{align*}\)
for some constant \(K \in \R\).
Tags: analysis complex_analysis complex_analyticity
<!--ID: 1626993604668-->
END
\end{anki}

We will need some simple properties of composition, which we give now as an exercise.
\begin{hw}
	Let \(f:U\to V\) be a holomorphic function. Suppose \(g\) is harmonic on \(V\). Then \(g\circ f \) is harmonic on \(U\).\\

	Suppose instead that \(g\) is harmonic on the set
	\begin{align*}
		V' = \left\{ \overline{z} \mid z \in V \right\} .
	\end{align*}
	Then the function
	\begin{align*}
		g(\overline{f(z)})
	\end{align*}
	is harmonic on \(U\).
\end{hw}

Now we prove the mean value theorem for harmonic functions via Cauchy's theorem.
\begin{thm}[Mean Value Theorem]
	Let \(\Omega \subset \C\) be an open set and \(u\) a harmonic function on \(\Omega \). Let \(z_0 \in U\) be given and consider \(\overline{D_r(z_0)}\subset \Omega \) for some \(r>0\) sufficiently small. Then
	\begin{align*}
		u(z_0) = \frac{1}{2\pi } \int_{0}^{2\pi } u(z_0+re^{i\theta }) \,d \theta . 
	\end{align*}
\end{thm}
Naturally, the maximum principle for harmonic functions falls out from this.

\begin{anki}
START
MathJaxCloze
Text: **Mean Value Theorem for Harmonic Functions**
Let \(\Omega \subset \C\) be an open set and \(u\) a harmonic function on \(\Omega \). Let \(z_0 \in U\) be given and consider \(\overline{D_r(z_0)}\subset \Omega \) for some \(r>0\) sufficiently small. Then
 {{c1::\(\begin{align*}
         	u(z_0) = \frac{1}{2\pi } \int_{0}^{2\pi } u(z_0+re^{i\theta }) \,d \theta . 
         \end{align*}\)}} 
Tags: analysis complex_analysis harmonic_functions
<!--ID: 1626993604684-->
END
\end{anki}

\begin{thm}[Maximum Principle of Harmonic Functions]
	Let \(\Omega \subset \C\) be a connected open set with \(u\) harmonic on \(\Omega \).
\begin{itemize}
	\item If \(u\) has a maximum at a point \(z_0 \in \Omega \), then \(u\) is constant.
	\item If \(u\) is continuous on \(\overline{\Omega }\) and not constant on \(U\), then \(u\) obtains its maximum on \(\partial U\).
\end{itemize}
\end{thm}

We can apply the maximum modulus principle in a lot of non-intuitive circumstances. In fact, the maximum modulus is not arbitrary-- we can bound the maximum modulus on a circle provided we have information on the maximum modulus of \(f\) on concentric circles near it.

\begin{anki}
START
MathJaxCloze
Text: **Maximum Principle of Harmonic Functions**
Let \(\Omega \subset \C\) be a connected open set with \(u\) harmonic on \(\Omega \).

* If \(u\) has a maximum at a point \(z_0 \in \Omega \), then {{c1::\(u\) is constant}}.
* If \(u\) is continuous on \(\overline{\Omega }\) and not constant on \(U\), then {{c1::\(u\) obtains its maximum on \(\partial U\)}}.
Tags: analysis complex_analysis harmonic_functions
<!--ID: 1626993604707-->
END
\end{anki}

\begin{thm}[Hadamard's Three-Circle Theorem]
	Let \(0<R_1<R_2<\infty\) be given and consider the annulus given by \(\Omega = D_{R_2} \setminus \overline{D_{R_1}}\). Let
	\begin{align*}
		\prescript{}{r}\|f\|_\infty := \textrm{max}_{z \in \partial D_r} \left| f(z) \right| .
	\end{align*}
	If a function \(f\) is holomorphic on the open annulus and continuous on the closure of the annulus, then
	\begin{align*}
		\prescript{}{r}\|f\|_{\infty} &\leq \prescript{}{R_1}\|f\|_{\infty}^{\alpha } \prescript{}{R_2}\|f\|_{\infty}^{1-\alpha }\\
		\alpha &= \frac{\ln(\sfrac{R_2}{r})}{\ln(\sfrac{R_2}{R_1})}.
	\end{align*}
	In other words, the maximum modulus of \(f\) on a circle \(\partial D_r\) is bounded by a logarithmic interpolation between the maximum modulus of \(f\) on concentric circles surrounding \(\partial D_r\).
\end{thm}

\begin{anki}
START
MathJaxCloze
Text: **Hadamard's Three-Circle Theorem**
Let \(0<R_1<R_2<\infty\) be given and consider the annulus given by \(\Omega = D_{R_2} \setminus \overline{D_{R_1}}\). Let
\(\begin{align*}
  	\prescript{}{r}\|f\|_\infty := \textrm{max}_{z \in \partial D_r} \left| f(z) \right| .
  \end{align*}\)
If a function \(f\) is holomorphic on the open annulus and continuous on the closure of the annulus, then
{{c1::\(\begin{align*}
      	\prescript{}{r}\|f\|_{\infty} &\leq \prescript{}{R_1}\|f\|_{\infty}^{\alpha } \prescript{}{R_2}\|f\|_{\infty}^{1-\alpha }\\
      	\alpha &= \frac{\ln(\sfrac{R_2}{r})}{\ln(\sfrac{R_2}{R_1})}.
        \end{align*}\)}}
In other words, the maximum modulus of \(f\) on a circle \(\partial D_r\) is {{c1::bounded by a logarithmic interpolation}} between the {{c1::maximum modulus of \(f\) on concentric circles surrounding \(\partial D_r\)}}.
Tags: analysis complex_analysis harmonic_functions
<!--ID: 1626993604724-->
END
\end{anki}


\subsection{Harmonic Functions on Annuli}
\label{sub:harmonic_functions_on_annuli}

The annulus is an interesting case because it is not simply connected, and hence we cannot utilize our previous theorems to induce a holomorphic function. However, it has plenty of structure that allows us to induce the holomorphic function a different way.

\begin{thm}[Correspondence Between Harmonic and Holomorphic Functions (Annulus)]
	Let \(0\leq R_1<R_2\) be given and consider the annulus given by \(\Omega = D_{R_2}\setminus \overline{D_{R_1}}\). Suppose \(u\) is a harmonic function on \(\Omega \). Then there exists a constant \(a \in \R\) and a holomorphic function \(f \in \textrm{Hol}(\Omega )\) such that
	\begin{align*}
		\textrm{Re}(f) = u - a \ln(\left| z \right| ).
	\end{align*}
\end{thm}
Note that we allow \(R_1=0\) or \(R_2=\infty\) to include the special cases of the punctured plane \(\C\setminus\left\{ 0 \right\} \) and the punctured disk \(D_1\setminus\left\{ 0 \right\} \).

\begin{proof}
	
\end{proof}

\begin{anki}
START
MathJaxCloze
Text: **Correspondence between Harmonic and Holomorphic on the Annulus**
Let \(0\leq R_1<R_2\) be given and consider the annulus given by \(\Omega = D_{R_2}\setminus \overline{D_{R_1}}\). Suppose \(u\) is a harmonic function on \(\Omega \). Then there exists a constant \(a \in \R\) and a holomorphic function \(f \in \textrm{Hol}(\Omega )\) such that
{{c1::\(\begin{align*}
        	\textrm{Re}(f) = u - a \ln(\left| z \right| ).
        \end{align*}\)::real part of f}} 
Extra: Note that we allow \(R_1=0\) or \(R_2=\infty\) to include the special cases of the punctured plane \(\C\setminus\left\{ 0 \right\} \) and the punctured disk \(D_1\setminus\left\{ 0 \right\} \).
This also gives us a simple way to compute integrals of harmonic functions on the annulus:
\(\begin{align*}
  	\frac{1}{2\pi }\int_{0}^{2\pi } u(r,\theta )\,d \theta  = a \ln(\left| z \right| ) + b.
  \end{align*}\)
Tags: analysis complex_analysis harmonic_functions
<!--ID: 1626993604740-->
END
\end{anki}

This also gives us a simple way to compute integrals of harmonic functions on the annulus:
\begin{cor}
	Let \(0\leq R_1<R_2\) be given and consider the annulus given by \(\Omega  = D_{R_2}\setminus\left\{ D_{R_1} \right\} \). Let \(u\) be harmonic on \(\Omega \). Then there exists constants \(a,b\in \C\) so that
	\begin{align*}
		\frac{1}{2\pi }\int_{0}^{2\pi } u(r,\theta )\,d \theta  = a \ln(\left| z \right| ) + b.
	\end{align*}
\end{cor}

Nevertheless, it will continue to prove difficult to unify our study of harmonic and holomorphic functions until we can find conditions under which we can extend harmonic functions on a domain that is not simply connected. We will prove a result that we will use to extend to the general case.
\begin{lemma}[Extension of Bounded Harmonic Functions]
	Let \(\Omega = D_1\setminus\left\{ 0 \right\} \) be the punctured disc and suppose that \(u\) is a bounded harmonic function on \(\Omega \). Then \(u\) extends to a harmonic function on \(D_1\).
\end{lemma}

This gives us hope that harmonic functions can extend on removable singularities, but first we will need one more lemma.

\begin{lemma}[Analytic Continuation for Harmonic Functions]
	Let \(\Omega \subset \C\) be a connected open set. Suppose \(u\) is a harmonic function on \(\Omega \) and \(f\) a holomorphic function on \(\Omega \). If \(u = \textrm{Re}(f)\) on some \(D_r\subset \Omega \) for \(r>0\), then \(u = \textrm{Re}(f)\) everywhere in \(\Omega \).
\end{lemma}

Finally, we obtain the desired result:
\begin{thm}[Correspondence Between Harmonic and Holomorphic Functions (General)]
	Let \(\Omega\subset \C \) be a simply connected open set, and let \(\left\{ z_i \right\}_{i=1}^{n}\) be a sequence of distinct points in \(\Omega \). Consider \(\Omega^{*}= \Omega \setminus] \left\{ z_i \right\}_{i=1}^{n}\), and let \(u\) be a real harmonic function on \(\Omega ^{*}\). Then there exists constants \(\left\{ a_i \right\}_{i=1}^{n}\subset \C\) and a holomorphic function \(f \in \textrm{Hol}(\Omega^{*})\) such that, for all \(z \in \Omega^{*}\),
	\begin{align*}
		\textrm{Re}(f(z)) = u(z) - \sum_{i=1}^{n} a_i \ln \left| z-z_i \right| .
	\end{align*}
\end{thm}
In fact, we can actually consider deleted discs \(D_{r_i}(z_i)\) and obtain the same result, in which case our theorem on the annulus is a special case of the above theorem.

\begin{anki}
START
MathJaxCloze
Text: Let \(\Omega\subset \C \) be a simply connected open set, and let \(\left\{ z_i \right\}_{i=1}^{n}\) be a sequence of distinct points in \(\Omega \). Consider \(\Omega^{*}= \Omega \setminus] \left\{ z_i \right\}_{i=1}^{n}\), and let \(u\) be a real harmonic function on \(\Omega ^{*}\). Then there exists constants \(\left\{ a_i \right\}_{i=1}^{n}\subset \C\) and a holomorphic function \(f \in \textrm{Hol}(\Omega^{*})\) such that, for all \(z \in \Omega^{*}\),
\(\begin{align*}
  	\textrm{Re}(f(z)) = u(z) - \sum_{i=1}^{n} a_i \ln \left| z-z_i \right| .
  \end{align*}\)
Extra: We can actually consider deleted discs \(D_{r_i}(z_i)\) and obtain the same result, in which case our theorem on the annulus is a special case of the above theorem.
Tags: analysis complex_analysis harmonic_functions
<!--ID: 1626993604756-->
END
\end{anki}


\begin{proof}
	
\end{proof}

%\subsection{Poisson Formula}
%\label{sub:poisson_formula}

%\subsection{Construction of Harmonic Functions}
%\label{sub:construction_of_harmonic_functions}



% \printindex
\end{document}
