\documentclass{memoir}
\usepackage{linalg}

% Contains notes on 12-02-19 to 12-04-19

\begin{document}
\chapter{Generalized Eigenvectors and Nilpotent Operators}
\label{cha:generalized_eigenvectors_and_nilpotent_operators}
\section{Null Spaces of Powers of an Operator}
\label{sec:null_spaces_of_powers_of_an_operator}
\begin{prop}
	Suppose \(T \in \mathcal{L}(V)\). Then
	\begin{align*}
		\left\{ 0 \right\} = \textrm{null} T^{0} \subset \textrm{null}T^{1} \subset \ldots\subset \textrm{null}T^{k} \subset \textrm{null}T^{k+1}\subset \ldots .
	\end{align*}
\end{prop}

\begin{prop}
	Suppose \(T \in \mathcal{L}(V)\). Suppose \(m\) is a nonnegative integer such that \( \textrm{null}T^{m} = \textrm{null}T^{m+1}\). Then
	\begin{align*}
		\textrm{null}T^{m} = \textrm{null}T^{m+1} = \ldots .
	\end{align*}
\end{prop}
\begin{prop}
	Suppose \(T \in \mathcal{L}(V)\), where \(V\) is a finite-dimensional vector space. Let \(n = \textrm{dim}V\). Then
	\begin{align*}
		\textrm{null}T^{n} = \textrm{null}T^{n+1} = \ldots .
	\end{align*}
\end{prop}
\begin{prop}
	Suppose \(T \in \mathcal{L}(V)\), where \(V\) is a finite-dimensional vector space.. Let \(n = \textrm{dim}V\). Then
	\begin{align*}
		V = \textrm{null}T^{n} \oplus \textrm{range} T^{n}.
	\end{align*}
\end{prop}

\section{Generalized Eigenvectors}
\label{sec:generalized_eigenvectors}
Some operators do not have enough eigenvectors to completely describe it. Thus we introduce a generalized eigenvectors which will allow us to describe operators more fully (no inner product means no guarantee of spectral theorem, hence this expands it).
\begin{defn}[Generalized Eigenvector]
	Suppose \(T \in \mathcal{L}(V)\) and \(\lambda\) is an eigenvalue of \(T\). A vector \(v \in V\) is called a \textbf{generalized eigenvector} of \(T\) corresponding to \(\lambda\) if \(v\neq 0\) and
	\begin{align*}
		(T-\lambda I)^{j}v = 0
	\end{align*}
	for some positive integer \(j\).
\end{defn}
\begin{defn}[Generalized Eigenspace]
	Suppose \(T \in \mathcal{L}(V)\) and \(\lambda \in F\). The \textbf{generalized eigenspace} of \(T\) corresponding to \(\lambda\), denoted \(G(\lambda,T)\), is defined to be the set of all generalized eigenvectors of \(T\) corresponding to \(\lambda\), along with the 0 vector.
\end{defn}
\begin{prop}
	Suppose \(T \in \mathcal{L}(V)\) where \(V\) is a finite-dimensional vector space and \(\lambda \in F\). Then \(G(\lambda,T) = \textrm{null}(T-\lambda I)^{ \textrm{dim}V}\).
\end{prop}

\begin{prop}
	Let \(T \in \mathcal{L}(V)\). Suppose \(\lambda_1,\ldots,\lambda_m\) are distinct eigenvalues of \(T\) and \(v_1,\ldots,v_m\) are corresponding generalized eigenvectors. Then \(v_1,\ldots,v_m\) is linearly independent.
\end{prop}

\section{Nilpotent Operators}
\label{sec:nilpotent_operators}

\begin{defn}
	An operator is called \textbf{nilpotent} if some power of it equals \(0\).
\end{defn}
\begin{prop}
	Suppose \(N \in \mathcal{L}(V)\) is nilpotent. Then \(N^{ \textrm{dim}V} = 0\).
\end{prop}
\begin{prop}
	Suppose \(N\) is a nilpotent operator on \(V\). Then there is a basis of \(V\) with respect to which the matrix of \(N\) has the form
	\begin{align*}
		\begin{bmatrix} 0 & & * \\ & \ddots & \\ 0 & & 0 \end{bmatrix} 
	\end{align*}
or in other words, all entries on and below the diagonal are \(0\).
\end{prop}

We can actually construct a basis that corresponds to a nilpotent operator. This will be important when we get to Jordan form.

\begin{prop}
	Suppose \(N \in \mathcal{L}(V)\) is nilpotent. Then there exist vectors \(v_1,\ldots,v_n \in V\) and nonnegative integers \(m_1,\ldots,m_n\) such that
	\begin{itemize}
		\item \(\left\{  N^{m_1}v_1,\ldots,N^2,Nv_1,v_1,N^{m_2}v_2,\ldots,N^{m_n}v_n,\ldots,v_n \right\}\) is a basis of \(V\) 
		\item \(N^{m_1+1}v_1 = \ldots = N^{m_n+1}v_n = 0\).
	\end{itemize}
\end{prop}

\section{Decomposition of an Operator}
\label{sec:decomposition_of_an_operator}

\subsection{Description of Operators on Complex Vector Spaces}
\label{subsec:description_of_operators_on_complex_vector_spaces}
\begin{prop}
	Suppose \(T \in \mathcal{L}(V)\) and \(p \in P(F)\). Then \( \textrm{null}p(T)\) and \( \textrm{range}p(T)\) are invariant under \(T\).
\end{prop}

\begin{thm}[Description of operators on complex vector spaces]
	Suppose \(V\) is a complex vector space and \(T \in \mathcal{L}(V)\). Let \(\lambda_1,\ldots,\lambda_m\) be the distinct eigenvalues of \(T\). Then
	\begin{itemize}
		\item \(V = G(\lambda_1,T) \oplus \ldots \oplus G(\lambda_m,T)\) 
		\item each \(G(\lambda_j,T)\) is invariant under \(T\) 
		\item each \((T-\lambda_jI)\mid_{G(\lambda_j,T)}\) is nilpotent
	\end{itemize}
\end{thm}
\begin{prop}
	Suppose \(V\) is a complex vector space and \(T \in \mathcal{L}(V)\). Then there is a basis of \(V \) consisting of generalized eigenvectors of \(T\).
\end{prop}

\subsection{Multiplicity of an Eigenvalue}
\label{subsec:multiplicity_of_an_eigenvalue}
\begin{defn}[Multiplicity]
	Suppose \(T \in \mathcal{L}(V)\). The \textbf{multiplicity} of an eigenvalue \(\lambda\) of \(T\) is defined to be the dimension of the corresponding generalized eigenspace \(G(\lambda,T)\).\\

	In other words, the multipicity of an eigenvalue \(\lambda\) of \(T\) equals \( \textrm{dim} \textrm{null}(T-\lambda I)^{ \textrm{dim} V}\).
\end{defn}
\begin{prop}
	Suppose \(V\) is a complex vector space and \(T \in \mathcal{L}(V)\). Then the sum of the multiplicities of all the eigenvalues of \(T\) equals \( \textrm{dim}V\).
\end{prop}

\subsection{Block Diagonal Matrices}
\label{subsec:block_diagonal_matrices}

\begin{prop}[First approximation to Jordan Form]
	Suppose \(V\) is a complex vector space and \(T \in \mathcal{L}(V)\). Let \(\lambda_1,\ldots,\lambda_m\) be the distinct eigenvalues of \(T\), with multiplicities \(d_1,\ldots,d_m\). Then there is a basis of \(V\) with respect to which \(T\) has a block diagonal matrix of the form
	\begin{align*}
		\begin{bmatrix} A_1 & & 0 \\ & \ddots & \\ 0 & & A_m \end{bmatrix} 
	\end{align*}
where each \(A_j\) is a \(d_j\)-by-\(d_j\) upper-triangular matrix of the form
\begin{align*}
	\begin{bmatrix} \lambda_j & &* \\ & \ddots & \\ 0 & & \lambda_j \end{bmatrix} 
\end{align*}
\end{prop}

\section{Jordan Form}
\label{sec:jordan_form}

\begin{defn}[Jordan Basis]
	Suppose \(T \in \mathcal{L}(V)\), where \(V\) is a finite-dimensional vector space. A basis of \(V\) is called a \textbf{Jordan basis}for \(T\) if with respect to this basis, \(T\) has a block diagonal matrix
	\begin{align*}
		\begin{bmatrix} A_1 & \ldots & 0 \\ \vdots & \ddots & \vdots \\ 0 & \ldots & A_p \end{bmatrix} 
	\end{align*} 
	where each \(A_j\) is an upper-triangular matrix of the form
\begin{align*}
	A_j = \begin{bmatrix} \lambda_j & 1 & \ldots & 0 \\ \vdots &\lambda_j & \ddots & \vdots \\ & & \ddots & 1 \\ 0 & \ldots & & \lambda_j \end{bmatrix} 
\end{align*}
The matrix is said to be in \textbf{Jordan form}, and the \(A_j\) are called \textbf{Jordan blocks}.
\end{defn}
Note: Some define the Jordan blocks with the \(1\)'s on the subdiagonal. This corresponds to reversing the order of the basis elements in each Jordan block.

\end{document}
