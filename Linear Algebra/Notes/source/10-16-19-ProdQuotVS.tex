\documentclass{memoir}
\usepackage{linalg}

% \begin{figure}[ht]
%     \centering
%     \incfig{riemmans-theorem}
%     \caption{Riemmans theorem}
%     \label{fig:riemmans-theorem}
% \end{figure}

\begin{document}
% \section{}	
\chapter{Special Vector Spaces}
\label{cha:products_and_quotients_of_vector_spaces}

\section{Products and Direct Sums}
\begin{defn}[Product of Vector Spaces]
	Suppose $V_1,\ldots,V_m$ are vector spaces over $F$.
	\begin{itemize}
		\item The \textbf{product} $V_1\times \ldots \times V_m$ is defined by
			\begin{align*}
				V_1 \times \ldots \times V_m = \left\{(v_1,\ldots,v_m) \mid v_1 \in V_1,\ldots,v_m \in V_m \right\} .
			\end{align*}
	\item Addition on $V_1\times \ldots \times V_m$ is defined by
		\begin{align*}
			(u_1,\ldots,u_m) + (v_1,\ldots,v_m) = (u_1+v_1,\ldots,u_m+v_m).
		\end{align*}
	\item Scalar multiplication on $V_1\times \ldots \times V_m$ is defined by
		\begin{align*}
			 \lambda(v_1,\ldots,v_m) = (\lambda v_1,\ldots, \lambda v_m).
		\end{align*}
	\end{itemize}
\end{defn}
\begin{lemma}
	Suppose $V_1,\ldots,V_m$ are vector spaces over $F$. Then $V_1\times \ldots \times V_m$ is a vector space over $F$.
\end{lemma}
\begin{lemma}[Dimension of a product is sum of dimensions]
	Suppose $V_1,\ldots,V_m$ are finite-dimensional vector spaces. Then $V_1 \times \ldots \times V_m$ is finite-dimensional and
	\begin{align*}
		\textrm{dim}(V_1\times \ldots\times V_m) = \textrm{dim}V_1 + \ldots + \textrm{dim}V_m .
	\end{align*}
\end{lemma}
\begin{lemma}[Products and direct sums]
	Suppose that $U_1,\ldots,U_m$ are subspaces of $V$. Define a linear map $\Gamma: U_1\times \ldots\times U_m \to U_1+\ldots+U_m$ by
	\begin{align*}
		\Gamma(u_1,\ldots,u_m) = u_1+\ldots+u_m.
	\end{align*}
	Then $U_1 + \ldots + U_m$ is a direct sum if and only if $\Gamma$ is injective (and thus invertible).
\end{lemma}
\begin{lemma}
	Suppose $V$ is finite-dimensional and $U_1,\ldots,U_m$ are subspaces of $V$. Then $U_1+\ldots+U_m$ is a direct sum if and only if
	\begin{align*}
		\textrm{dim}(U_1+\ldots+U_m) = \textrm{dim}U_1 + \ldots + \textrm{dim}U_m.
	\end{align*}
\end{lemma}
\section{Quotients of Vector Spaces}
\label{sec:quotients_of_vector_spaces}
\begin{defn}[v + U]
	Suppose $v \in V$ and $U$ is a subspace of $V$. Then $v+U$ is the subset of $V$ defined by 
	\begin{align*}
		v + U = \left\{v+u \mid u \in U \right\} .
	\end{align*}
\end{defn}
\begin{defn}[affine subset, parallel]
	An \textbf{affine subset} of $V$ is a subset of $V$ of the form $v+U$ for some $v \in V$ and some subspace $U$ of $V$.\\

	For $v \in V$ and $U$ a subspace of $V$, the affine subset $v+U$ is said to be \textbf{parallel} to $U$.
\end{defn}
\begin{defn}[Quotient space]
	Suppose $U$ is a subspace of $V$. Then the \textbf{quotient space} $V / U$ is the set of all affine subsets of $V$ parallel to $U$. In other words,
	\begin{align*}
		V / U = \left\{v + U \mid v \in V \right\}.
	\end{align*}
\end{defn}
\begin{lemma}
	Suppose $U$ is a subspace of $V$ and $v,w \in V$. Then the following are equivalent:
	\begin{itemize}
		\item $v - w \in U$ 
		\item $v + U = w + U$ 
		\item $(v+U) \cap (w+U) \neq \emptyset$
	\end{itemize}
\end{lemma}
\begin{defn}[addition and scalar multiplication on $V / U$]
	Suppose $U$ is a subspace of $V$. Then \textbf{addition} and \textbf{scalar multiplication} are defined on $V / U$ by 
	\begin{align*}
		(v+U) + (w + U) = (v + w) + U\\
		\lambda(v+U) = (\lambda v) + U
	\end{align*}
	for $v,w \in V$ and $\lambda in F$.
\end{defn}
\begin{thm}
	Suppose $U$ is a subspace of $V$. Then $V / U$ is a vector space.
\end{thm}
\begin{defn}[quotient map, $\pi$ ]
	Suppose $U$ is a subspace of $V$. The \textbf{quotient map} $\pi$ is the linear map $\pi:V\to V / U$ defined by
	\begin{align*}
		\pi(v) = v + U
	\end{align*}
	for $v \in V$.
\end{defn}
\begin{lemma}
	Suppose $V $ is finite-dimensional and $U$ is a subspace of $V$. Then
	\begin{align*}
		\textrm{dim}V / U = \textrm{dim}V - \textrm{dim}U.
	\end{align*}
\end{lemma}
Each linear map $T$ on $V$ induces a linear map $\hat{T} on V / ( \textrm{null}T)$, which we define:
\begin{defn}
	Suppose $T \in \mathcal{L}(V,W)$. Define $\hat{T}:V / ( \textrm{null}T) \to W$ by
	\begin{align*}
		\hat{T}(v+ \textrm{null}T) = Tv.
	\end{align*}
\end{defn}
\begin{lemma}[Null space and range of $\hat{T}$]
	Suppose $T \in  \mathcal{L}(V,W)$. Then
	\begin{itemize}
		\item $\hat{T}$ is a linear map from $V / ( \textrm{null}T)$ to $W$ 
			\item $\hat{T}$ is injective
			\item \textrm{range} $\hat{T}$ = \textrm{range} $T$ 
			\item $V / ( \textrm{null} T)$ is isomorphic to $ \textrm{range} T$.
	\end{itemize}
\end{lemma}
\section{Dual Space and Dual Map}
\label{sec:dual_space_and_dual_map}
\begin{defn}[Linear functional]
	A \textbf{linear functional} on $V$ is a linear map from $V$ to $F$. In other words, a linear functional is an element of $\mathcal{L}(V,F)$.
\end{defn}
\begin{defn}[Dual Space]
	The \textbf{dual space} of $V$, denoted $V'$, is the vector space of all linear functionals on $V $. In other words, $V' = \mathcal{L}(V,F)$.
\end{defn}
\begin{cor}
	Suppose $V $ is finite-dimensional. Then $V'$ is also finite-dimensional and $ \textrm{dim}V' = \textrm{dim}V$
\end{cor}
\begin{defn}[Dual Basis]
	If $v_1,\ldots,v_n$ is a basis of $V$, then the \textbf{dual basis} of $v_1,\ldots,v_n$ is the list $\varphi_1,\ldots,\varphi_n$ of elements of $V'$, where each $\varphi_j$ is the linear functional on $V$ such that
	\begin{align*}
		\varphi_j (v_j) = \begin{cases}
			1 & \text{if } k = j\\
			0 & \text{if } k \neq j
		\end{cases}
	\end{align*}
\end{defn}
\begin{lemma}
	Suppose $V$ is finite-dimensional. Then the dual basis of a basis of $V$ is a basis of $V'$.
\end{lemma}

\begin{defn}[dual map]
	If $T \in \mathcal{L}(V,W)$, then the \textbf{dual map} of $T$ is the linear map $T' \in \mathcal{L}(W',V')$ defined by $T'(\varphi) = \varphi \circ T$ for $\varphi \in W'$.
\end{defn}
\begin{lemma}[Algebraic properties of dual maps]
	\begin{itemize}
		\item $(S+T)' = S' + T'$ for all $S,T \in \mathcal{L}(V,W)$.
		\item $(\lambda T)' = \lambda T'$ for all $\lambda \in F$, and all $T \in \mathcal{L}(V,W)$.
		\item $(ST)' = T'S'$ for all $T \in \mathcal{L}(U,V)$ and all $S \in \mathcal{L}(V,W)$.
	\end{itemize}
\end{lemma}

\begin{defn}[Transpose]
The \textbf{transpose} of a matrix $A$, denoted $A^{T}$, is the matrix obtained from $A$ by interchanging the rows and columns. More specifically, if $A$ is an $m$-by-$n$ matrix, then $A^{T}$ is the $n$-by-$m$ matrix whose entries are given by
\begin{align*}
	(A^{T})_{k,j} = A_{j,k}
\end{align*}
\end{defn}
\begin{cor}
	If $A$ is $m\times n$, and $C$ is $n\times p$, then $(AC)^{T} = C^{T}A^{T}$
\end{cor}
These properties extend to their matrices.
\begin{thm}[The matrix of T' is the transpose of the matrix of T]
	Suppose $T \in \mathcal{L}(V,W)$. Then $\mathcal{M}(T') = (\mathcal{M}(T))^{T}$
\end{thm}
\end{document}
