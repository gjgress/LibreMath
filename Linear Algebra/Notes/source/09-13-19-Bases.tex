\documentclass{memoir}
\usepackage{notestemplate}

\begin{document}
\section{Bases}
\begin{defn}[Basis]
A \textbf{basis} of $V$ is a list of vectors in $V$ that is linearly independent and spans $V$.
\end{defn}
\begin{lemma}[Criterion for basis]
	A list $v_1,\ldots,v_n$ of vectors in $V$ is a basis of $V$ if and only if every $v \in V$ can be written uniquely in the form
	\begin{align*}
		v = a_1v_1+\ldots+a_nv_n
	\end{align*}
	where $a_1,\ldots,a_n \in F$.
\end{lemma}
\begin{thm}
	Every spanning list in a vector space can be reduced to a basis of the vector space.
\end{thm}
	Hence, every finite-dimensional vector space has a basis.
\begin{cor}
	Every linearly independent list of vectors in a finite-dimensional vector space can be extended to a basis of the vector space.	
\end{cor}

\begin{thm}[Soulmate space]
	Let $(V,F,+,\cdot )$ be a finite dimensional vector space. let $U\subset V$ be a subspace. Then there exists a $ W \subset V$ such that
	\begin{align*}
	U \oplus W = V
	\end{align*}
\end{thm}
Note that \(W\) is a finite-dimensional vector space.
\end{document}
