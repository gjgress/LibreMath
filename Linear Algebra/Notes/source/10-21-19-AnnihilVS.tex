\documentclass{memoir}
\usepackage{linalg}

% \begin{figure}[ht]
%     \centering
%     \incfig{riemmans-theorem}
%     \caption{Riemmans theorem}
%     \label{fig:riemmans-theorem}
% \end{figure}

\begin{document}
\section{The Null Space and Range of the Dual of a Linear Map}	
\begin{defn}[Annihilator]
	For $U \subset V$, the \textbf{annihilator} of $U $, denoted $U^{0}$, is defined by
	\begin{align*}
		U^{0} = \left\{\varphi\in V' \mid \varphi(u) = 0 \; \forall u \in U \right\} .
	\end{align*}
\end{defn}
\begin{exmp}
	Suppose $U$ is the subspace of $P(\R)$ consisting of all polynomial multiples of $x^2$. If $\varphi $ is the linear functional on $P(\R)$ defined by $\varphi(p) = p'(0)$, then $\varphi \in U^{0}.$
\end{exmp}
\begin{cor}
	Suppose $U \subset V$. Then $U^{0}$ is a subspace of $V'$.
\end{cor}
\begin{thm}[Dimension of Annihilator]
	Suppose $V$ is finite-dimensional and $U$ is a subspace of $V$. Then
	\begin{align*}
		\textrm{dim}U + \textrm{dim}U^0 = \textrm{dim}V.
	\end{align*}
\end{thm}
\begin{thm}[Null Space of $T'$ ]
	Suppose $V,W$ are finite-dimensional and $T \in \mathcal{L}(V,W)$. Then
	\begin{itemize}
		\item $\textrm{null}T' = ( \textrm{range}T)^{0}$
		\item $ \textrm{dim} \textrm{ null} (T') = \textrm{dim} \textrm{ null} (T) + \textrm{dim} (W) - \textrm{dim} (V)$
	\end{itemize}
\end{thm}
\begin{cor}
	Suppose $V,W$ are finite dimensional and $T \in \mathcal{L}(V,W)$. Then $T$ is surjective if and only if $T'$ is injective.
\end{cor}
\begin{thm}[Range of $T'$ ]
	Suppose $V,W$ are finite-dimensional and $T \in \mathcal{L}(V,W)$. Then
	\begin{itemize}
		\item $ \textrm{dim} \textrm{ range }T' = \textrm{dim} \textrm{ range }T $
		\item $\textrm{range }T' = ( \textrm{null} T)^0$
	\end{itemize}
\end{thm}
\begin{cor}
	Suppose $V,W$ are finite-dimensional and $T \in \mathcal{L}(V,W)$. Then $T$ is injective if and only if $T'$ is surjective.
\end{cor}

\subsection{Rank}
% Missing some definitions and theorems-- fill in later, Axler 3F
We will define row and column ranks initially, but keep in mind that this terminology will soon prove to be superfluous.
\begin{defn}[Row and Column Rank]
	Suppose \(A\) is an \(m\)-by-\(n\) matrix with entries in \(F\). The \textbf{row rank} of \(A\) is the dimension of the span of the rows of \(A\) in \(F^{1,n}\), and the \textbf{column rank} of \(A\) is the dimension of the span of the columns of \(A\) in \(F^{m,1}\).
\end{defn}
One can then prove that the column rank of \(\mathcal{M}(T)\) is equal to \(\textrm{dim}\textrm{Im}(T)\).

\begin{lemma}
Suppose $A \in F^{m,n}$. Then the row rank of $A$ equals the column rank of $A$.
\end{lemma}
This is what allows us to get rid of the superfluous terminology and create a simpler definition:
\begin{defn}[rank]
	The \textbf{rank} of a matrix \(A \in F^{m,n}\) is the column rank of \(A\). That is, the dimension of the span of the columns of \(A\) in \(F^{m,1}\), or the dimension of \(\textrm{Im}(A)\).
\end{defn}

\end{document}
