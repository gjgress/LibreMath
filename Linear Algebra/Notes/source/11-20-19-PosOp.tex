\documentclass{memoir}
\usepackage{linalg}

\begin{document}
% Covers 11-20-19 and 11-22-19 (probably)

\section{Positive Operators and Isometries}
\label{sec:positive_operators_and_isometries}
\begin{defn}[Positive Operator]
	An operator \(T \in \mathcal{L}(V)\) is called \textbf{positive} if \(T\) is self-adjoint and
	\begin{align*}
		\langle Tv, v \rangle \geq 0
	\end{align*}
	for all \(v \in V\).
\end{defn}
If \(V\) is a complex vector space, then \(T\) does not need to be self-adjoint (if an inner product is real for all values, then the operator must be self-adjoint; positivity implies real).
\begin{defn}[Square Root]
	An operator \(R\) is called a \textbf{square root} of an operator \(T\) if \(R^2 = T\).
\end{defn}
\begin{lemma}[Properties of Positive Operators]
	Let \(T \in \mathcal{L}(V)\). Then the following are equivalent:
	\begin{itemize}
		\item \(T\) is positive
		\item \(T\) is self-adjoint and all the eigenvalues of \(T\) are nonnegative
		\item \(T\) has a positive square root
		\item \(T\) has a self-adjoint square root
		\item there exists an operator \(R \in \mathcal{L}(V)\) such that \(T = R^{*}R\).
	\end{itemize}
\end{lemma}
\begin{cor}
	Each positive operator on \(V\) has a unique positive square root.
\end{cor}

Using more tools, one can further state that if \(V\) is a complex vector space and \(T\) an invertible operator, then \(T\) has a square root.

\subsection{Isometries}
\label{subsec:isometries}

We can impose one further condition on isomorphisms to strengthen the relation.
\begin{defn}[Isometry]
	An operator \(A \in \mathcal{L}(V)\) is an \textbf{isometry} if
	\begin{align*}
		\|Av\| = \|v\|
	\end{align*}
	for all \(v \in V\).
\end{defn}
Note that we haven't explicitly stated that isometries are isomorphisms. It turns out that this follows directly from the definition above.

\begin{prop}[Characterization of Isometries]
	Let \(A \in \mathcal{L}(V)\). The following are equivalent:
	\begin{itemize}
		\item \(A\) is an isometry
		\item \(\langle Au,Av \rangle = \langle u,v \rangle \) for all \(u,v \in V\)
		\item \(Ae_1,\ldots,Ae_n\) is orthonormal for all choices of orthonormal vectors \(e_1,\ldots,e_n \in V\)
		\item There exists an orthonormal basis \(e_1,\ldots,e_n \in V\) such that \(Se_1,\ldots,Se_n\) is orthonormal
		\item \(S^{*}S = SS^{*}= I\) 
		\item \(S^{*}\) is an isometry
		\item \(S\) is invertible and \(S^{-1} = S^{*}\)
	\end{itemize}
\end{prop}
Of course, this gives us that isometries are normal. This in fact gives us one more vital characterzation:
\begin{prop}
	Let \(V\) be a complex inner product space and \(A \in \mathcal{L}(V)\). Then there is an orthonormal basis of \(V\) consisting of eigenvalues of \(A\) with corresponding eigenvalues equal to \(\left| 1 \right| \) if and only if \(A\) is an isometry.
\end{prop}
\end{document}
