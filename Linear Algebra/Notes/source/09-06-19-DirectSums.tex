\documentclass{memoir}
\usepackage{notestemplate}

\begin{document}
\section{Sums of Subspaces}
Let $(V,F,+,\cdot )$ be a vector space, and let $U_1,\ldots,U_m$ be subspaces. Define 
\begin{align*}
	U_1 +\ldots+U_m := \left\{ u_1+\ldots+u_m \mid u_i \in U_i \text{ for } i=1,\ldots,m \right\}
\end{align*}
to be the sum of two subspaces.

\begin{exmp}
Let $V = \R^{3}$, $F=\R$ and consider
\begin{align*}
	U_1 = \left\{ (x,0,0) \mid x\in \R \right\} \in \R^{3}\\
	U_2 = \left\{ (0,y,0) \mid y \in \R \right\} \in \R^{3}
\end{align*}. Then
\begin{align*}
	U_1 + U_2 = \left\{ (x,y,0) \mid x,y\in \R \right\} .
\end{align*}
\end{exmp}

\begin{thm}[Sum of subspaces]
	If $U_1,\ldots,U_m \subseteq V$ are subspaces of $(V,F,+,\cdot )$, then $U_1+\ldots+U_m$ is the "smallest" subspace of $V$ that contains $U_1,\ldots,U_m$. That is, if there exists \(U'\subset V\) such that \(U_1,\ldots,U_m\subset U'\), then \(U_1+\ldots+U_m \subset U'\).
\end{thm}
\begin{proof}
	First we show that $U_1 + \ldots + U_m$ is a subspace of $V$. It is easy to see that $\overline{0} = \overline{0}+\ldots+\overline{0}$ and so the zero vector is in our space. We also know that
	\begin{align*}
		(u_1+\ldots+u_m) + (v_1+\ldots+v_m) = (u_1+v_1)+(u_2+v_2)+\ldots+(u_m+v_m) \in U_1+\ldots+U_m
	\end{align*}
	Finally, 

	\begin{align*}
		\lambda \cdot (u_1+\ldots+u_m) = \lambda \cdot u_1 + \ldots + \lambda \cdot u_m \in U_1 + \ldots+U_m.
	\end{align*}
	which gives us that \(U_1+\ldots+U_m\) is a subspace of \(V\) as desired.\\

	Now we show that $U_i \subseteq U_1 + \ldots + U_m$. Let $u_i \in U_i.$ Then 
	\begin{align*}
		u_2 = \overline{0} + \ldots + \overline{0} + u_i + \overline{0} + \ldots + \overline{0} \implies u_i \in U_1 + \ldots + U_m
	\end{align*}
	as desired.

Finally, we want to show the statement. Let $U' \subseteq V$ be a subspace with $U_1,\ldots,U_m \subseteq U'$, and let $u_1+\ldots+u_m \in U_1+\ldots+U_m$. Our work above gives us that $u_i \in U_i$. Because we have \(U_i \subset U'\), that implies that each \(u_i \in U'\), and so
\begin{align*}
	u_1 + \ldots+ u_m \in U'
\end{align*}
as desired.
\end{proof}

	\section{Direct sums}
	\begin{defn}[Direct sum]
		Let $U_1,\ldots,U_m$ be subspace of $(V,F,+,\cdot )$. If each $v \in U_1 + \ldots +U_m$ can be written in \textit{exactly} one way as
\begin{align*}
	v = u_1 + \ldots +u_m \quad u_i \in  U_i
\end{align*}
		 then we say $U_1 + \ldots + U_m$ is a \textbf{direct sum}, and denote it by
		\begin{align*}
		U_1 \oplus U_2 \oplus \ldots \oplus U_m .
		\end{align*}
	\end{defn}
	\begin{exmp}
	Let $V = \R^{3}$, $F = \R$, and consider $U_1 = \left\{ (x,y,0) \mid x,y \in R \right\} $ and $U_2 = \left\{ (0,0,z) | z \in \R \right\} $. Then $U_1 \bigoplus U_2$ is direct.\\

	But consider $U_3 = \left\{ (0,y,y) \mid y \in \R \right\} $. Then $U_1 + U_2 + U_3$ is NOT direct. 
\end{exmp}
\begin{lemma}[Criterion for direct sums]
	$U_1 \oplus \ldots \oplus U_m \iff \left[ u_1 + \ldots + u_m = 0 \implies u_1,\ldots,u_m=0\right]$ where $u_i \in U_i$.
\end{lemma}
	\begin{proof}
	First we will prove the forward direction. Observe that $\overline{0} = \overline{0} + \ldots + \overline{0}$. Because it is a direct sum, it is the only way to write the zero vector, and so each $u_i \in U_i$ must be $\overline{0}$. \\ 

	For the reverse direction, suppose that the only way to write $\overline{0} = u_1 + \ldots + u_m$ is to take $u_i = \overline{0}$ . \\ Let $v \in U_1 + \ldots + U_m$. Suppose that
	\begin{align*}
	v = u_1+\ldots+u_m \quad u_i \in U_i \\
	v = u'_1+\ldots+u'_m \quad u'_i \in U_i
	\end{align*}
	Subtraction yields $\overline{0}=(u_1 - u'_1) + \ldots + (u_m - u'_m)$. By hypothesis, each parentheses must be zero, and so \(u_i = u'_i\) and hence the representation of \(v\) is unique.
\end{proof}

\begin{lemma}[Direct sum of two subspaces]
Let $U,W \subseteq V$ be two subspaces. Then
\begin{align*}
	U\oplus W \iff U\cap W = \left\{ \vec{0} \right\} . 
\end{align*}
\end{lemma}

\end{document}
