\documentclass{memoir}
\usepackage{notestemplate}

%\logo{~/School-Work/Auxiliary-Files/resources/png/logo.png}
%\institute{Rice University}
%\faculty{Faculty of Whatever Sciences}
%\department{Department of Mathematics}
%\title{Class Notes}
%\subtitle{Based on MATH xxx}
%\author{\textit{Author}\\Gabriel \textsc{Gress}}
%\supervisor{Linus \textsc{Torvalds}}
%\context{Well, I was bored...}
%\date{\today}

\begin{document}

% \maketitle

% Notes taken on 05/19/21

\section{Existence of minimizers by direct method}
\label{sec:existence_of_minimizers_by_direct_method}

We will focus on finding the existence of minimizers to problems of ther form
\begin{align*}
	\inf_{u \in C(\overline{\Omega }) } \left\{\mathcal{F}(u) \mid u=u_0 \text{ on }\partial \Omega  \right\} \\
	\inf_{i} \left\{ \mathcal{F}(u) \mid g \right\} 
\end{align*}

For classes of the form
\begin{align*}
	\inf_{u \in \mathcal{A}} \mathcal{F}(u)
\end{align*}
where \(\mathcal{A}\) is some competition class, we start by picking a minimizing sequence (implicitly we are taking \(\mathcal{A}\neq \emptyset\)). We will pick the minimizing sequence so that:
\begin{align*}
	\lim_{j \to \infty} \mathcal{F}(u_j) = \inf_{\mathcal{A}} \mathcal{F}(u) < \infty\\
	\left\{ u_j \right\}_j \subset \mathcal{A}
\end{align*}
Then by sequential compactness, there exists a subsequence \(\left\{ \hat{u}_j \right\}_j\) for which \(\lim_{j \to \infty} \left\{ \tilde{u}_j \right\}_j \to u\) for some \(u\). Note that while \(u\) may not be in our space, the energy of \(u\) will be in our space.\\

Now we restrict our functions to be lower semicontinuous. Then we can show that if \(v_j \to v\), then \(\mathcal{F}(v) \leq \liminf_{j \to \infty} \mathcal{F}(v_j)\).\\

Now we need to show that the limit to which we converge to is within our competition class-- that is, \(u \in \mathcal{A}\). Once we have this, then we have that
\begin{align*}
	\inf_{\mathcal{A}} \mathcal{F} \geq \mathcal{F}(u) \leq \liminf_{j \to \infty} \mathcal{F}(\tilde{u}_j) = \lim_{j \to \infty} \mathcal{F}(\tilde{u}_j) = \inf_{\mathcal{A}} \mathcal{F}\\
	\implies \mathcal{F}(u) = \inf_{\mathcal{A}} \mathcal{F}
\end{align*}

\begin{exmp}
	Let \(\mathcal{F}(u) = \int_{0}^{1} u^2+((u')^2-1) \) and choose our competition class to be
	\begin{align*}
		\mathcal{A} = \left\{u \in C^{1}([0,1]) \mid u(0)=u(1)=0 \right\} 
	\end{align*}
	First, observe that \(\mathcal{F}(u) > 0\) for all \(u \in \mathcal{A}\). Positivity of the integral tells us that if \(\mathcal{F}(u) = 0\) then \(u=0\), but \(\mathcal{F}(0_{\mathcal{A}}) = 1\).\\

	However, we can see that \(\inf_{\mathcal{A}} \mathcal{F} = 0\). Choose \(u_j \in \textrm{Lip}(0,1), u_j' = \pm 1, 0\leq u_j\leq \frac{1}{2j}\). This satisfies \(u_j(0) = u_j(1) = 0\), and
	\begin{align*}
		F(u_j) = \int_{0}^{1} u_j^2 + \left( (\pm 1)^2 - 1 \right)^2 = \int_{0}^{1} (u_j)^2 \leq  \frac{1}{(4j)^2} 
	\end{align*}
	giving us the infimum.
\end{exmp}

\begin{exmp}
	Let
	\begin{align*}
		\mathcal{F}(u) = \int_{\R^{n}}\left| \nabla u \right|^2\\
		\mathcal{A} = \left\{u \in C^{1}(\R^{n}) \mid \int_{\R^{n}}\left| u \right|^{q}=1 \right\} \quad 1\leq q<\infty
	\end{align*}
	We can see that \(\mathcal{F}(u) > 0\) for all \(u \in \mathcal{A}\) (check). However, \(\inf_{\mathcal{A}} \mathcal{F} = 0\) ONLY when \(n\geq 3\) AND \(q = \dfrac{2n}{n-2}\). We can see this by picking any \(u_0 \in \mathcal{A}\) and setting \(u_\lambda (x) = \lambda^{\frac{n}{q}}u(\lambda x)\) :
	\begin{align*}
		\int_{\R^{n}}\left| u_\lambda  \right|^{q} = \int_{\R^{n}} \left| u_0 \right|^{q} = 1 \quad \forall \lambda >0\\
		\int_{\R^{n}}\left| \nabla u_\lambda  \right|^2 = \int_{\R^{n}} \left| \lambda^{\frac{n}{q}+1} \nabla u(\lambda x) \right|^2 \,d x\\
		= \lambda^{2\left( \frac{n}{q}+1 \right) } \int_{\R^{n}} \left| \nabla u(\lambda x) \right|^2 \,d x = \lambda^{2\left( \frac{n}{q}+1 \right) -n} \int_{\R^{n}} \left| \nabla u_0 \right|^2
	\end{align*}
	in which case, unless \(2\left( \frac{n}{q}+1 \right) -n = 0\), we get that the above integral goes to zero.
\end{exmp}

\begin{exmp}[Dirichlet problem]
	Let \(\Omega \subset \R^{n}\) be open and bounded, and let
	\begin{align*}
		\mathcal{F} = \left\{ \frac{1}{2} \int_{\Omega } \left| \nabla u \right|^2 \mid  \right\} \\
		\mathcal{A} = \left\{ u \in C^{1}(\overline{\Omega })\mid u = u_0 \text{ on }\partial\Omega  \right\} 
	\end{align*}
	It follows immediately that \(\mathcal{A}\neq \emptyset\) and \(\inf_{\mathcal{A}} \mathcal{F} < \infty\).\\

	Now consider a sequence of minimizers \(\left\{ u_j \right\}_j \subset C^{1}(\overline{\Omega })\) that converge to \(\inf_{\mathcal{A}} \mathcal{F}< \infty\). 
\end{exmp}

\begin{rmrk}[Properties of \(L^2(\Omega\) ]
	We define
	\begin{align*}
		L^2(\Omega ) := \left\{h: \Omega \to \R \mid h \text{ is measurable and } \int_\Omega \left| h \right|^2 < \infty\right\} .
	\end{align*}
	Note that we say that two functions are equivalent if they agree Lebesgue almost-everywhere on \(\Omega \).\\

	\(L^2(\Omega )\) is a complete separable metric space with a metric given by
	\begin{align*}
		d(h_1,h_2) = \|h_1 - h_2\|_{L^2(\Omega )} = \left( \int_\Omega \left| h_1 - h_2 \right|^2 \right)^{ \dfrac{1}{2}}\\
		\|h\|_{L^2(\Omega )} = \sqrt{ \langle h \mid h \rangle_{L^2(\Omega )}} \\
		\langle h \mid g \rangle_{L^2(\Omega )} = \int_\Omega fg
	\end{align*}
	That is, \(L^2(\Omega )\) is a Hilbert space. There exists an orthonormal basis \(\left\{ h_j \right\}_j \subset L^2(\Omega )\) such that
	\begin{align*}
		\langle h_i \mid h_j \rangle_{L^2} = \delta_{ij}\\
		\lim_{n \to \infty} \|h - \sum_{j=1}^{n} \langle h \mid h_j \rangle h_j\|_{L^2} = 0
	\end{align*}
	We also have the \textbf{Riesz representation theorem}. If \(\ell:L^2(\Omega ) \to \R\) is a (continous) linear functional, then there exists \(h \in L^2(\Omega )\) such that
	\begin{align*}
		\ell(g) = \langle h\mid g \rangle_{L^2(\Omega )} \quad \forall g \in L^2(\Omega )
	\end{align*}
	This only works if and only if there exists a \(C\) such that
	\begin{align*}
		\left| \ell(g) \right| \leq C \|g\|_{L^2}\quad \forall g \in L^2
	\end{align*}
	This is just another way of saying that linear functionals on bounded spaces are always continuous.
\end{rmrk}
	These properties combine to tell us that, if \(\left\{ h_j \right\}_j\) is a bounded sequence in \(L^2(\Omega )\), then it converges to some \(h \in L^2(\Omega )\) in some subsequence.\\

	However, without boundedness, we might not have traditional convegence-- even with subsequences. If \(\left\{ h_j \right\}_j\) is a sequence in \(L^2(\Omega )\) such that \(\int_\Omega \left| h_j \right|^2 = 1\), there does not necessarily exist \(h \in L^2(\Omega )\) such that \(h_j \to j\) in \(L^2(\Omega )\). To see this, consider the following example:
\begin{exmp}
	Let \(\Omega = (0,\pi )\), \(h_j(x) = \frac{2}{\pi } \sin(jx)\) with \(j \in \Z_+\). Observe that
	\begin{align*}
		\int_{0}^{\pi } \left| h_j \right|^2 = 1 \quad \forall j 
	\end{align*}
	By our properties from earlier, we have that for all \(g \in L^2(0,\pi )\), \(g = \lim_{N \to \infty} \sum_{j=1}^{N} \langle g \mid h_j \rangle h_j\) which implies
	\begin{align*}
		\int_{0}^{\pi } g^2 = \sum_{j=1}^{\infty} \langle g\mid h_j \rangle^2 \implies \lim_{j \to \infty} \langle g \mid h_j \rangle = 0. 
	\end{align*}
	Where the last parts follow from the fact that \(\int_{0}^{\pi } g(x)\sin(jx) \,d x \to 0 \) as \(j\) gets sufficiently large. But now we have an issue-- for any subsequence \(\left\{ \tilde{h}_j \right\}_j \subset \left\{ h_j \right\}_j\), there exists an \(h \in L^2(0,\pi )\) such that
	\begin{align*}
		0 = \lim_{j \to \infty} \int_{0}^{\pi } \left| h- \tilde{h}_j \right|^2 = \int_{0}^{\pi } h^2 + \int_{0}^{\pi } \tilde{h}_j ^2 - 2 \int_{0}^{\pi } h \tilde{h}_j = 2    
	\end{align*}
	and hence cannot exist.
\end{exmp}
In other words, \(L^2(\Omega )\) is not compact. However-- it is compact in the weak sense (sequentially compact when using weak convergence).

\begin{prop}
	If \(\left\{ h_j \right\}_j \subset L^2(\Omega )\) such that
	\begin{align*}
		\int_\Omega \left| h_j \right|^2 \leq C \quad \forall j
	\end{align*}
	then there exists an \(h \in L^2(\Omega )\) and a subsequence \(\left\{ \tilde{h}_j \right\}_j\) such that
	\begin{align*}
		\int_\Omega \tilde{h}_j g \to \int_\Omega h g \quad \forall g \in C^2(\Omega ).
	\end{align*}
	In other words, we have convergence in the inner product of our subsequence with a fixed test function \(g\).
\end{prop}

\begin{proof}[Proof of weak compactness]
	Take our sequence
	\begin{align*}
		\int_\Omega \left| h_j \right|^2 \leq C < \infty \quad \forall j
	\end{align*}
	Let \(Q = \left\{ \varphi_k \right\}_{k=1}^{\infty}\) be a countable dense subset of \(L^2(\Omega )\). Then for all fixed \(k\), we have
	\begin{align*}
		\int_\Omega h_j \varphi_k \subset \R \text{ bounded}\\
		\left| \int_\Omega h_j \varphi_k \right| \subset \|h_j\|_{L^2} \|\varphi_k\|_{L^2} \leq C \|\varphi_k\|_{L^2(\Omega )}
	\end{align*}
	In other words, we have our weak convergence on our countable dense subset, as the boundedness gives us a convergent subsequence.\\

	By countable density, we have that there exists a subsequence \(\left\{ \tilde{h}_j \right\}_j\) such that
	\begin{align*}
		\lim_{j \to \infty} \int_\Omega \tilde{h}_j \varphi_k \to \ell_k\quad \forall k \in \Z_{+}
	\end{align*}
	 and we can take \(\ell:Q\to \R\) to be
	 \begin{align*}
		 \ell(\varphi_k) = \lim_{j \to \infty} \int_\Omega \tilde{h}_j \varphi _k = \ell_k\\
		 \left| \ell(\varphi _k) \right| \leq C \|\varphi_k\|_{L^2(\Omega )}
	 \end{align*}
	 In other words, \(\ell\) is a bounded linear functional. Then we can extend \(\ell:L^2(\Omega )\to \R\) by defining
	 \begin{align*}
		 \ell(\varphi ) = \lim_{k \to \infty} \ell(\tilde{\varphi }_k)\\
		 \lim^{L^2}_{k \to \infty} \left\{ \tilde{\varphi_k }_k \right\} = \varphi 
	 \end{align*}
	 This is well-posed, and moreover \(\left| \ell\left( \varphi  \right) \leq C \|\varphi \|_{L^2} \right| \) for all \(\varphi  \in L^2\) by the density. \\

	 We can apply the Riesz representation theorem to get that there exists an \(h \in L^2(\Omega )\) such that \(\ell(\varphi ) = \int_\Omega \varphi h\) because
	 \begin{align*}
	 	\int_\Omega h \varphi_k = \ell_k = \lim_{j \to \infty} \int_\Omega \tilde{h}_j \varphi_k \implies\\
		\lim_{j \to \infty} \langle \tilde{h}_j \mid \varphi_k \rangle = \langle h \mid \varphi_k \rangle \quad \forall k \in \Z_+
	 \end{align*}
	 and by the density of \(Q = \left\{ \varphi_k \right\}_k\) in \(L^2\), we have convergence in the whole space.
\end{proof}

\end{document}
