\documentclass[course=Introduction\ to\ Functional\ Analysis, semester=Fall\ 2021]{syllabustemplate}
\usepackage{syllabustemplate}

\def\classtime{N/A}
\def\classroom{N/A}
\def\officeloc{N/A}
\def\officehours{N/A}
\def\webpage{N/A}

\begin{document}


\chapterstyle{article-4}  % alternative styles are defined in latex-custom-kjh/needs-memoir/

\begin{center}
\includegraphics[height=0.15\textheight]{~/School/Auxiliary-Files/resources/png/unmlogo.png}
\end{center}
{\let\newpage\relax\maketitle}
\thispagestyle{empty}
\vspace{0.15\textheight}
\begin{center}
	\hrule\vspace{1em}
	\noindent
	\begin{tabularx}{\hsize}{llXll}
		\textbf{Email:} & \email & & \textbf{Webpage:} & \webpage \\
		\textbf{Office Hours:} & \officehours & & \textbf{Class Time:} & \classtime\\
		\textbf{Office:} & \officeloc & & \textbf{Class Location:} & \classroom%\\
		%\textbf{TA:} & \officeloc & & %TA email? & 
\end{tabularx}
\vspace{1em}
	\hrule
\end{center}

\pagebreak

\section{Course Description}
\label{sec:course_description}

Functional analysis is the study of (potentially infinite-dimensional) vector spaces and linear functions on them. We will begin by discussing the definition of Banach (normed) and Hilbert (inner product) spaces, then get into the meat of linear operators-- boundedness, extensions, etc. Functional vector spaces are pretty key examples throughout this and so we will spend time constructing these examples. The goal will be to build up to spectral theory, a subfield of functional analysis that concerns with defining a more abstract form of "eigenvalue" for general vector spaces. Spectral theory proves ubiquitous because we can learn a lot about differential operators and more on function spaces by studying their spectrum. 

\section{Course Objectives}
\label{sec:course_objectives}
At the end of this course you should
\begin{enumerate}
	\item Be familiar with --
	\item Be able to state, understand, and apply --
\end{enumerate}
\section{Texts}
\label{sec:texts}
\textit{Introductory Functional Analysis}, Kreyszig (edition).\\

\textit{Applied Analysis}, Hunter (edition).

\section{Course Policies}
\label{sec:course_policies}

N/A

%\subsection{Grading Policy}
%\label{sub:grading_policy}
%
%\subsection{Attendance Policy}
%\label{sub:attendance_policy}
%
%\subsection{Email Policy}
%\label{sub:email_policy}
%
%\subsection{Make-up Exam Policy}
%\label{sub:make_up_exam_policy}
%
%\subsection{Academic Dishonesty Policy}
%\label{sub:academic_dishonesty_policy}
%
%\subsection{Diversity within the Classroom}
%\label{sub:diversity_within_the_classroom}
%
%\subsection{Additional Learning Needs}
%\label{sub:additional_learning_needs}

\pagebreak
\section{Class Schedule}
\label{sec:class_schedule}

{\parindent0pt

\subsection{Week 1, 10/27 - 11/3: Preliminaries: Metric Spaces}
\label{sub:week_1_10_27_-_11_3_preliminaries_metric_spaces}

Be able to prove:
\begin{itemize}
	\item Whether a bilinear map is a metric
	\item Other formulations of the triangle inequality
	\item Whether simple metric spaces are separable
	\item The equivalence between closure and limits of convergent sequences
	\item The equivalence between topological continuous functions and sequential continuous functions
	\item Basic convergence properties about Cauchy and convergent sequences
\end{itemize}

Be able to state, understand, and apply:
\begin{itemize}
	\item Definition of a metric and know some standard metrics
	\item Holder inequality for sums (and the Cauchy-Schwarz inequality for sums)
	\item Minkowski inequality for sums
	\item Definitions and basic properties of open/closed sets and their variants (interior, closure)
	\item Definition of continuity
	\item Definition of dense/separable sets
	\item Definitions and basic properties of Cauchy and convergent sequences
	\item Definition of completeness
\end{itemize}

Be familiar with:
\begin{itemize}
	\item The definition of a sequence space \(l^{p}\)
	\item The definition of a function space \(C[a,b]\)
	\item Important examples of open/closed sets (\(X\), \(\emptyset\), \(B\), \(\overline{B}\), etc.)
	\item The proofs of non-separability of \(l^{\infty}\) and separability of \(l^{p}\)
	\item Important examples of dense/separable and complete spaces
\end{itemize}
\vspace{1em}

\textbf{Read:}\\
Chapter 1.1-1.4 of Kreyszig (Primary)\\
Chapter 1 of Hunter (Supplementary)\\

\textbf{Turn in:}\\
Exercise 1.1 \#12 of Kreyszig\\
Exercises 1.2 \#4, \#13, \#15 of Kreyszig\\
Exercises 1.3 \#8, \#12 of Kreyszig\\
Exercise 1.4 \#1 of Kreyszig\\
\hrule
\subsection{Week 2, 11/8 - 11/12: Preliminaries: More Metric Spaces and Vector Spaces}
\label{sub:week_2_11_3_11_10_preliminaries_more_metric_spaces_and_vector_spaces}


Be able to prove:
\begin{itemize}
	\item That a given set is (not) compact, by both open covers and sequential compactness
	\item Facts about subsets of compact/complete/closed sets
	\item Facts about subspaces of vector spaces
	\item That a given basis is linearly (in)dependent
\end{itemize}

Be able to state, understand, and apply:
\begin{itemize}
	\item Definition of uniform convergence
	\item Both definitions of compactness
	\item Heine-Borel Theorem
	\item Bolzano-Weierstrass theorem
	\item Theorems relating continous functions and compact sets
	\item Definition of vector spaces
	\item Definition of linear independence and bases (and dimension)
	\item Definition of quotient spaces
\end{itemize}

Be familiar with:
\begin{itemize}
	\item Important examples of (in)complete spaces (with intuition as to why)
	\item The intuition as to how completions of metric spaces are constructed
	\item The proofs that \(\left[ 0,1 \right] \) is compact and \(\left( 0,1 \right) \) is not compact
	\item The equivalence of sequential compactness and compactness in metric spaces
	\item Examples of finite and infinite-dimensional vector spaces (including quotient spaces)
\end{itemize}
\vspace{1em}

\textbf{Read:}\\
Chapter 1.5-1.6, 2.1 of Kreyszig (Required)\\
Chapter 1.7 of Hunter (Required)
Chapter 1 of Hunter (Optional)\\

\textbf{Turn in:}\\
Exercises 1.5 \#13-15 of Kreyszig\\
Exercises 1.6 \#6 of Kreyszig\\
Exercises 2.1 \#5, \#14 of Kreyszig\\
\hrule
\subsection{Week 3, 11/15 - 11/19: Preliminaries: More Metric Spaces and Vector Spaces}
\label{sub:week_2_11_3_11_10_preliminaries_more_metric_spaces_and_vector_spaces}


Be able to prove:
\begin{itemize}
	\item 
	\item 
\end{itemize}

Be able to state, understand, and apply:
\begin{itemize}
	\item 
	\item 
\end{itemize}

Be familiar with:
\begin{itemize}
	\item 
	\item 
\end{itemize}
\vspace{1em}

\textbf{Read:}\\
Chapter 2.2-2.4 of Kreyszig (Required)\\
Chapter 5 of Hunter (Optional)\\

\textbf{Turn in:}\\
Exercises 2.2 \#8, 15\\
Exercises 2.3 \#9, 15\\
Exercises 2.4 \#1, 6\\
\hrule
}


% \printindex
\end{document}
